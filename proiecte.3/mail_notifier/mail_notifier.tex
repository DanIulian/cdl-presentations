\documentclass{beamer}

\usepackage[romanian]{babel}
\mode<presentation>
{ \usetheme{Rochester}} 

\title{Mail Notifier}
\subtitle{Propunere proiect CDL}
\institute{ROSEdu}
\author{Andrei Soare \\ \texttt{andrei.soare@rosedu.org}}

\begin{document}

\setbeamertemplate{frametitle continuation}[from second]
\setbeamertemplate{footline}[frame number]

\frame{\titlepage}

\begin{frame}{Overview}
\begin{itemize}
\item Program lightweight care verifica regulat daca primim mailuri
\item Configurabil pentru orice tip de mail\vspace{1cm}\pause
\item Cam comun, stiu ... but wait and see
\end{itemize}
\end{frame}

\begin{frame}{Componente}
\begin{itemize}
\item daemon care ruleaza in background si face check pe servere
\item interfata grafica: widget in gnome-panel
\item fisiere criptate de configurare
\end{itemize}
\end{frame}

\begin{frame}[allowframebreaks]
\frametitle{Functionalitate}
\begin{itemize}
\item initial: linie de comanda
\item suport POP3 si IMAP
\item suport provideri cunoscuti de mail (google, yahoo, aol, etc) - de obicei
ofera deja un API care se integreaza usor in aplicatie
\item configurare noi conturi real-time:\\fiecare cont va avea un thread separat
\item downloadat preview mail si afisata notificare pe desktop folosind noul sistem
de notificari din Ubuntu 9.04\framebreak
\item [(bonus)] sincronizat prin bluetooth cu un maimutoi de pe masa de langa calculator
care face urat cand primim mailuri noi :P
\item [(optional)]internationalizat
\item [(optional)] eventual conceput sub forma de biblioteca
\end{itemize}
\end{frame}

\begin{frame}{Limbaje propuse}
\begin{itemize}
\item C\\\begin{itemize}\item fast \item powerful \item hardcore\end{itemize}
\item gtk\\\begin{itemize}\item interfata grafica\end{itemize}
\pause \vspace{1cm}
\item nu e alegerea mea ;)
\end{itemize}
\end{frame}

\begin{frame}{Version Control}
\begin{itemize}
\item git \vspace{1cm}
\item ...din nou, nu e alegerea mea ;)
\end{itemize}
\end{frame}

\begin{frame}{Ce se invata in urma proiectului?}
\begin{itemize}
\item dbus
\item socket programming
\item protocoale de comunicatie (pop3, imap)
\item multithreading
\end{itemize}
\end{frame}

\begin{frame}{Reinventing the wheel ?}
\pause
... so what ?! asa se invata ...
\end{frame}

\end{document}
