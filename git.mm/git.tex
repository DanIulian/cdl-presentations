% vim: set tw=78 aw sw=2 sts=2 noet:
\documentclass{beamer}

%\includeonlyframes{c} -- speeding up compilation speed during debug

\usepackage[utf8x]{inputenc} % diacritice
\usepackage[romanian]{babel}
\usepackage{hyperref}        % folositi \url{http://...}
\mode<presentation>
\usetheme{CDL}

\title[]{Versionarea codului}
\subtitle{CDL - Cursul 2}
\institute[]{ROSEdu}
\author[]{
  Andrei Petre \texttt{anpetre@rosedu.org}\\
  Mihai Maruseac \texttt{mihai@rosedu.org}
}

\setbeamertemplate{frametitle continuation}[from second]
\setbeamertemplate{footline}[frame number]

\pgfdeclareimage[width=3cm]{vd01}{img/versionare-dummy-01}
\pgfdeclareimage[height=5cm]{vd02}{img/versionare-dummy-02}
\pgfdeclareimage[height=6cm]{ae}{img/all-equal}
\pgfdeclareimage[height=6cm]{dict}{img/dictator}
\pgfdeclareimage[height=6cm]{objs}{img/objs}
\pgfdeclareimage[height=4cm]{snapshots}{img/snapshots}
\pgfdeclareimage[height=3cm]{parent}{img/parent}
\pgfdeclareimage[height=6cm]{tags}{img/tags}
\pgfdeclareimage[height=6cm]{zones}{img/zones}
\pgfdeclareimage[height=1cm]{wtc}{img/wtc}
\pgfdeclareimage[height=8cm]{bm}{img/bm}
\pgfdeclareimage[width=5cm]{decentr}{img/decentr}
\pgfdeclareimage[height=8cm]{scms}{img/scms}
\pgfdeclareimage[width=8cm]{merge}{img/merge}
\pgfdeclareimage[height=6cm]{gh}{img/gh}
\pgfdeclareimage[height=6cm]{gpr}{img/gpr}
\pgfdeclareimage[height=8cm]{git}{img/git}

\begin{document}

\maketitle

\begin{frame}{Ce înseamnă versionare?}
  \begin{itemize}
    \item informații despre modificările din cod
  \end{itemize}
  \begin{center}
    \pgfuseimage{vd01}
  \end{center}
\end{frame}

\begin{frame}{După ceva timp}
  \begin{center}
    \pgfuseimage{vd02}
  \end{center}
\end{frame}

\begin{frame}{Ce vrem de la o soluție de versionare?}
  \begin{itemize}
    \item istoricul modificărilor
    \pause
    \item posibilitatea de revenire
      \pause
    \item posibilitatea de interogare
      \pause
    \item posibilități de dezvoltare în ramuri, istoric arborescent
  \end{itemize}
\end{frame}

\begin{frame}{Soluții versionare}
  \begin{center}
    \pgfuseimage{scms}
  \end{center}
\end{frame}

\begin{frame}{De ce Git?}
  \begin{itemize}
    \item cool
    \item toată lumea îl folosește
    \item distribuit
    \item merge și fără conexiune la internet
    \item documentație
    \item GitHub, Gitorious, BitBucket, \ldots
    \item Linux, Mac, Windows, \ldots
    \item dezvoltare continuă
    \item facilități super cool (powerusers)
    \item integrare cu tool-uri third-party (more power)
  \end{itemize}
  \vspace{-6cm}
  \begin{center}
    \hspace{6cm}
    \pgfuseimage{decentr}
  \end{center}
\end{frame}

\begin{frame}{Design Git}
  \begin{itemize}
    \item performanță
    \item dezvoltare neliniară
    \item integritate
  \end{itemize}
\end{frame}

\begin{frame}{Operații cu soluții de versionare}
  \begin{itemize}
    \item creare repository
    \item clonare repository
    \item creare de commit-uri
    \item actualizare/editare commit-uri (teh Git shit)
    \item sincronizare repository-uri (push/pull)
    \item tagging
    \item branching
    \item revenire la o versiune anterioară
  \end{itemize}
\end{frame}

\begin{frame}{All repositories are created equal...}
  \begin{center}
    \pgfuseimage{ae}
  \end{center}
\end{frame}

\begin{frame}{... but some are more equal than others}
  \begin{center}
    \pgfuseimage{dict}
  \end{center}
\end{frame}

\begin{frame}{Object types}
  \begin{center}
    \pgfuseimage{objs}
  \end{center}
\end{frame}

\begin{frame}{Commit = snapshot, nu delta}
  \begin{center}
    \pgfuseimage{snapshots}
  \end{center}
\end{frame}

\begin{frame}{Each commit knows his parent}
  \begin{center}
    \pgfuseimage{parent}
  \end{center}
\end{frame}

\begin{frame}{Tags \& Branches}
  \begin{center}
    \pgfuseimage{tags}
  \end{center}
\end{frame}

\begin{frame}{Merge}
  \begin{center}
    \pgfuseimage{merge}
  \end{center}
\end{frame}

\begin{frame}{Comenzi uzuale}
  \begin{center}
    \pgfuseimage{zones}
  \end{center}
\end{frame}

\begin{frame}{Workflow local}
  \begin{enumerate}
    \item creezi repository: \texttt{git init .}
    \item creezi fișiere/directoare: editoare, \texttt{mkdir}, \texttt{cp}
    \item adaugi \textbf{conținut} pentru a crea un commit: \texttt{git
      add}
    \item creezi commit: \texttt{git commit -m "Mesaj"}
    \item mergi la pasul 2
  \end{enumerate}
\end{frame}

\begin{frame}{Workflow contribuitor}
  \begin{enumerate}
    \item clonezi repository: \texttt{git clone URL}
    \item creezi fișiere/directoare: editoare, \texttt{mkdir}, \texttt{cp}
    \item adaugi \textbf{conținut} pentru a crea un commit: \texttt{git
      add}
    \item creezi commit: \texttt{git commit -m "Mesaj"}
    \item dacă vrei să lucrezi în continuare local mergi la pasul 2
    \item actualizezi repository: \texttt{git pull --rebase}
    \item faci push la modificări: \texttt{git push origin master}
    \item mergi la pasul 2
  \end{enumerate}
\end{frame}

\begin{frame}{Recap :: Cuvinte cheie}
  \begin{itemize}
    \item repository
    \item clone
    \item commit
    \item tree
    \item tag
    \item HEAD
    \item remote
    \item push/pull
    \item branch
    \item merge
    \item patch
    \item staging area
    \item fetch/checkout
  \end{itemize}
\end{frame}

\begin{frame}{Configurare inițială :: Configurații globale}
  \begin{itemize}
    \item \texttt{git config --global user.name "\textit{nume prenume}"}
    \item \texttt{git config --global user.email "\textit{nume@dom.com}"}
    \item \texttt{git config --global color.ui auto}
    \item \texttt{git config --global color.pager true}
    \item \texttt{git config --global core.editor \textit{editor}}
  \end{itemize}
  \begin{itemize}
    \item Verificare: \texttt{cat $\sim$/.gitconfig}
  \end{itemize}
\end{frame}

%\begin{frame}{Configurare inițială :: Configurare cheie SSH}
%  \begin{itemize}
%    \item \texttt{ssh-keygen}
%    \item În \texttt{$\sim$/.ssh}:
%      \begin{itemize}
%	\item \texttt{id\_rsa}
%	\item \texttt{id\_rsa.pub}
%      \end{itemize}
%  \end{itemize}
%  \begin{block}{Task3}
%    Construiți-vă o pereche de chei SSH (dacă nu aveti deja una pe sistemul cu
%    care lucrați)
%  \end{block}
%\end{frame}
%
%\begin{frame}{Configurare inițială :: Cont GitHub}
%  \begin{center}
%    \pgfuseimage{gh}
%  \end{center}
%\end{frame}
%
%\begin{frame}{Configurare inițială :: GitHub tasks}
%  \begin{block}{Task4}
%    Creați-vă un cont pe GitHub. Spuneți-mi numele contului.
%  \end{block}
%  \begin{block}{Task5}
%    Adăugați-vă cheia la cont, dacă nu exista deja.
%  \end{block}
%\end{frame}

\begin{frame}{Fișiere binare}
  \begin{itemize}
    \item diff între binare, rewrite
    \item ocupă mult spațiu
    \item generate din alte surse
      \begin{itemize}
        \item regulă de bază: nu includeți fișiere care sunt generate
	\item fișiere obiect, fișiere executabile, biblioteci, fișiere PDF din
	  surse \LaTeX, fișiere imagine generate din Dia sau Inkscape
	\item dacă nu sunt generate (fișiere imagine), atunci trebuie incluse
      \end{itemize}
  \end{itemize}
\end{frame}

\begin{frame}{Ignore}
  \begin{itemize}
    \item local \texttt{.git/info/exclude}
    \item repository \texttt{.gitignore}
    \item system wide \texttt{excludesfile}
  \end{itemize}
  \pause
  \begin{itemize}
    \item \texttt{file.pdf}
    \item \texttt{/file.pdf}
    \item \texttt{!file.pdf}
    \item \texttt{/test/file.pdf}
    \item \texttt{*.pdf}
    \item \texttt{/*.pdf}
    \item \texttt{*.swp}, \texttt{*$\sim$}, \texttt{tags}, \texttt{TAGS},
      \texttt{cscope.out}
  \end{itemize}
\end{frame}

\begin{frame}{Commits}
  \begin{itemize}
    \item mici și dese
    \item \textit{Do one thing, do one thing well!}
  \end{itemize}
  \begin{center}
    \pgfuseimage{zones}
  \end{center}
\end{frame}

\begin{frame}{Mai multe despre commit}
  \begin{itemize}
    \item Remember: mici și dese
    \item \texttt{git add --patch} sau \texttt{git add -i}
    \item \texttt{git commit --amend} pentru a modifica un commit
    \item Modificați doar commituri locale!!!
  \end{itemize}
\end{frame}

%\begin{frame}{Primul commit}
%  \begin{block}{Task6}
%    \begin{itemize}
%      \item clonați \url{https://github.com/mihaimaruseac/dandroid}
%      \item intrați în directorul asociat numelui vostru
%      \item construiți un fișier \texttt{me.txt} cu un conținut la liberă alegere
%      \item faceți un commit cu acest fișier
%    \end{itemize}
%  \end{block}
%\end{frame}

\begin{frame}{Vizualizarea istoriei.}
  \begin{itemize}
    \item \texttt{git log}
    \item \texttt{gitk}, \texttt{gitg}, \texttt{tig}, \ldots
  \end{itemize}
%  \begin{block}{Task7}
%    Vizualizati istoria repo-ului clonat anterior.
%  \end{block}
\end{frame}

\begin{frame}{Adresare commit}
  \begin{itemize}
    \item \texttt{HEAD}
    \item tag
    \item \texttt{HEAD\textasciicircum}
      \texttt{HEAD\textasciicircum\textasciicircum}, \ldots
    \item \texttt{HEAD$\sim$10}
    \item hash
    \item partial hash
  \end{itemize}
%\end{frame}
%
%\begin{frame}{Afișare commit}
  \begin{itemize}
    \item \texttt{git show \textit{id}}
  \end{itemize}
%  \begin{block}{Task8}
%    Afișați al 24-lea commit din spatele commitului actual și apoi părintele acestuia
%  \end{block}
\end{frame}

\begin{frame}{Mesajele de commit}
  \begin{itemize}
    \item descriptive
    \item propoziție, începe cu majusculă
    \item recomandare de limitare la 50 de caractere
    \item dacă e nevoie de mai mult mesaj nou
    \scriptsize
    \item
      \url{http://tbaggery.com/2008/04/19/a-note-about-git-commit-messages.html}
    \normalsize
    \item \url{http://whatthecommit.com}
  \end{itemize}
  \begin{center}
    \pgfuseimage{wtc}
  \end{center}
\end{frame}

\begin{frame}{Modificarea istoriei}
  \begin{itemize}
    \item Nu modificați istoria remote !!!
    \item \texttt{git rebase -i \textit{id}}
      \begin{itemize}
	\item reword
	\item edit
	\item squash
	\item fixup
	\item remove
      \end{itemize}
  \end{itemize}
\end{frame}

%\begin{frame}{Merge automat}
%  \begin{block}{Task9}
%    \begin{itemize}
%      \item Modificați în fișierul \texttt{numbers} linia corespunzătoare
%	valorii $10 * n$ unde $n$ este numărul vostru din tabelul de înscrieri.
%      \item Faceti un commit urmat de push
%    \end{itemize}
%  \end{block}
%\end{frame}
%
%\begin{frame}{Merge cu conflicte}
%  \begin{block}{Task10}
%    Treceți-vă numele în fișierul \texttt{signatures} și faceți-i push.
%  \end{block}
%\end{frame}

\begin{frame}{Branching}
  \begin{itemize}
    \item editare colaborativă
    \item lucru foarte rapid în Git
    \item multiple ramuri de dezvoltare
      \begin{itemize}
        \item release
        \item new feature
        \item experimental feature
        \item bug fixing
        \item hot fixes
        \item \ldots
      \end{itemize}
  \end{itemize}
  \begin{itemize}
    \item \texttt{git branch}
    \item \texttt{git checkout}
  \end{itemize}
\end{frame}

\begin{frame}{Branching model}
  \begin{center}
    \pgfuseimage{bm}
  \end{center}
\end{frame}

\begin{frame}{Branch-uri}
  \begin{itemize}
    \item fiecare clonă a unui repo este un fel de branch pentru acesta
    \item \texttt{git pull} creează commit de merge: rezolvă conflicte
      \begin{itemize}
	\item modificări simultane asupra repository-ului
      \end{itemize}
  \end{itemize}
\end{frame}

\begin{frame}{\texttt{git pull --rebase}}
  \begin{center}
    \pgfuseimage{gpr}
  \end{center}
\end{frame}

%\begin{frame}{Branching tasks}
%  \begin{block}{Task11}
%    Creați un branch cu numele vostru în care ștergeți toate fișierele și
%    directoarele, mai puțin directorul vostru.
%  \end{block}
%  \begin{alertblock}{Atentie!!}
%    Asigurați-vă că sunteți pe branch-ul corespunzător!!.
%  \end{alertblock}
%\end{frame}

%\begin{frame}{Git debugging}
%  \begin{itemize}
%    \item \texttt{git blame}
%    \item \texttt{git bisect}
%  \end{itemize}
%%  \begin{block}{Task12}
%%    Identificați cine a modificat linia 10 din fișierul \texttt{signatures}.
%%  \end{block}
%%  \begin{block}{Task13}
%%    \begin{itemize}
%%      \item Scriptul \texttt{text\_math.py} conține un bug.
%%      \item Identificati bug-ul, considerând drept commit "good" commitul cu
%%	tag-ul \texttt{bad} (commit "bad" este commitul curent)
%%    \end{itemize}
%%  \end{block}
%%  \begin{alertblock}{Atentie!!}
%%    Asigurați-vă că sunteți pe \texttt{master}!!
%%  \end{alertblock}
%\end{frame}

%\begin{frame}{Selecție commit}
%  \begin{itemize}
%    \item pe un branch există un commit de care am avea nevoie
%    \item nu vrem să facem merge între branch-uri
%      \pause
%    \item \texttt{git cherry-pick}
%  \end{itemize}
%  \begin{block}{Task14}
%    Branch-ul \texttt{extra} conține un commit care trebuie să ajungă în
%    \texttt{master}. Aduceți-l acolo.
%  \end{block}
%\end{frame}
%
%\begin{frame}{Temporary swap space}
%  \begin{itemize}
%    \item lucrezi la un feature
%    \item și afli că ai un bug undeva
%    \item nu vrei să faci commit
%      \pause
%    \item \texttt{git stash}, \texttt{git stash apply}
%  \end{itemize}
%\end{frame}
%
%\begin{frame}{\texttt{send-email} and \texttt{format-patch}}
%  \begin{itemize}
%    \item kernelul Linux și alte proiecte
%    \item nu \texttt{git push}
%    \item mailing list, review, copiat commit
%    \item linie \texttt{Signed-Off-By:}
%  \end{itemize}
%\end{frame}

%\begin{frame}{Hook-uri}
%  \begin{itemize}
%    \item \texttt{.git/hooks} în repo
%    \item \texttt{post-receive}
%    \item publicarea de rezultate
%    \item crearea de arhive
%    \item trimitere de e-mail-uri
%    \item GitHub hooks
%  \end{itemize}
%\end{frame}

\begin{frame}{Git is...}
  \begin{center}
    \pgfuseimage{git}
  \end{center}
\end{frame}

\begin{frame}{Links}
  \begin{itemize}
    \item \href{http://gitimmersion.com/}{Git Immersion}
    \item \href{http://gitready.com/}{git ready}
    \item \href{http://www-cs-students.stanford.edu/~blynn/gitmagic/index.html}{Git Magic}
    \item \href{http://aht.github.com/whatisgit/\#/cover}{What is git?}
    \item \href{http://www.slideshare.net/err/git-machine}{Git: The Lean, Mean, Distributed Machine}
    \item \href{http://en.wikipedia.org/wiki/Comparison\_of\_revision\_control\_software}{Comparație între SCM-uri}
    \item \href{http://talks.rosedu.org/prezentari/prezentarea03}{Prezentare Tech Talks Git (Mircea Bardac)}
    \item \href{http://nvie.com/posts/a-successful-git-branching-model/}{A branching model}
    \item \href{http://github.com}{GitHub}
    \item \href{http://www.eqqon.com/index.php/Collaborative\_Github\_Workflow}{Workflow GitHub}
    \item \href{http://techblog.rosedu.org/git-good-practices.html}{Git Tips and Good Practices}
    \item \href{http://techblog.rosedu.org/contributing-upstream.html}{Contributing Upstream}
    \item \href{http://techblog.rosedu.org/git-good-practices.html}{Git Good Practices}
    \item \href{http://techblog.rosedu.org/git-useful-aliases.html}{Git Useful Aliases}
  \end{itemize}
\end{frame}

\end{document}
