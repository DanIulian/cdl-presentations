% vim: set tw=78 aw:
\documentclass{beamer}

\usepackage[utf8x]{inputenc} % diacritice
\usepackage[romanian]{babel}
\usepackage{amsmath}
\usepackage{amssymb}
\usepackage{color}			 % highlight
\usepackage{alltt}			 % highlight
\usepackage{code/highlight}	 % highlight
\usepackage{hyperref}        % folositi \url{http://...}
                             % sau \href{http://...}{Nume Link}
\mode<presentation>
{ \usetheme{Rochester} }		% TODO: settle this

% Titlul nu foloseşte Unicode pentru că e o problemă căreia nu i-am dat de
% cap.
\title[Unit Testing and Debugging]{Unit Testing and Debugging}
\subtitle{CDL - Cursul 4}
\institute{ROSEdu}
\author{Mihai Maruseac \\\texttt{mihai.maruseac@gmail.com}}

\begin{document}

% Slide-urile cu mai multe părţi sunt marcate cu textul (cont.)
\setbeamertemplate{frametitle continuation}[from second]
% Arătăm numărul frame-ului
\setbeamertemplate{footline}[frame number]

\frame{\titlepage}

\frame{\tableofcontents}

\section{Unit testing}
\frame{\tableofcontents[currentsection]}

\begin{frame}{De ce testing?}
\begin{itemize}
\item Nu ar trebui să scriem întâi cod și apoi să ne gândim la a-l testa?
\item \pause Nici măcar o bucată de cod?
\item \pause Scriem bucata 1 de cod pentru a obține funcționalitatea A
\item \pause Scriem bucata 2 pentru a obține și funcționalitatea B.
\item \pause Le legăm împreună fără a le testa în prealabil.
\item \pause \textit{Where did this bug come from?}
\item \pause O bucată conține cod incompatibil cu cealaltă.
\item \pause Exemplu real: \pause Hammerfall
\item \pause Solutia? \pause Unit testing
\end{itemize}
\end{frame}

\begin{frame}{Unit Testing}
\begin{itemize}
\item Paradigmă de dezvoltare prin care programatorii dobândesc încredere în faptul că bucăți individuale din codul lor sunt potrivite pentru folosirea în orice situații.
\item \pause Unitatea: cea mai mică bucată de cod care poate funcționa independent de restul aplicației
\item \pause Scopul: izolarea porțiunilor de program și demonstrarea funcționalității lor corecte
\item \pause Un unit-test complet oferă garanția că unitatea testată funcționează corect în orice situație care s-ar putea ivi.
\item \pause Test-driven development
\item \pause Unul din fundamentele Extreme Programming
\end{itemize}
\end{frame}

\begin{frame}{Extreme Programming (XP)}
\begin{itemize}
\item Metodologie de programare, stil de dezvoltare, disciplină de lucru
\item \pause Scop principal: reducerea costului asociat unei schimbări minore în cod.
\end{itemize}
\end{frame}

\begin{frame}{Ideile XP}
\begin{itemize}
\item Codul $=$ cel mai important produs al procesului de dezvoltare al unui proiect
\item \pause Codul poate fi (și TREBUIE) redactat clar, concis și într-o manieră care nu lasă loc la alte interpretări.
\item \pause Codul poate fi folosit și pentru a transmite ideile în cazul în care alte alternative eșuează
\item \pause Nimeni nu poate fi sigur de ceva până ce nu se efectuează teste
\item \pause Nu poți fi sigur $100\%$ că ai scris ce trebuia scris
\item \pause UNIT TESTS $\rightarrow$ testathon
\item \pause Proiectul pornește de la o etapă de design
\end{itemize}
\end{frame}

\begin{frame}{Paradigme XP utile}
\begin{itemize}
\item Pair Programming
\item \pause Unit Testing (Test Driven Development)
\end{itemize}
\end{frame}

\begin{frame}{Back to Unit Testing}
\begin{itemize}
\item Ok, am înțeles ce este și de ce e util
\item \pause Dar cum îl folosim?
\item \pause Scriem testele
\item \pause Și testăm. \pause Cum?
\begin{itemize}
\item manual
\item \pause automat
\end{itemize}
\item \pause Vom vedea ambele metode prin exemple.
\end{itemize}
\end{frame}

\begin{frame}{Problema propusă}
\begin{itemize}
\item Se cere să construim o funcție ce calculează cel mai mic divizor comun pentru două numere întregi.
\item \pause NU vom începe prin a scrie funcția
\item \pause Ne gândim ce axiome respectă funcția noastră:
\begin{itemize}
\item \pause gcd(a, b) $>$ 0 $\forall a,b \in Z$
\item gcd(a, a) = a, $\forall a \in Z$
\item gcd(1, a) = 1, $\forall a \in Z$
\item gcd(0, a) = a, $\forall a \in Z$
\item gcd(a,nb) = gcd(a, b), $\forall a,b,n \in Z$
\item gcd(a, b) = gcd(b, a), $\forall a,b \in Z$
\item gcd(a, b) = gcd(-a, b), $\forall a,b \in Z$
\end{itemize}
\item \pause Totuși, noi scriem cod nu matematică.
\item \pause Cum trecem axiomele în cod?
\item \pause Valori particulare sau aleatorii (sau combinații)
\end{itemize}
\end{frame}

\begin{frame}{Testare manuală}
\begin{itemize}
\item Testele se scriu unul câte unul și se folosesc multe instrucțiuni if pentru a vedea dacă testul a fost trecut sau nu
\item \pause Poate deveni (extrem de) supărător la un moment dat
\item \pause Uneori, nu avem altă alternativă ieftină.
\end{itemize}
\end{frame}

\begin{frame}{Testare manuală - exercițiu}
\begin{itemize}
\item Descărcați gcd.c de pe http://anaconda.cs.pub.ro/$\sim$mmaruseac/CDL/UT/gcdC/
\item Identificați în main testele lipsă, adăugați-le, compilați și rulați executabilul creat.
\item \pause Completați funcția gcd până ce va trece toate testele
\item \pause Se observă cantitatea mare de linii de cod scrise pentru o simplă testare.
\item \pause Să vedem dacă nu putem mai bine. :D
\end{itemize}
\end{frame}

\begin{frame}{Testare automată}
\begin{itemize}
\item Cineva a scris pentru noi un framework de testare
\item \pause Noi doar vom scrie testele
\item \pause Simplu, nu? \pause Să vedem...
\end{itemize}
\end{frame}

\begin{frame}{Testare automată + teaser Python}
\begin{itemize}
\item Descărcați arhiva ZIP de la http://anaconda.cs.pub.ro/$\sim$mmaruseac/CDL/UT/gcdPy/ și dezarhivați-o (unzip)
\item Folosiți \linebreak {\bfseries python gcdtest.py}\linebreak pentru a rula fișierul Python
\item \pause Adăugați celelalte teste după modelul din exemplu.
\item \pause Testați programul.
\item \pause Completați funcția Gcd (din gcd.py).
\item \pause Testați programul (repetați pasul anterior până merg toate testele)
\end{itemize}
\end{frame}

\begin{frame}{Unit testing - concluzii}
\begin{itemize}
\item ÎNAINTE de a scrie codul, ne obligă să detaliem aspecte esențiale ale proiectului, într-o manieră cu adevărat folositoare.
\item \pause În timpul SCRIERII codului, putem evita capcana scrierii a prea mult cod. \pause Putem considera un modul ca fiind complet în momentul în care a trecut toate testele scrise pentru el.
\item \pause În timpul MODIFICĂRII codului, suntem siguri că orice vom obține va fi cel puțin la fel de bun ca rezultatul dinainte de modificare.
\item \pause În cazul scrierii codului în ECHIPĂ, ajută la creșterea certitudinii că liniile scrise de altcineva nu vor induce erori în liniile proprii.
\end{itemize}
\end{frame}

\section{Debugging}
\frame{\tableofcontents[currentsection]}

\begin{frame}{Introducere}
\begin{itemize}
\item Ce ne facem când avem erori?
\item \pause Depanăm (debugging)
\item \pause \textit{It's easy to cry `bug' when the truth is that you've got a complex system and sometimes it takes a while to get all the components to co-exist peacefully}
\end{itemize}
\end{frame}

\begin{frame}{Brute force}
\begin{itemize}
\item hardcore
\item \pause printf și convenții proprii
\item \pause Multe modificări prin cod
\item \pause Atenție la convenții
\item \pause Analiza textului afișat $>$ analiza codului
\item \pause Nu funcționează mereu
\item \pause Și nici nu e eficient \pause(dureri de cap după un timp)
\item \pause Dar este metoda preferată de mulți.
\end{itemize}
\end{frame}

\begin{frame}{Metoda eficientă}
\begin{itemize}
\item Folosită de puțini \pause(\textit{`puțini dar cu cap'})
\item \pause Mai multe programe ajutătoare:
\begin{itemize}
\item GDB (GNU Debugger) + clone
\item Valgrind
\item Electric Fence
\item Data Display Debugger
\item etc
\item ș.a.m.d
\item ...
\end{itemize}
\end{itemize}
\end{frame}

\begin{frame}{GDB}
\begin{itemize}
\item Depanatorul standard
\item Portabil: multe sisteme Unix și multe limbaje de programare (Ada, C, C++, Basic, Fortran)
\item Stallman, 1986
\item \pause GDB poate face 4 lucruri majore și importante:
\begin{itemize}
\item \pause Pornirea programului specificând orice parametru care i-ar putea afecta comportamentul
\item \pause Forțarea opririi programului în cazul îndeplinirii anumitor condiții
\item \pause Examinarea pașilor ce au dus la oprirea prematură (seg fault?) a programului
\item \pause Schimbarea anumitor variabile (și instrucțiuni uneori) din program pentru a-l corecta.
\end{itemize}
\end{itemize}
\end{frame}

\begin{frame}{GDB - limitări}
\begin{itemize}
\item GDB NU poate fi rulat înapoi \pause decât pe anumite arhitecturi și sisteme.
\item \pause În 2008, adăugarea acestei posibilități era listată ca unul dintre proiectele \textit{top priority} ale FSF
\item \pause De asemenea, doar pe anumite arhitecturi se pot modifica instrucțiuni din program în timpul execuției lui.
\item \pause GDB nu are interfață grafică $\rightarrow$ DDD.
\end{itemize}
\end{frame}

\begin{frame}{GDB - basic}
\begin{itemize}
\item Compilarea sursei se face special:
\begin{itemize}\item Trebuie să permit afișarea unor informații de debug 
\item Renunțăm la orice optimizare pentru a nu fi încurcați de optimizările făcute de compilator
\item \pause Ultima versiune poate fi neglijată uneori, mai ales dacă bug-ul apare doar în versiunile optimizate.
\end{itemize}
\pause {\bfseries gcc -g -Wall -O0 \textit{sursa}}
\item \pause Eu folosesc un target special în makefile pentru chestia asta (nu e standard)
\item \pause Pornire proces debug:\linebreak{\bfseries gdb \textit{executabil}}
\item \pause Vedem pe un exemplu
\end{itemize}
\end{frame}

\begin{frame}{GDB - exercițiu}
\begin{enumerate}
\item Descărcați exemplul p1.c de la http://anaconda.cs.pub.ro/$\sim$mmaruseac/CDL/DBG/
\item Compilați sursa normal și rulați-o
\item \pause \textit{Oups, seg fault}
\item \pause Compilați sursa pentru debugging
\item Porniți gdb având ca argument executabilul creat anterior
\item \pause Rulați programul în gdb:\linebreak {\bfseries r}un (r)
\item \pause Logic, seg fault
\item \pause Să vedem contextul în care crapă:
\begin{itemize}
\item Liniile din cod se pot afla cu\linebreak {\bfseries l}ist (l)
\item O vedere de ansamblu a stivei se poate vedea cu\linebreak backtrace (bt)
\end{itemize}
\item Desigur, putem vedea eroarea chiar și acum dacă suntem norocoși și nu e codul prea urât scris.
\end{enumerate}
\end{frame}

\begin{frame}{GDB - exercițiu, cont. (dacă nu ne prindem?)}
\begin{enumerate}
\item Vom adăuga un breakpoint cu\linebreak break \textit{linenumber/function name/address} (b)
\item \pause Pornim programul 
\item \pause Programul se va opri la primul breakpoint întâlnit.
\item \pause Poate continua în cel puțin 4 moduri:
\begin{itemize}
\item continue (c) - repornește execuția, pâna la următorul breakpoint sau până la final.
\item next (n) - următoarea instrucțiune, NU coboară în stiva de apeluri
\item step (s) - următoarea instrucțiune
\item finish - execută pâna la return, printează valorea și se oprește
\end{itemize}
\item \pause Rulăm pas cu pas și identificăm linia unde crapă și motivul.
\item \pause Ah, uitasem de o $*$ la pointer. $:($
\end{enumerate}
\end{frame}

\begin{frame}{Hints:}
\begin{itemize}
\item O linie goală înseamnă repetarea comenzii anterioare (tastare eficientă)
\item \pause Dorim afișarea conținutului unui variabile? print (p)
\item \pause Dar dacă vrem mai multe? display
\item \pause Sau dacă vrem să le afișăm doar dacă se modifică? watch
\item \pause Ce facem dacă vrem să modificăm o variabilă (nu are ce valoare vrem noi)? \linebreak {\bfseries set var expr} 
\item \pause Pentru a șterge un breakpoint, folosim clear
\item \pause Breakpoint-urile pot fi și condiționale:\linebreak b if cond
\item \pause Cea mai importantă comandă? \pause help
\end{itemize}
\end{frame}

\begin{frame}{GDB - exercițiu, cont. (după rezolvarea unui bug)}
\begin{enumerate}
\item Recompilăm programul și rulăm
\item \pause Oups, alt seg fault
\item \pause Depanăm și repetăm până ce merge totul corect.
\end{enumerate}
\end{frame}

\begin{frame}{Valgrind}
\begin{itemize}
\item Inițial construit a fi un depanator de memorie liber
\item \pause Evoluat până la a deveni un framework generic pentru instrumente de analiză dinamică.
\item \pause Valgrind = Mașină virtuală utilizând JIT.
\item \pause cod \pause$\rightarrow$ Intermediate Representation \pause$\xrightarrow{Instrument\ de\ testare}$ Modificări ale IR \pause$\rightarrow$ Cod mașină rulat în gazdă.
\item \pause Se pierde puțin din performanță \pause(4-5 ori mai lent).
\item \pause Totuși, codul IR este o formă perfectă ce permite scrierea facilă de instrumente de testare.
\end{itemize}
\end{frame}

\begin{frame}{Valgrind - tools}
\begin{itemize}
\item \textit{Memcheck}
\begin{itemize}
\item \pause Implicit
\item \pause Inserează cod extra în jurul fiecărei instrucțiuni
\item \pause Sanitizarea memoriei: \pause validitate \pause și adresabilitate
\item \pause Înlocuiește mecanismul de alocare standard cu cel propriu
\item \pause Problemele pe care le poate detecta:
\begin{itemize}
\item \pause Utilizare a memoriei neinițializate
\item \pause Citire/scriere în zone de memorie eliberate
\item \pause Citire/scriere la finalul blocurilor alocate
\item \pause Scurgeri de memorie
\end{itemize}
\item \pause De 20 de ori mai lent \pause dar extrem de util
\end{itemize}
\item \pause \textit{Massif} Heap profiler
\item \pause \textit{Helgrind} Util în codul multithreaded.
\end{itemize}
\end{frame}

\begin{frame}{Valgrind - utilizare}
\begin{itemize}
\item Având programul de la gdb, rulați valgrind pe el\linebreak
valgrind \textit{path/to/file}
\item \pause Veti obține un detaliu al memoriei folosite.
\item \pause Faceți free la toată memoria alocată până nu mai aveți memory leaks
\item \pause Valgrind nu e doar pentru memomy leaks. Să vedem alte funcționalități.
\item \pause Descărcați același fișier sursă cu buguri (de la gdb).
\item \pause Compilați și rulați valgrind pe output.
\item \pause Valgrind vă spune foarte clar ce erori aveți în modul de lucru cu memoria.
\end{itemize}
\end{frame}

\begin{frame}{Electric Fence}
\begin{itemize}
\item Memory debugger
\item Bibliotecă ce poate fi legată la cod pentru a suprascrie managementul clasic al memoriei
\item Generează un crash (Segmentation fault) imediat ce apare un acces invalid la memorie.
\item \pause Pare util, nu? \pause Dar, (trebuie să existe așa ceva)
\item \pause Crește rapid memoria utilizată de program
\end{itemize}
\end{frame}

\begin{frame}{Data Display Debugger}
\begin{itemize}
\item GDB cu interfață grafică
\item \pause De fapt și PDB, JDB, BashDB etc cu interfețe grafice
\item \pause Avantaj: \pause Se studiază programul prin vizualizarea datelor, nu prin cea a instrucțiunilor.
\item \pause Să vedem câteva screenshoturi din DDD pentru a realiza cât de facilă este folosirea lui. Mergeți la adresa \linebreak {\bfseries http://anaconda.cs.pub.ro/$\sim$mmaruseac/CDL/DDD/}\linebreak
pentru a vizualiza câteva screenshoturi (oficiale)
\end{itemize}
\end{frame}

\begin{frame}{Concluzii}
\textit{Programming today is a race between software engineers striving to build bigger and better idiot-proof programs, and the Universe trying to produce bigger and better idiots. So far, the Universe is winning.}
\end{frame}

\section{Linkuri utile}
\frame{\tableofcontents[currentsection]}
\begin{frame}{Linkuri utile}
\begin{itemize}
\item \href{http://www.gnu.org/software/gdb/}{GDB Homepage}
\item \href{http://sourceware.org/gdb/current/onlinedocs/gdb.html\#SEC_Top}{Debugging with GDB}
\item \href{http://valgrind.org/}{Valgrind Homepage}
\item \href{http://valgrind.org/docs/manual/QuickStart.html}{Valgrind Quick Start Guide}
\item \href{http://freshmeat.net/projects/efence/}{Efence}
\item \href{http://www.gnu.org/software/ddd/}{DDD homepage}
\item \href{http://www.gnu.org/manual/ddd/}{DDD manual}
\end{itemize}
\end{frame}

\end{document}
