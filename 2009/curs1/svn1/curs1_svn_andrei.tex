% vim: set tw=78 aw:
\documentclass{beamer}

\usepackage[utf8x]{inputenc}		% diacritice
\usepackage[romanian]{babel}
\usepackage{color}			% highlight
\usepackage{alltt}			% highlight
\mode<presentation>
{ \usetheme{Rochester} }		% TODO: settle this

% Titlul nu foloseşte Unicode pentru că e o problemă căreia nu i-am dat de
% cap.
\title[SVN]{Introducere \^in Subversion}
\subtitle{CDL - Cursul 1}
\institute{ROSEdu}
\author{Andrei Buhaiu\\{\footnotesize andrewbwm@rosedu.org}}

\begin{document}

% Slide-urile cu mai multe părţi sunt marcate cu textul (cont.)
\setbeamertemplate{frametitle continuation}[from second]
% Arătăm numărul frame-ului
\setbeamertemplate{footline}[frame number]

\frame{\titlepage}

\frame{\tableofcontents}

% NB: Secţiunile nu sunt marcate vizual, ci doar apar în cuprins.
\section{Introducere}

% Pentru reamintirea periodică a cuprinsului şi unde ne aflăm:
\frame{\tableofcontents[currentsection]}

% Titlul unui frame se specifică fie în acolade, imediat după \begin{frame},
% fie folosind \frametitle
\begin{frame}{Introducere}
\begin{itemize} % Just like normal LaTeX
\item Unul din cele mai importante utilitare în dezvoltarea proiectelor Open Source
\newline 
\item Facilitează lucrul simultan
\newline 
\item Program de versionare
\newline 
\item Permite o manevrare foarte bună a surselor
\end{itemize}
\end{frame}

\begin{frame}[allowframebreaks] % Spargem paginile mai mari automat
% Atenţie: allowframebreaks nu funcţionează cu overlay-uri
\frametitle{Repository}
\begin{itemize}
\item Un repository este un depozit de fișiere ce este sincronizat și poate fi accesat de mai mulți utilizatori
\newline 
\item În general într-un repository se găsesc sursele unui program, dar pot fi adăugate și alte fișiere, cum  ar fi documentația
\newline 
\item Locația unui repository este dată printr-un URL similar cu cel html: protocol://[user@]domeniu/cale
\end{itemize}
\end{frame}

\begin{frame}[allowframebreaks] % Spargem paginile mai mari automat
% Atenţie: allowframebreaks nu funcţionează cu overlay-uri
\frametitle{Exemple de locații}
\begin{itemize}
\item svn://svn.rosedu.org/test
\newline 
\item svn+ssh://svn.rosedu.org/test
\newline 
\item svn+ssh://gigi@svn.rosedu.org/test
\newline 
\item git+ssh://gigi@git.rosedu.org/test
\end{itemize}
\end{frame}

\section{The Basics}
\frame{\tableofcontents[currentsection]}

\begin{frame}{The Basics}
\begin{itemize} % Just like normal LaTeX
\item svn checkout svn://svn.rosedu.org/test/
\newline obținere copie locală a repository-ului
\newline 
\item svn commit -m “Mesaj relevant”
\newline se adaugă la repository-ul central modificările    făcute local
\newline 
\item svn update
\newline aduce la ultima versiune repository-ul local
\end{itemize}
\end{frame}

\begin{frame}{The Basics (Exerciții)}
\begin{itemize} % Just like normal LaTeX
\item Dați jos repository-ul ce se află la adresa svn.rosedu.org/cdl\_test
\newline
\item Modificați un fișier și încărcați-l pe repository-ul central
\newline
\item Actulizați-vă repository-ul pentru a primi și fișierele modificate de colegii voștri
\end{itemize}
\end{frame}

\begin{frame}{The Basics (2)}
\begin{itemize} % Just like normal LaTeX
\item svn add test.c
\newline se adaugă un nou fișier la repository-ul central
\newline 
\item svn delete test.c
\newline se șterge un fișier din repository-ul central
\newline 
\item svn status
\newline afișează starea fișierelor în raport cu repository-ul central
\end{itemize}
\end{frame}

\begin{frame}{The Basics (2) (Exerciții)}
\begin{itemize} % Just like normal LaTeX
\item Adăugați un fișier nou (cu numele vostru de familie) la repository-ul central
\newline 
\item Observați modificările
\newline 
\item Ștergeți fișierul adăugat anterior
\newline 
\item Observați modificările
\newline 
\item Faceți modificări într-un fișier și dați svn status
\end{itemize}
\end{frame}

\section{More Advanced}
\frame{\tableofcontents[currentsection]}

\begin{frame}{More Advanced}
\begin{itemize} % Just like normal LaTeX
\item svn checkout -r x svn://svn.rosedu.org/cdl\_test/
\newline obținere copie locală a repository-ului la revizia specificată
\newline 
\item svnadmin create /home/student/test/
\newline se creează un repository în folder-ul specificat
\newline 
\item svnadmin import /home/student/programming 
\newline file:///home/cdl/test/ -m "Initial import"
\newline importarea unor surse în repository
\end{itemize}
\end{frame}

\begin{frame}{More Advanced (Exerciții)}
\begin{itemize} % Just like normal LaTeX
\item Descărcați de pe net versiunea 13 a proiectului
\newline 
\item Creați-vă fiecare un repository pe mașina locală 
\newline 
\item Importați surse în repository-ul creat mai sus
\newline 
\end{itemize}
\end{frame}

\begin{frame}{Rezolvarea Conflictelor}
\begin{itemize} % Just like normal LaTeX
\item Renunțarea la sursa proprie
\newline revert test.c
\newline update test.c
\newline
\item Menținerea sursei proprii
\newline cp test.c.mine test.c
\newline svn resolved test.c
\newline 
\item Combinarea celor două variante, trebuie editat manual fișierul
\newline svn resolved test.c
\newline
\end{itemize}
\end{frame}

\begin{frame}{Alte comenzi}
\begin{itemize} % Just like normal LaTeX
\item svn diff
\newline
\item svnadmin dump
\newline
\item svnlook (tree)
\newline
\item svn list
\newline
\item svn resolve
\end{itemize}
\end{frame}

\end{document}
