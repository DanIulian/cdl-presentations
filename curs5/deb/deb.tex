% vim: set tw=78 aw:
\documentclass{beamer}

\usepackage[utf8x]{inputenc} % diacritice
\usepackage[romanian]{babel}
\usepackage{color}			 % highlight
\usepackage{alltt}			 % highlight
\usepackage{code/highlight}	 % highlight
\usepackage{hyperref}        % folositi \url{http://...}
% sau \href{http://...}{Nume Link}
\mode<presentation>
{ \usetheme{Rochester} }		% TODO: settle this

% Titlul nu foloseşte Unicode pentru că e o problemă căreia nu i-am dat de
% cap.
\title[Debian packaging]{Debian packaging}
\subtitle{CDL - Cursul 5}
\institute{ROSEdu}
\author{Lucian Adrian Grijincu \\ \texttt{lucian.grijincu@rosedu.org}}

\begin{document}

% Slide-urile cu mai multe părţi sunt marcate cu textul (cont.)
\setbeamertemplate{frametitle continuation}[from second]
% Arătăm numărul frame-ului
\setbeamertemplate{footline}[frame number]

\frame{\titlepage}

\frame{\tableofcontents}

% NB: Secţiunile nu sunt marcate vizual, ci doar apar în cuprins.
\section{Notiuni introductive}

% Pentru reamintirea periodică a cuprinsului şi unde ne aflăm:
\frame{\tableofcontents[currentsection]}

% Titlul unui frame se specifică fie în acolade, imediat după \begin{frame},
% fie folosind \frametitle
\begin{frame}{Utilitate uzuale}
  \begin{itemize}
  \item \textbf{./configure} - script generat cu autotools
    \begin{itemize}
    \item verifică dependențele programului
    \item folosit pentru a trece peste diferențele între sisteme tip UNIX
    \item generază un \textit{Makefile} dintr-un \textit{Makefile.ac} (un template)
    \end{itemize}
  \item \textbf{make} - construiește executabilul final din surse
  \item \textbf{???} - generare pachet .deb
  \item \textbf{dpgk} - manager de pachete pentru Debian
  \item \textbf{apt-*}, \textbf{aptitude} - interfețe de nivel înalt pentru dpkg
  \end{itemize}
\end{frame}

\begin{frame}{dpkg - manager de pachete pentru Debian}
  \begin{itemize}
  \item pentru toate pachetele de care știe ține minte starea pachetului: 
    instalat complet, doar fișierele de configurație rămase, instalare începută,
    dar neterminată, etc.
  \item \textbf{-i\/\texttt{--}install f.deb} - instalare fișier .deb
  \item \textbf{-r\/\texttt{--}remove pachet} - dezinstalare \textit{pachet}
  \item \textbf{-P\/\texttt{--}purge pachet} - dezinstalare \textit{pachet} și a fișierelor de configurație
  \item \textbf{-L\/\texttt{--}listfile pachet} - listarea fișierelor instalate de \textit{pachet}
  \item \textbf{-c\/\texttt{--}contents f.deb} - listarea fișierelor dintr-un .deb
  \item \textbf{-S\/\texttt{--}search file} - caută pachetele care instalează fișierul \textit{file} \\
    ex. \textit{dpkg -S \/bin\/ls} \\
    coreutils: \/bin\/ls
  \end{itemize}
\end{frame}


\begin{frame}{apt}
  \begin{itemize}
  \item \textbf{apt-cache dump} - listează toate pachetele din cache
  \item \textbf{apt-cache policy} - listează repo-urile din care e un pachet \\
    apt-cache policy firefox \\
    \texttt{   firefox:}\\
    Instalat: 3.0.7+nobinonly-0ubuntu0.8.10.1 \\
    Candidează: 3.0.8+nobinonly-0ubuntu0.8.10.2
  \item \textbf{apt-cache show} - listează informații despre un pachetul binar (maintainer, dependențe, descriere, etc.)
  \item \textbf{apt-cache showsrc} - listează informații despre un pachetul sursă (maintainer, dependențe, descriere, etc.)
  \item \textbf{apt-cache rdepends} - listează toate pachetele care depind de acesta
  \end{itemize}

\end{frame}


\section{Notiuni extroductive}

% Pentru reamintirea periodică a cuprinsului şi unde ne aflăm:
\frame{\tableofcontents[currentsection]}

\end{document}
