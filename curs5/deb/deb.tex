% vim: set tw=78 aw:
\documentclass{beamer}

\usepackage[utf8x]{inputenc} % diacritice
\usepackage[romanian]{babel}
\usepackage{color}			 % highlight
\usepackage{alltt}			 % highlight
\usepackage{code/highlight}	 % highlight
\usepackage{hyperref}        % folositi \url{http://...}
% sau \href{http://...}{Nume Link}
\mode<presentation>
{ \usetheme{Rochester} }		% TODO: settle this

% Titlul nu foloseşte Unicode pentru că e o problemă căreia nu i-am dat de
% cap.
\title[Debian packaging]{Debian packaging}
\subtitle{CDL - Cursul 5}
\institute{ROSEdu}
\author{Lucian Adrian Grijincu \\ \texttt{lucian.grijincu@rosedu.org}}

\begin{document}

% Slide-urile cu mai multe părţi sunt marcate cu textul (cont.)
\setbeamertemplate{frametitle continuation}[from second]
% Arătăm numărul frame-ului
\setbeamertemplate{footline}[frame number]

\frame{\titlepage}

\frame{\tableofcontents}

% NB: Secţiunile nu sunt marcate vizual, ci doar apar în cuprins.
\section{Notiuni introductive}

% Pentru reamintirea periodică a cuprinsului şi unde ne aflăm:
\frame{\tableofcontents[currentsection]}

% Titlul unui frame se specifică fie în acolade, imediat după \begin{frame},
% fie folosind \frametitle
\begin{frame}{Utilitate uzuale}
  \begin{itemize}
  \item \textbf{./configure} - script generat cu autotools
    \begin{itemize}
    \item verifică dependențele programului
    \item folosit pentru a trece peste diferențele între sisteme tip UNIX
    \item generază un \textit{Makefile} dintr-un \textit{Makefile.ac} (un template)
    \end{itemize}
  \item \textbf{make} - construiește executabilul final din surse
  \item \textbf{???} - generare pachet .deb
  \item \textbf{dpgk} - manager de pachete pentru Debian
  \item \textbf{apt-*}, \textbf{aptitude} - interfețe de nivel înalt pentru dpkg
  \end{itemize}
\end{frame}

\begin{frame}{dpkg - manager de pachete pentru Debian}
  \begin{itemize}
  \item pentru toate pachetele de care știe ține minte starea pachetului: 
    instalat complet, doar fișierele de configurație rămase, instalare începută,
    dar neterminată, etc.
  \item \textbf{-i\/\texttt{--}install f.deb} - instalare fișier .deb
  \item \textbf{-r\/\texttt{--}remove pachet} - dezinstalare \textit{pachet}
  \item \textbf{-P\/\texttt{--}purge pachet} - dezinstalare \textit{pachet} și a fișierelor de configurație
  \item \textbf{-L\/\texttt{--}listfile pachet} - listarea fișierelor instalate de \textit{pachet}
  \item \textbf{-c\/\texttt{--}contents f.deb} - listarea fișierelor dintr-un .deb
  \item \textbf{-S\/\texttt{--}search file} - caută pachetele care instalează fișierul \textit{file} \\
    ex. \textit{dpkg -S /bin/ls} \\
    coreutils: /bin/ls
  \end{itemize}
\end{frame}



\begin{frame}[allowframebreaks] % Spargem paginile mai mari automat
  % Atenţie: allowframebreaks nu funcţionează cu overlay-uri
  \frametitle{Câte ceva despre istoria FLOSS}
  \begin{itemize}
  \item Punctul de plecare: noua imprimantă de la MIT
  \item Proiectul GNU, activismul lui Richard Stallman (1983)
  \item Cele patru drepturi. Programele libere nu sunt neapărat gratuite
  \item Minix, sistem de operare UNIX educațional. Micronucleu (1987)
  \item GNU: un sistem de operare liber dar incomplet. Hurd nu este gata (1990)
  \item Linux: nucleul monolitic devenit liber (1991). Primele distribuții
    GNU/Linux
  \item Afaceri. Companiile multinaționale devin și ele contribuitori la
    programele libere
  \item OSI și definiția Open Source (1998). Diferențe de perspectivă și
    aplicabilitate
  \item Patentele pe programe din SUA amenință programele libere. Procese (2004
    - prezent)
  \item Microsot, Adobe și Nokia încep să dezvolte programe Open Source, nu
    toate libere (2008)
  \item Echipa Debian se alătură efortului internațional de a termina
    revoluționarul GNU Hurd
  \end{itemize}
\end{frame}

\section{Cod}
\frame{\tableofcontents[currentsection]}

\begin{frame}
  \frametitle{Exemplu de cod}
  Acesta este un script bash:\\
  \noindent
\ttfamily
\hlstd{}\hlline{\ \ \ \ 1\ }\hlslc{\#!/bin/bash}\\
\hlline{\ \ \ \ 2\ }\hlstd{\\
\hlline{\ \ \ \ 3\ }MAX}\hlsym{=}\hlstd{}\hlnum{10}\\
\hlline{\ \ \ \ 4\ }\hlstd{}\\
\hlline{\ \ \ \ 5\ }\hlkwa{for\ }\hlstd{i\ }\hlkwa{in\ }\hlstd{\$}\hlsym{(}\hlstd{}\hlkwc{seq\ }\hlstd{}\hlnum{1\ }\hlstd{}\hlkwd{\$\{MAX\}}\hlstd{}\hlsym{);\ }\hlstd{}\hlkwa{do}\\
\hlline{\ \ \ \ 6\ }\hlstd{}\hlstd{\ \ \ \ \ \ \ \ }\hlstd{}\hlkwb{echo\ }\hlstd{}\hlstr{"Testing\ \$\{i\}"}\hlstd{}\\
\hlline{\ \ \ \ 7\ }\hlkwa{done}\\
\hlline{\ \ \ \ 8\ }\hlstd{}\\
\hlline{\ \ \ \ 9\ }\hlkwb{exit\ }\hlstd{}\hlnum{0}\hlstd{}\\
\mbox{}
\normalfont
 % includem codul
\end{frame}

\begin{frame}{Încă un exemplu de cod}
  Acesta este un program C:

  \noindent
\ttfamily
\hlstd{}\hlline{\ \ \ \ 1\ }\hldir{\#include\ $<$stdio.h$>$}\\
\hlline{\ \ \ \ 2\ }\hlstd{}\hldir{\#include\ $<$stdlib.h$>$}\\
\hlline{\ \ \ \ 3\ }\hlstd{}\\
\hlline{\ \ \ \ 4\ }\hldir{\#define\ WHATEVER\ 1}\\
\hlline{\ \ \ \ 5\ }\hlstd{}\\
\hlline{\ \ \ \ 6\ }\hlkwb{int}\\
\hlline{\ \ \ \ 7\ }\hlstd{}\hlkwd{main}\hlstd{}\hlsym{(}\hlstd{}\hlkwb{int\ }\hlstd{argc}\hlsym{,\ }\hlstd{}\hlkwb{char\ }\hlstd{}\hlsym{{*}{*}}\hlstd{argv}\hlsym{)}\\
\hlline{\ \ \ \ 8\ }\hlstd{}\hlsym{\{}\\
\hlline{\ \ \ \ 9\ }\hlstd{}\hlstd{\ \ \ \ \ \ \ \ }\hlstd{}\hlkwd{puts}\hlstd{}\hlsym{(}\hlstd{}\hlstr{"Hello,\ World!}\hlesc{$\backslash$n}\hlstr{"}\hlstd{}\hlsym{);}\\
\hlline{\ \ \ 10\ }\hlstd{\\
\hlline{\ \ \ 11\ }}\hlstd{\ \ \ \ \ \ \ \ }\hlstd{}\hlkwa{return\ }\hlstd{}\hlnum{0}\hlstd{}\hlsym{;}\\
\hlline{\ \ \ 12\ }\hlstd{}\hlsym{\}}\hlstd{}\\
\mbox{}
\normalfont

\end{frame}

\section{Overlay-uri}
\frame{\tableofcontents[currentsection]}

\begin{frame}{Comanda `pause'}
  `Pause' ne lasă să arătăm textul `gradual'.

  \pause Putem să folosim pause cu itemize sau orice altă facilitate.
  \begin{itemize}
    \pause \item De exemplu.
    \pause \item Încă un exemplu.
    \pause \item Dar nu e foarte flexibil...
  \end{itemize}
\end{frame}

\begin{frame}{Overlay-uri mai avansate}
  \only<1>{Cue chapter 9.3 from the Beamer manual.}
  \only<2>{Facem o mare ciorbă de overlay-uri. \textbf{Notă:} E vorba de un
    singur slide -- numărul şi titlul se păstrează.}
  \begin{itemize}
  \item<3-> orice comandă poate fi urmată (fără spaţiu) de un specificator de
    overlay;
  \item<4-> specificatoarele sunt marcate cu \texttt{<} şi \texttt{>};
  \item<5-> \texttt{<4>} înseamnă ``doar pe overlay-ul 4'';
  \item<5-> \texttt{<1-3>} înseamnă ``doar pe overlay-urile de la 1 până la 3,
    inclusiv'';
  \item<6-> \texttt{<5->} înseamnă ``de la overlay-ul 5 până la sfârşit'' (de
    fapt, până la slide-ul următor, fireşte);
  \item<6-> \texttt{<-4>} înseamnă ``de la început până la overlay-ul 4'';
  \item<7-> Toate exemplele de mai sus au avut comanda \texttt{item} urmată de
    un overlay. Trebuie să ţinem cont de ordine manual, dar nu cred că vom avea
    nevoie de cine-ştie-ce efecte.
  \end{itemize}
\end{frame}


\end{document}
