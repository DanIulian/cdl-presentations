% vim: set tw=78 aw:
\documentclass{beamer}

\usepackage[utf8x]{inputenc} % diacritice
\usepackage[romanian]{babel}
\usepackage{hyperref}        % folositi \url{http://...}
% sau \href{http://...}{Nume Link}
\mode<presentation>
{ \usetheme{Rochester} }		% TODO: settle this

% Titlul nu foloseşte Unicode pentru că e o problemă căreia nu i-am dat de
% cap.
\title[GNU Privacy Guard]{GNU Privacy Guard}
\subtitle{CDL - Cursul 5}
\institute{Ceata}
\author{Tibi Turbureanu}

\begin{document}

% Slide-urile cu mai multe părţi sunt marcate cu textul (cont.)
\setbeamertemplate{frametitle continuation}[from second]
% Arătăm numărul frame-ului
\setbeamertemplate{footline}[frame number]

\frame{\titlepage}

\frame{\tableofcontents}

% NB: Secţiunile nu sunt marcate vizual, ci doar apar în cuprins.
\section{No\c{t}iuni generale}

% Pentru reamintirea periodică a cuprinsului şi unde ne aflăm:
\frame{\tableofcontents[currentsection]}

% Titlul unui frame se specifică fie în acolade, imediat după \begin{frame},
% fie folosind \frametitle
\begin{frame}{PGP}
  \begin{itemize}
    \item Pretty Good Privacy
    \item original scris de Philip Zimmermann în 1991
    \item \textbf{programul} îndeplinește două funcții: 
    \begin{itemize}
      \item semnarea digitală a fișierelor
      \item criptarea și decriptarea fișierelor
    \end{itemize} 
    \item des folosit pentru semnarea și criptarea/decriptarea scrisorilor
    electronice
    \item semnarea se bazează pe algoritmii RSA sau DSA
    \item criptarea este asimetrică (cheie publică, cheie privată)
    \item criptarea se bazează pe algoritmul patentat IDEA
    \item programul PGP este proprietar
  \end{itemize}
\end{frame}

\begin{frame}{OpenPGP}
  \begin{itemize}
    \item \textbf{standard} deschis pentru PGP
    \item este în continuă dezvoltare
    \item specificația actuală este dată de RFC 4880 (2007)
    \item PGP compilează cu standardul OpenPGP
  \end{itemize}
\end{frame}

\begin{frame}{GnuPG sau GPG (1)}
  \begin{itemize}
    \item GNU Privacy Guard
    \item original scris de Werner Koch în 1999
    \item \textbf{programul} îndeplinește aceleași două funcții:
    \begin{itemize}
      \item semnarea digitală a fișierelor
      \item criptarea și decriptarea fișierelor
    \end{itemize}
    \item apărut după lansarea standardului OpenPGP
    \item compilează cu OpenPGP și deci este compatibil cu PGP
    \item este program liber (GPLv3)
  \end{itemize}
\end{frame}

\begin{frame}{GnuPG sau GPG (2)}
  \begin{itemize}
    \item semnarea se bazează pe algoritmii RSA sau DSA
    \item criptarea este hibridă
    \begin{itemize}
      \item criptare simetrică pentru viteză
      \item criptare asimetrică pentru securitate sporită
      \item criptarea se bazează pe algoritmi nepatentați (publici) ca:
      ElGamal, CAST5, Triple DES, AES, Blowfish, Twofish
    \end{itemize}
    \item realizarea documentației și portarea pe Windows a fost finanțată de 
    Guvernul German
  \end{itemize}
\end{frame}

\section{Comenzi}
\frame{\tableofcontents[currentsection]}

\begin{frame}{Generarea unei perechi de chei}
  \begin{itemize}
    \item \texttt{\$ gpg}
    \begin{itemize}
      \item crearea fișierelor de configurare
    \end{itemize}
    \item \texttt{\$ gpg --gen-key}
    \begin{itemize}
      \item implicit DSA pentru semnare
      \item implicit ElGamal pentru criptare/decriptare
      \item implicit dimensiunea cheii DSA este de 1024 biți
      \item implicit dimensiunea cheii ElGamal este de 2048 biți
      \item implicit cheia nu expiră
      \item informații: nume, adresă poștală, alint (comentariu)
      \item ex. "Tibi Turbureanu (tct) tct@ceata.org"
      \item parolă: de preferat o propoziție scrisă doar cu inițialele
      cuvintelor, litere mari și mici, cifre, caractere non-alfanumerice
    \end{itemize}
  \end{itemize}
\end{frame}

\begin{frame}{Schimbul de chei publice cu alți utilizatori (1)}
  \begin{itemize}
    \item \texttt{\$ gpg --armor --output tct-gpg-pub-key --export
    tct@ceata.org}
    \begin{itemize}
      \item exportul cheii publice pentru adresa \texttt{tct@ceata.org}
    \end{itemize}
    \item \texttt{\$ gpg --import mariusl-pub-key}
    \begin{itemize}
      \item importul cheii publice a lui Marius Latu
    \end{itemize}
  \end{itemize}
\end{frame}

\begin{frame}{Schimbul de chei publice cu alți utilizatori (2)}
  \begin{itemize}
    \item \texttt{\$ gpg --edit-key marius@ceata.org}
    \item \texttt{Command> }
    \begin{itemize}
      \item pentru siguranță în folosirea cheii, este \textbf{necesară}
      verificarea identității autorului (este el într-adevăr Marius Latu?)
      \item \texttt{Command> fpr}
      \item verifică prin telefon sau \textbf{cel mai bine} față-în-față că
      semnătura (\texttt{fingerprint}) coincide cu cea dictată de presupusul
      autor (Marius?)
      \item dacă corespunde, atunci semneaz-o:
      \item \texttt{Command> sign}
    \end{itemize}
  \end{itemize}
\end{frame}

\begin{frame}{Semnarea și verificarea fișierelor}
  \begin{itemize}
    \item nu presupune criptarea fișierelor
    \item este o măsură de asigurare a autenticității fișierului
    \item \texttt{\$ gpg --clearsign tct-file}
    \begin{itemize}
      \item \texttt{Enter passphrase: }
      \item se generează fișierul semnat \texttt{tct-file.asc}
    \end{itemize}
    \item \texttt{\$ gpg --verify tct-file.asc}
    \begin{itemize}
      \item utilizatorii care au semnat cheia publică a autorului (tct) vor 
      putea să verifice autenticitatea fișierului
    \end{itemize}
  \end{itemize}
\end{frame}

\begin{frame}{Criptarea și decriptarea fișierelor (1)}
  \begin{itemize}
    \item este o măsură de asigurare a securității unui fișier privat
    \item \textbf{prima metodă}: fișierul va putea fi citit de un
    \textbf{singur} utilizator
    \item \texttt{\$ gpg --output tct-file.gpg --encrypt --recipient
    marius@ceata.org tct-file}
    \item \texttt{\$ gpg --output tct-file --decrypt tct-file.gpg}
    \begin{itemize}
      \item \texttt{Enter passphrase: }
      \item destinatarul (Marius) decriptează fișierul introducând parola de
      deblocare a cheii lui private
    \end{itemize}
  \end{itemize}
\end{frame}

\begin{frame}{Criptarea și decriptarea fișierelor (2)}
  \begin{itemize}
    \item \textbf{a doua metodă}: fișierul va putea fi citit de un
    \textbf{grup} de utilizatori folosind o \textbf{parolă comună} atent
    distribuită
    \item \texttt{\$ gpg --output tct-file.gpg --symmetric tct-file}
    \item \texttt{\$ gpg --output tct-file --decrypt tct-file.gpg}
    \begin{itemize}
      \item \texttt{Enter passphrase: }
      \item destinatarii decriptează fișierul introducând parola 
      partajată de autorul fișierului
    \end{itemize}
  \end{itemize}
\end{frame}

\section{No\c{t}iuni avansate}
\frame{\tableofcontents[currentsection]}

\begin{frame}{Utilizări avansate (1)}
  \begin{itemize}
    \item clienți de poștă electronică (Evolution, Mutt) cu suport pentru GPG
    \item Mailman cu suport pentru GPG (\texttt{patch})
    \item folosirea serverelor de chei publice (\texttt{keyservers})
    \item autentificarea prin OpenSSH folosind cheia GPG
    \item pachet complet de securitate cu cardul inteligent :-D (FSFE Smart
    Card)
  \end{itemize}
\end{frame}

\begin{frame}{Utilizări avansate (2)}
  \begin{itemize}
    \item semnarea versiunilor oficiale de programe
    \item Exemplu:
    \begin{itemize}
      \item la fiecare lansare Mailman, dezvoltatorii principali (Barry
      Warshaw, Mark Sapiro și Tokio Kikuchi) semnează arhiva cu cheile lor
      private
      \item cheile dezvoltatorilor principali ai Poștașului ;-) sunt
      public disponibile pe saitul oficial, așa încât orice utilizator să
      poată verifica autenticitatea distribuției de Mailman pe care o deține
    \end{itemize}
  \end{itemize}
\end{frame}

\begin{frame}{KSP}
  \begin{itemize}
    \item Key Signing Party
    \item verificare de chei în „masă” :-D
    \item „petrecere” pentru că este distractiv să iei contact cu membri
    activi ai comunităților din jurul programelor libere, să socializezi,
    să realizezi schimburi culturale :-)
    \item o ocazie bună de ieșit pe urmă la bere
    \item Exemplu: în fiecare an la FOSDEM (Joost van Baal), anul acesta la
    eLiberatica
    \item utilizarea \texttt{caff} pentru semnarea tuturor cheilor verificate
    la petrecere, folosind poșta electronică
  \end{itemize}
\end{frame}

\section{Bibliografie}
\frame{\tableofcontents[currentsection]}

\begin{frame}{Bibliografie (1)}
  \begin{itemize}
    \item \htmladdnormallink{http://en.wikipedia.org/wiki/Pretty\_Good\_Privacy}
    {http://en.wikipedia.org/wiki/Pretty\_Good\_Privacy}
    \item \htmladdnormallink{http://en.wikipedia.org/wiki/OpenPGP\#OpenPGP} 
    {http://en.wikipedia.org/wiki/OpenPGP\#OpenPGP}
    \item \htmladdnormallink{http://en.wikipedia.org/wiki/GNU\_Privacy\_Guard}
    {http://en.wikipedia.org/wiki/GNU\_Privacy\_Guard}
    \item \htmladdnormallink{http://en.wikipedia.org/wiki/RSA}
    {http://en.wikipedia.org/wiki/RSA}
    \item
    \htmladdnormallink{http://en.wikipedia.org/wiki/Digital\_Signature\_Algorithm}
    {http://en.wikipedia.org/wiki/Digital\_Signature\_Algorithm}
    \item \htmladdnormallink{http://en.wikipedia.org/wiki/Elgamal}
    {http://en.wikipedia.org/wiki/Elgamal}
    \item \htmladdnormallink{http://www.eecs.umich.edu/dco/faq/gpg/}
    {http://www.eecs.umich.edu/dco/faq/gpg/}
    \item \htmladdnormallink{http://mdcc.cx/gnupg/gpg\_5\_min.html}
    {http://mdcc.cx/gnupg/gpg\_5\_min.html}
  \end{itemize}
\end{frame}

\begin{frame}{Bibliografie (2)}
  \begin{itemize}
    \item \htmladdnormallink{http://non-gnu.uvt.nl/mailman-pgp-smime}
    {http://non-gnu.uvt.nl/mailman-pgp-smime/}
    \item
    \htmladdnormallink{http://www.gnu.org/software/mailman/download.html}
    {http://www.gnu.org/software/mailman/download.html}
    \item
    \htmladdnormallink{http://www.fsfe.org/en/fellows/tyrael/fsfe\_card\_complete
    \_gnupg\_ssh\_login\_how\_to}
    {http://www.fsfe.org/en/fellows/tyrael/fsfe\_card\_complete\_gnupg\_ssh\_login\_how\_to}
    \item \htmladdnormallink{http://en.wikipedia.org/wiki/Key\_signing\_party}
    {http://en.wikipedia.org/wiki/Key\_signing\_party}
    \item
    \htmladdnormallink{http://manpages.ubuntu.com/manpages/jaunty/
    en/man1/caff.1.html}
    {http://manpages.ubuntu.com/manpages/jaunty/en/man1/caff.1.html}
    \item \htmladdnormallink{http://ksp.mdcc.cx/} {http://ksp.mdcc.cx/}
    \item \htmladdnormallink{http://mdcc.cx/pgp-kspkeyserver}
    {http://mdcc.cx/pgp-kspkeyserver/}
  \end{itemize}
\end{frame}

\end{document}
