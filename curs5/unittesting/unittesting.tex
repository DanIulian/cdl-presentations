% vim: set tw=78 aw sw=2 sts=2 noet:
\documentclass{beamer}

%\includeonlyframes{c} -- speeding up compilation speed during debug

\usepackage[utf8x]{inputenc} % diacritice
\usepackage[romanian]{babel}
\usepackage{hyperref}        % folositi \url{http://...}
\mode<presentation>
\usetheme{CDL}

\title[]{Unit testing}
\subtitle{CDL - Cursul 5}
\institute[]{ROSEdu}
\author[]{Mihai Maruseac \\ \texttt{mihai.maruseac@rosedu.org}}

\setbeamertemplate{frametitle continuation}[from second]
\setbeamertemplate{footline}[frame number]

\pgfdeclareimage[height=5cm]{m0}{img/m0}
\pgfdeclareimage[height=7cm]{m1}{img/m1}
\pgfdeclareimage[height=7cm]{m2}{img/m2}
\pgfdeclareimage[height=7cm]{m3}{img/m3}

\begin{document}

\maketitle

\begin{frame}{Before we begin}
  \begin{block}{$>$ echo "Test" $|$ grep T}
    Test\\
    (linux.c, 36): Cannot wait for child: No child processes
  \end{block}
  \begin{block}{$>$ echo "Test" $|$ grep T}
    Test\\
    $>$ exit
  \end{block}
\end{frame}

\begin{frame}{Studiu de caz}
  \transwipe[direction=90,duration=2]
  \begin{pgfpicture}{0cm}{0cm}{10cm}{5cm}
    \pgfsetcolor{red}
    \pgfrect[stroke]{\pgfpoint{3cm}{1cm}}{\pgfpoint{6cm}{1.5cm}}
  \end{pgfpicture}

  1. Pornim cu un nucleu mic...
\end{frame}

\begin{frame}{Studiu de caz}
  \begin{pgfpicture}{0cm}{0cm}{10cm}{5cm}
    \pgfsetcolor{red}
    \pgfline{\pgfpoint{3cm}{1cm}}{\pgfpoint{3cm}{2.5cm}}
    \pgfline{\pgfpoint{3cm}{2.5cm}}{\pgfpoint{5cm}{2.5cm}}
    \pgfline{\pgfpoint{5cm}{2.5cm}}{\pgfpoint{6cm}{1.5cm}}
    \pgfline{\pgfpoint{6cm}{1.5cm}}{\pgfpoint{7cm}{2.5cm}}
    \pgfline{\pgfpoint{7cm}{2.5cm}}{\pgfpoint{9cm}{2.5cm}}
    \pgfline{\pgfpoint{9cm}{2.5cm}}{\pgfpoint{9cm}{1cm}}
    \pgfline{\pgfpoint{9cm}{1cm}}{\pgfpoint{3cm}{1cm}}
    \pgfsetcolor{blue}
    \pgfline{\pgfpoint{5cm}{2.52cm}}{\pgfpoint{5cm}{4cm}}
    \pgfline{\pgfpoint{5cm}{4cm}}{\pgfpoint{7cm}{4cm}}
    \pgfline{\pgfpoint{7cm}{4cm}}{\pgfpoint{7cm}{2.52cm}}
    \pgfline{\pgfpoint{7cm}{2.52cm}}{\pgfpoint{6cm}{1.52cm}}
    \pgfline{\pgfpoint{6cm}{1.52cm}}{\pgfpoint{5cm}{2.52cm}}
  \end{pgfpicture}

  2. Adăugăm la el element cu element...
\end{frame}

\begin{frame}{Studiu de caz}
  \pgfuseimage{m0}

  3. Și ajungem aici.
\end{frame}

\begin{frame}{Soluții}
  \begin{itemize}[<+->]
    \item Ne structurăm mai bine codul de la început.
    \item Putem ajunge la buguri ce vor fi vizibile abia la integrarea a două
    porțiuni de cod.
    \item Sau, putem testa fiecare unitate de cod (funcție, clasă etc.)
  \end{itemize}
\end{frame}

\section{Unit Testing}

\begin{frame}{Ce presupune unit testing?}
  \begin{description}[<+->]
    \item[garanție]  = code that just works
    \item[unitate] = cea mai mică bucată independentă din cod
    \item[scop] = izolare bucăți cod, demonstrare corectitudine
  \end{description}
\end{frame}

\begin{frame}{Unit testing și scrierea codului}
  \begin{description}[<+->]
    \item[ÎNAINTE] : detaliere aspecte proiect într-o manieră folositoare
    \item[SCRIERE] : se evită scrierea de prea mult cod
    \item[MODIFICARE] : garantează că nu se strică ce mergea deja
  \end{description}
\end{frame}

\begin{frame}{Cum testăm?}
  \begin{itemize}[<+->]
    \item manual: \textbf{assert} sau \textbf{if}
    \item automat: folosim un framework existent
  \end{itemize}
\end{frame}

\begin{frame}{Exercițiu 1}
  \begin{alertblock}{Task 1}
    Scrieți o funcție mystrrev ce va realiza inversarea unui șir de caractere
    dat argument. Funcția va returna un pointer către șirul rezultat.
  \end{alertblock}
  \pause
  \begin{exampleblock}{Task 2}
    Scrieți o funcție check\_mystrrev ce va testa funcția anterioară:\\
    mystrrev(mystrrev(sir)) == sir
  \end{exampleblock}
  \pause
  Ați tratat și cazul argumentelor nule?\\
  \pause
  Ați verificat memoria?
\end{frame}

\section{Test Driven Development}

\begin{frame}{Testing vs Coding}
  \begin{itemize}[<+->]
    \item Pe ce punem mai mult accent? Scris teste sau scris cod?
    \item ...
    \item ...
    \item Depinde
  \end{itemize}
\end{frame}

\begin{frame}{Test Driven Development (TDD)}
  \begin{itemize}[<+->]
    \item scriem testele întâi
    \item de fiecare dată când modificăm codul verificăm să nu crească numărul
    testelor picate (și care sunt ele)
    \item pică un test trecut anterior = am modificat ceva ce nu trebuia
    modificat (\textbf{don't commit})
    \item toate testele trecute = proiect terminat
    \item apare un bug nou $\rightarrow$ se scriu noi teste și repetăm
    secvența
  \end{itemize}
\end{frame}

\begin{frame}{1 pic = 1000 words}
  \pgfuseimage{m1}
\end{frame}

\begin{frame}{Exercițiu 2}
  \begin{block}{Descriere}
    Vom implementa funcția mystrcmp ce va compara dacă două șiruri de
    caractere sunt egale. Vom începe prin a defini testul și apoi vom scrie
    codul.
  \end{block}
  \begin{block}{Task 1}
    Scrieți o funcție de test ce va verifica că mystrcmp returnează ce
    trebuie. Nu implementați mystrcmp, scrieți doar un stub.
  \end{block}
  \pause
  \begin{block}{Task 2}
    Implementați mystrcmp, pe etape. Rulați funcția de test de fiecare dată
    când compilați.
  \end{block}
\end{frame}

\begin{frame}{Timeline}
  \pgfuseimage{m2}
\end{frame}

\begin{frame}{6 motive pentru TDD}
  \begin{enumerate}[<+->]
    \item Focus
    \item Știi când ai terminat
    \item Dacă un unit merge o dată va merge mereu
    \item Cod mult mai modular
    \item Cod mai curat și mai ușor de înțeles
    \item Satisfacția de a vedea că lucrurile merg
    \item Progres vizibil
    \item ``Documentare'' cod
  \end{enumerate}
\end{frame}

\begin{frame}{No silver bullet}
  \begin{itemize}[<+->]
    \item Nu merge mereu
    \item Prea multe teste
    \item Linii de test de 100 de ori mai multe ca linii de cod utile
    \item Teste nesemnificative
    \item Teste pe cazuri nespecificate
    \item Teste incorect specificate
    \item Teste depinzând de implementare
    \item Uneori testarea durează prea mult
    \item Testarea manuală poate fi sărită
    \item not thinking ahead
  \end{itemize}
\end{frame}

\begin{frame}{Not thinking ahead}
  \pgfuseimage{m3}
\end{frame}

\section{no TDD uses}

\begin{frame}{Testarea codului scris deja}
  \begin{itemize}[<+->]
    \item goes without saying
    \item se poate face manual, prin script sau prin cod
    \item ignorată
  \end{itemize}
\end{frame}

\begin{frame}{Integrarea cu SCM-uri}
  \begin{itemize}[<+->]
    \item \textbf{git bisect}
    \item testăm manual?
    \item putem face un script pentru testare
  \end{itemize}
\end{frame}

\begin{frame}{Integrarea cu SCM-uri - Demo}
  \begin{block}{Descriere}
    Descărcați sursele unui proiect aflate pe git. Repository-ul are adresa:\\
    git://gitorious.org/pali/pali.git\\
    Avem de afișat toate palindroamele dintr-o frază.
  \end{block}
  \begin{block}{Problema}
    Compilați sursele și rulați programul. Vom încerca să găsim commit-ul ce a
    introdus eroarea folosind git bisect.
  \end{block} \pause
  \begin{block}{Soluție}
    Descărcați cele două fișiere bash de la\\
    http://swarm.cs.pub.ro/$\sim$mihai/cdl/2010/.\\
    Rulați ut.sh și veți obține commit-ul ce a introdus greșeala. Corectați-o
    introducând un nou commit corect (nu faceți push).
  \end{block}
\end{frame}

\begin{frame}{Stress testing}
  \begin{itemize}[<+->]
    \item Robustețea programului
    \item Teste peste limitele operării normale
    \item Paralelism, sisteme de operare, middleware, biblioteci, memory leaks
    \item Zburați cu un avion al cărui software nu a fost testat astfel?
  \end{itemize}
\end{frame}

\begin{frame}{Concluzii}
  \begin{itemize}[<+->]
    \item Testing is good \pause \alert{if done properly}
    \item \textit{``Testing can show the presence of bugs, but not the absence''}
    \item suntem siguri că testăm tot?
    \item \textit{``Testing is the process of executing a program with the intent of finding errors''}
    \item Testarea codului duce și la cod lizibil, module separate etc.
  \end{itemize}
\end{frame}

\begin{frame}{Întrebări?}
\end{frame}

\begin{frame}{Bibliografie}
  \begin{itemize}
    \item \href{http://www.agiledata.org/essays/tdd.html}{Test Driven Development}
    \item \href{http://www.lispcast.com/uncategorized/6-reasons-to-develop-your-tests-first/}{Motive pentru TDD}
    \item \href{http://stackoverflow.com/questions/301693/why-didnt-unit-testing-work-out-for-your-project}{Eșecuri TDD}
    \item \href{http://docs.python.org/library/unittest.html}{Unit testing în Python}
    \item \href{http://check.sourceforge.net/}{Framework pentru unit testing în C}
    \item \href{http://www.cs.chalmers.se/~rjmh/QuickCheck/}{QuickCheck - Haskell}
    \item \href{http://www.junit.org/}{JUnit - Java}
    \item \href{http://www.youtube.com/watch?v=XP4o0ArkP4s}{http://www.youtube.com/watch?v=XP4o0ArkP4s}
  \end{itemize}
\end{frame}

\end{document}
