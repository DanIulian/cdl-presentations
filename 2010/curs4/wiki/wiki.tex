% vim: set tw=78 sts=2 sw=2 ts=8 aw et:
\documentclass{beamer}

\usepackage[utf8x]{inputenc}		% diacritice
\usepackage[romanian]{babel}
%\usepackage{color}			% highlight
%\usepackage{alltt}			% highlight
%\usepackage{code/highlight}		% highlight
\usepackage{hyperref}			% folosiți \url{http://...}
                                        % sau \href{http://...}{Nume Link}
\usepackage{verbatim}
\usepackage{tabularx}

\mode<presentation>
\usetheme{CDL}

% Titlul nu foloseşte Unicode pentru că e o problemă căreia nu i-am dat de
% cap.
\title[Wiki-uri]{Wiki-uri}
\subtitle{CDL - Cursul 4}
\institute[ROSEdu]{ROSEdu}
\date{20 martie 2010}
\author{Răzvan Deaconescu \\ \texttt{razvan@rosedu.org}}

\begin{document}

% Slide-urile cu mai multe părți sunt marcate cu textul (cont.)
\setbeamertemplate{frametitle continuation}[from second]
% Arătăm numărul frame-ului
\setbeamertemplate{footline}[frame number]

\frame{\titlepage}

\frame{\tableofcontents}


% NB: Secțiunile nu sunt marcate vizual, ci doar apar în cuprins.
\section{No\textcommabelow{t}iuni generale}

% Pentru reamintirea periodică a cuprinsului și unde ne aflăm:
\frame{\tableofcontents[currentsection]}

% Titlul unui frame se specifică fie în acolade, imediat după \begin{frame},
% fie folosind \frametitle

\begin{frame}{De ce wiki-uri?}
  \begin{itemize}
    \item TODO
  \end{itemize}
\end{frame}

\begin{frame}{Ce este un wiki?}
  \begin{itemize}
    \item TODO
  \end{itemize}
\end{frame}

\begin{frame}{Wiki versus \ldots}
  \begin{itemize}
    \item TODO
  \end{itemize}
\end{frame}

\begin{frame}{Cazuri de utilizare}
  \begin{itemize}
    \item TODO
  \end{itemize}
\end{frame}

\begin{frame}{Exemple cunoscute}
  \begin{itemize}
    \item TODO
  \end{itemize}
\end{frame}

\section{DokuWiki}

% Pentru reamintirea periodică a cuprinsului și unde ne aflăm:
\frame{\tableofcontents[currentsection]}

\begin{frame}{De ce DokuWiki?}
  \begin{itemize}
    \item TODO
  \end{itemize}
\end{frame}

\begin{frame}{Sintaxă Creole}
  \begin{itemize}
    \item TODO
  \end{itemize}
\end{frame}

\begin{frame}{Hints}
  \begin{itemize}
    \item TODO
  \end{itemize}
\end{frame}

\section{Redmine Wiki}

% Pentru reamintirea periodică a cuprinsului și unde ne aflăm:
\frame{\tableofcontents[currentsection]}

\begin{frame}{De ce Redmine?}
  \begin{itemize}
    \item TODO
  \end{itemize}
\end{frame}

\begin{frame}{Sintaxă Redmine wiki}
  \begin{itemize}
    \item TODO
  \end{itemize}
\end{frame}


\section{Concluzii}

% Pentru reamintirea periodică a cuprinsului și unde ne aflăm:
\frame{\tableofcontents[currentsection]}

\begin{frame}{Comparison chart}
  \begin{center}
    \begin{tabular}{|c|c|c|}
      \hline
      \textbf{Format} & \textbf{Creole} & \textbf{Redmine} \\
      \hline
      \hline
      heading 1 & = Nume = & .h1 Nume \\
      \hline
    \end{tabular}
  \end{center}
\end{frame}

\begin{frame}{Cuvinte cheie}
  \begin{columns}
    \begin{column}[l]{0.5\textwidth}
      \begin{itemize}
        \item TODO
      \end{itemize}
    \end{column}
    \begin{column}[l]{0.5\textwidth}
      \begin{itemize}
        \item TODO
      \end{itemize}
    \end{column}
  \end{columns}
\end{frame}

\begin{frame}{Link-uri utile}
  \begin{itemize}
    \item TODO
  \end{itemize}
\end{frame}


\section{\^{I}ntrebări}

% Pentru reamintirea periodică a cuprinsului și unde ne aflăm:
\frame{\tableofcontents[currentsection]}

\end{document}
