% vim: set tw=78 aw:
\documentclass{beamer}

\usepackage[utf8x]{inputenc} % diacritice
\usepackage[romanian]{babel}
\usepackage{color}			 % highlight
\usepackage{alltt}			 % highlight
\usepackage{code/highlight}	 % highlight
\usepackage{hyperref}        % folositi \url{http://...}
% sau \href{http://...}{Nume Link}
\mode<presentation>
\usetheme{CDL}

% Titlul nu foloseşte Unicode pentru că e o problemă căreia nu i-am dat de
% cap.
\title[Design Patterns]{Design Patterns}
\subtitle{CDL - Cursul 4}
\institute[ROSEdu]{ROSEdu}
\author[VictorU]{Victor Ungureanu \\ \texttt{victor.ungureanu@cti.pub.ro} }

\begin{document}

% Slide-urile cu mai multe părţi sunt marcate cu textul (cont.)
\setbeamertemplate{frametitle continuation}[from second]
% Arătăm numărul frame-ului
\setbeamertemplate{footline}[frame number]

\frame{\titlepage}

\frame{\tableofcontents}


\section{"Tiparul modelului"}

\begin{frame}{Întrebarea}
	\begin{itemize}
  		\item Design software
  		\vspace{4mm}
  		\item Mai multe probleme:
  		\begin{itemize}
	  		\item comune
  			\item similare
  			\item repetitive
  		\end{itemize}
	\end{itemize}
\end{frame}

\begin{frame}{Răspunsul}
	\begin{itemize}
		\item \textbf{Design Patterns}
		\vspace{4mm}
  		\item Soluţie:
  		\begin{itemize}
	  		\item generală
  			\item reutilizabilă
  			\item care rezolvă probleme comune
  		\end{itemize}
	\end{itemize}
\end{frame}

\begin{frame}{Un design pattern}
	\begin{itemize}
	  	\item \textbf{nu} este un design în formă finală
	  	\vspace{4mm}
	  	\item \textbf{nu} poate fi transformat direct in cod
	  	\vspace{4mm}
	  	\item \textbf{este} o descriere a soluţiei
	  	\vspace{4mm}
	  	\item \textbf{este} un template ce poate fi aplicat pentru rezolvarea problemei
	\end{itemize}
\end{frame}

\begin{frame}{Next level}
	\begin{itemize}
	  	\item Design Pattern-urile $\in$ domeniului $($modulelor $\cup$ interconexiunilor$)$
	  	\vspace{4mm}
	  	\item La un \textbf{nivel mai înalt}: \textbf{pattern-urile arhitecturale} (Arhitectural Patterns) descriu:
	  		\begin{itemize}
	  			\item pattern-ul global utilizat
	  			\item întregul sistem
	  			\item exemplu: \textbf{Model–View–Controller} (MVC)
	  		\end{itemize}
	  	\vspace{4mm}
	  	\item Nu toate \textbf{pattern-urile software} sunt \textbf{design patterns}
	  		\begin{itemize}
	  			\item exemplu: algoritmii care rezolvă probleme computaţionale
	  		\end{itemize}
	\end{itemize}
\end{frame}

\begin{frame}{Categorii de design pattern-uri}
	\begin{itemize}
		\item \textbf{Creational Patterns} = delegare
			\begin{itemize}
				\item implementează mecanisme de creare a obiectelor
				\item exemple: \textbf{Singleton}, Factory
			\end{itemize}
		\vspace{4mm}
		\item \textbf{Structural Patterns} = agregare
			\begin{itemize}
				\item găsesc metode de a defini relaţiile dintre entităţi
				\item exemple: \textbf{Decorator}
			\end{itemize}
		\vspace{4mm}
		\item \textbf{Behavioral Patterns} = consultare
			\begin{itemize}
				\item definesc modul în care obiectele comunică între ele
				\item exemple: \textbf{Visitor}, Observer, Command
			\end{itemize}
	\end{itemize}
\end{frame}


\section{Singleton Pattern}

\begin{frame}{Descriere singleton pattern}
	\begin{itemize}
		\item Utilizat pentru a restricţiona numărul de instanţieri a unei clase la un singur obiect
		\vspace{4mm}
		\item \alert{Acest pattern îngreunează testarea aplicaţiei, deoarece introduce stări globale}
	\end{itemize}
\end{frame}

\begin{frame}{Utilizări singleton pattern}
	\begin{itemize}
		\item Ca parte integrată în pattern-urile Abstract Factory, Builder, Prototype
		\vspace{4mm}
		\item Pentru \textbf{obiectele ce reprezintă stări} sunt folosite deseori Singleton-uri
		\vspace{4mm}
		\item \textbf{Singleton} este de preferat \textbf{variabilelor globale} deoarece:
			\begin{itemize}
				\item nu poluează \textbf{namespace-ul global} cu variabile nenecesare 
				\item permite \textbf{iniţializarea întârziată} (lazy initialization) pentru a nu consuma inutil resursele sistemului
			\end{itemize}
	\end{itemize}
\end{frame}

\begin{frame}{Implementare singleton pattern}
	\begin{itemize}
		\item La baza pattern-ului Singleton stă o metodă ce permite crearea unei \textbf{noi instanţe a clasei} dacă aceasta nu există deja
		\vspace{4mm}
		\item Dacă instanţa există deja, atunci întoarce o \textbf{referinţă} către acel obiect
		\vspace{4mm}
		\item Pentru a asigura o singură instanţiere a clasei, \textbf{constructorul} trebuie făcut \textbf{private}
		\vspace{4mm}
		\item Diferenţa dintre o \textbf{clasă statică} şi un \textbf{Singleton} este aceea că Singleton-ul permite \textbf{instanţierea întârziată}
	\end{itemize}
\end{frame}

\begin{frame}{singleton.h}
	\footnotesize{\input{code/1singletonh}}
\end{frame}

\begin{frame}{singleton.cc}
	\footnotesize{\input{code/1singleton}}
\end{frame}

\begin{frame}{main.cc}
	\footnotesize{\input{code/1main}}
\end{frame}


\section{Visitor Pattern}

\begin{frame}{Descriere visitor pattern}
	\begin{itemize}
		\item Oferă o modalitate de a separa un \textbf{algoritm} de \textbf{structura} pe care acesta operează
		\vspace{4mm}
		% avantajul constă în faptul că
		\item Putem adăuga \textbf{noi posibilităţi de prelucrare} a structurii, fără să o modificăm
		\vspace{4mm}
		% extrapolând, folosing Visitor
		\item Putem adăuga noi funcţii care realizează \textbf{prelucrări} asupra unei \textbf{familii de clase}, fără a modifica \textbf{structura claselor}
	\end{itemize}
\end{frame}

\begin{frame}{Utilizări visitor pattern}
	\begin{itemize}
		\item Prelucrarea unei \textbf{structuri complexe}, ce cuprinde mai multe obiecte, având \textbf{tipuri diferite}
		\vspace{4mm}
		\item Definirea de \textbf{operaţii distincte} pe aceeaşi structură
		\vspace{4mm}
		\item Previne \textbf{poluarea interfeţelor claselor} implicate cu detalii ce apaţin unor algoritmi diferiţi
		\vspace{4mm}
		\item \textbf{Centralizarea} aspectelor legate de \textbf{acelaşi algoritm}, în timp ce se \textbf{separă} detaliile ce ţin de \textbf{algoritmi diferiţi}
		\vspace{4mm}
		\item \textbf{Simplificare} claselor prelucrate şi a claselor care prelucrează
			\begin{itemize}
				\item orice date specifice algoritmului rezidă în clase care prelucrează
			\end{itemize}
		\vspace{4mm}
		\item \textbf{Dublă flexibilitate}
			\begin{itemize}
				\item clase prelucrate - clase care prelucrează
			\end{itemize}
	\end{itemize}
\end{frame}

\begin{frame}{Implementare visitor pattern}
	\begin{itemize}
		\item O \textbf{interfaţă Visitor} - reprezintă un algoritm oarecare ce va putea vizita orice clasă
			\begin{itemize}
				\item definirea unei \textbf{metode visit()} pentru fiecare clasă ce va putea fi vizitată
			\end{itemize}
		\vspace{4mm}
		\item O \textbf{interfaţă Visitable} - a cărei \textbf{metodă accept(Visitor)} permite rularea unui algoritm pe structura curenta
		\vspace{4mm}
		\item Implementări ale metodei accept(Visitor), în clasele care solicită vizitarea \textbf{instanţei curente} de către \textbf{vizitator}
	\end{itemize}
\end{frame}

\begin{frame}{visitor.h}
	\footnotesize{\input{code/2visitorh}}
\end{frame}

\begin{frame}{visitor.h (cont.)}
	\footnotesize{\input{code/2visitorh2}}
\end{frame}

\begin{frame}{visitor.cc}
	\footnotesize{\input{code/2visitor}}
\end{frame}

\begin{frame}{main.cc}
	\footnotesize{\input{code/2main}}
\end{frame}


\section{Decorator Pattern}

\begin{frame}{Descriere decorator pattern}
	\begin{itemize}
		\item Permite \textbf{extinderea} (decorarea) \textbf{funcţionalităţilor} unui obiect la rulare \textbf{independent} de alte instanţe ale aceleaşi clase
		\vspace{4mm}
		\item Decoratorul este o clasă care împachetează clasa originală, oferind noi funcţionalităţi
		\vspace{4mm}
		\item Mai mulţi \textbf{decoratori} pot fi \textbf{folosiţi unul peste altul}
			\begin{itemize}
				\item adaugă \textbf{funcţionalităţi noi}
			\end{itemize}
	\end{itemize}
\end{frame}

\begin{frame}{Utilizări decorator pattern}
	\begin{itemize}
		\item Atunci când sunt multe \textbf{moduri independente} de a extinde funcţionalitate unei clase
		\vspace{4mm}
		\item Atunci când este nevoie de \textbf{flexibilitate în proiectarea} funcţionalităţilor unei clase
		\vspace{4mm}
		\item În decorarea \textbf{ferestrelor}
			\begin{itemize}
				\item scrollbars (vertical, horizontal), butoane (minimize, maximize) etc.
			\end{itemize}
		\vspace{4mm}
		\item În implementările librăriilor de \textbf{intrare/ieşire} (input/output - I/O)
			\begin{itemize}
				\item Java I/O Streams
			\end{itemize}
	\end{itemize}
\end{frame}

\begin{frame}{Implementare decorator pattern}
	\begin{itemize}
		\item Nivelul \textbf{clasei decorate} pe lanţul de moştenire este cel mult egal cu nivelul \textbf{decoratorului}
		\vspace{4mm}
		\item Clasa ce se doreşte a fi \textbf{decorată devine} un \textbf{câmp al clasei decorator}
		\vspace{4mm}
		\item \textbf{Constructorul decoratorului} primeşte ca \textbf{argument} un \textbf{obiect al clasei} pe care o decorează
		\vspace{4mm}
		\item În decorator se suprascriu toate metodele \textbf{clasei decorate} al căror comportament trebuie modificat
		\vspace{4mm} permite \textbf{instanţierea întârziată}
	\end{itemize}
\end{frame}

\begin{frame}{decorator.h}
	\footnotesize{\input{code/3decoratorh}}
\end{frame}

\begin{frame}{decorator.cc}
	\footnotesize{\input{code/3decorator}}
\end{frame}

\begin{frame}{main.cc}
	\footnotesize{\input{code/3main}}
\end{frame}

\begin{frame}{Cuvinte cheie}
	\begin{itemize}
		\item design software
		\item design pattern
		\item reutilizabilitate
		\item template
		\item MVC
		\item creational, structural, behavioral patterns
		\item delegare, agregare, consultare
		\item Singleton, Visitor, Decorator
		\item Factory, Observer, Command
	\end{itemize}
\end{frame}

\begin{frame}{Referinţe}
	\begin{itemize}
		\item \textbf{Design Patterns} - Gang Of Four: Gamma, Helm, Johnson, Vlissides
		\vspace{4mm}
		\item \textbf{Thinking in C++}, 2nd edition, volume 2 - Bruce Eckel
		\vspace{4mm}
		\item \textbf{The Design Patterns Java Companion} - James W. Cooper
		\vspace{4mm}
		\item \textbf{Code Complete}, 2nd edition - Steve McConnell
		\vspace{4mm}
		\item \textbf{Summary:} \url{http://en.wikipedia.org/wiki/Design_pattern_(computer_science)}
		\vspace{4mm}
		\item \textbf{1-page C++ and Java Examples:} \url{http://www.vincehuston.org/dp/}
	\end{itemize}
\end{frame}

\end{document}
