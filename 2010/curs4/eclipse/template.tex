% vim: set tw=78 aw:
\documentclass{beamer}

\usepackage[utf8x]{inputenc} % diacritice
\usepackage[romanian]{babel}
\usepackage{color}			 % highlight
\usepackage{alltt}			 % highlight
\usepackage{code/highlight}	 % highlight
\usepackage{hyperref}        % folositi \url{http://...}
% sau \href{http://...}{Nume Link}
\mode<presentation>
\usetheme{CDL}

% Titlul nu foloseşte Unicode pentru că e o problemă căreia nu i-am dat de
% cap.
\title[Eclipse IDE]{Eclipse IDE}
\subtitle{CDL - Cursul 4}
\institute[ROSEdu]{ROSEdu}
\author[Victor]{Victor Cărbune (\small victor@rosedu.org)}

\begin{document}

% Slide-urile cu mai multe părţi sunt marcate cu textul (cont.)
\setbeamertemplate{frametitle continuation}[from second]
% Arătăm numărul frame-ului
\setbeamertemplate{footline}[frame number]

\frame{\titlepage}

\begin{frame}
\tableofcontents
\end{frame}

% NB: Secţiunile nu sunt marcate vizual, ci doar apar în cuprins.
\section{Introducere}

\pgfdeclareimage[height=4cm, width=4cm]{eclipse}{eclipse}

\begin{frame}{Istorie}
\pgfputat{\pgfxy(7,1)}{\pgfbox[left,top]{\pgfuseimage{eclipse}}}

\begin{itemize}
	\item IBM
 		\begin{itemize}
		\pause	
		\item Noiembrie, 2001
		\end{itemize}
	\pause
	\vspace{4mm} 
	\item Fundația Eclipse
		\begin{itemize}	
		\pause	
		\item Ianuarie, 2004
		\pause
		\item Se ocupă de management-ul, infrastructura și licențierea proiectelor 
		\pause	
		\item Dezvoltarea proiectelor este asigurată de "commiters"
			\begin{itemize}	
			\pause	
			\item angajați ai unei terțe organizații
			\pause	
			\item voluntari
			\end{itemize}
		\end{itemize}
	\vspace{4mm} 
	\pause	
	\item Versiuni: Eclipse 3.0 (2004), Callisto(2006), Europa(2007), Ganymede(2008), Galileo(2009) - Helios(2010)
\end{itemize}
\end{frame}

\begin{frame}{De ce să folosim un IDE?}
  \begin{itemize}
  \pause
  \item Productivitate sporită, în special în cazul proiectelor mari
  \pause
  \item \textbf{I}ntegrated \textbf{D}evelopment \textbf{E}nvironment:
		\begin{itemize} 
		\pause
	  \item Editor
  	\pause
	  \item Compilator, interpretor
  	\pause
	  \item Debugger
		\pause
	  \item Manager de fișiere
  	\end{itemize}
  \end{itemize}
\end{frame}

\section{De ce Eclipse?}

\begin{frame}{Aspecte tehnice}
  \begin{itemize}
  \pause
  \item Cross platform (scris în Java)
  \pause
  \item Este o platformă în sine, iar tool-urile pentru development sunt adaugate ca extensii
  \pause
  \item Același IDE pentru o varietate de limbaje:
		\begin{itemize} 
		\pause
	  \item Java
  	\pause
	  \item C/C++	
  	\pause
	  \item Python
		\pause
	  \item Perl
	  \pause
	  \item PHP
	  \pause
	  \item ...
  	\end{itemize}
\end{itemize}    
\end{frame}

\begin{frame}{Ai fost nevoit vreodată...}
		\begin{itemize} 
		\pause
	  \item Să cauți prin surse o anumită metodă, după semnătura ei?
  	\pause
	  \item Recuperezi un fișier șters din greșeală?
  	\pause
	  \item Compari rapid două versiuni ale unei surse sau două fișiere ale proiectului? 
		\pause
	  \item Integrezi rapid proiectul cu un SCM dat?
  	\end{itemize}
\end{frame}

\section{Instalarea CDT}

\begin{frame}{Lab}
  \begin{itemize}
  \item Instalați Eclipse
  \item Instalați CDT
		\begin{itemize} 
		\pause
	  \item Help ... Install New Software
  	\pause
	  \item Available Software ... Add Site
  	\item http://download.eclipse.org/tools/cdt/releases/galileo
  	\end{itemize}
  \item Creați un proiect C++ nou
  \end{itemize}
\end{frame}

\section{GIT/SVN și Eclipse}

\begin{frame}{SCM și Eclipse}
  \begin{itemize}
  \item Similar cu instalarea CDT, instalați GIT și SVN
	\pause
	\item http://download.eclipse.org/egit/updates/
	\pause
  \item Importați cu GIT proiectul ext3grep
  \end{itemize}
\end{frame}

\section{Type Hierarchy și Task-uri}

\begin{frame}{Type Hierarchy și Task-uri }
  \begin{itemize}
  \item Permite vizualizarea proiectului orientat obiect (ierarhii de clase)
	\pause
  \item Quick Type Hierarchy / Open Type Hierarchy
  \pause
  \item Tab-ul Tasks este folosit pentru intepretarea diverselor comentarii din cod (//TODO)
  \end{itemize}
\end{frame}

\section{Local History}

\begin{frame}{Ștergerea unui fișier}
  \begin{itemize}
  \item Eclipse are un sistem propriu de supraveghere a fișierelor
  \pause
  \item Ștergeți una din sursele proiectului nou creat
  \pause
  \item Click dreapta pe proiect și selectați Restore From Local History
  \end{itemize}    
\end{frame}

\begin{frame}{Compararea a două fișiere}
  \begin{itemize}
  \item Prin intermediul Local History, Eclipse reține și versiunile modificate 
  \pause
  \item Efectuați câteva modificări asupra unuia dintre fișierele sursă
  \pause
  \item Click dreapta pe fișier și selectați Compare
  \end{itemize}
\end{frame}

\section{Eclipse debugger}

\begin{frame}{Debug mode}
  \begin{itemize}
  \item Rulați proiectul în debug (F11) 
  \pause
  \item Variabilele sunt afișate în partea dreaptă sus
  \pause
  \item Inclusiv registrele 
  \pause
  \item Breakpoint-urile se setează în view-ul sursei
  \end{itemize}
\end{frame}

\begin{frame}{}
\end{frame}
\end{document}
