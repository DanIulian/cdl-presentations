\documentclass{beamer}

\usepackage[utf8x]{inputenc} % diacritice
\usepackage[romanian]{babel}
\usepackage{color}			 % highlight
\usepackage{alltt}			 % highlight
\usepackage{code/highlight}	 % highlight
\usepackage{hyperref}        % folositi \url{http://...}
% sau \href{http://...}{Nume Link}
\mode<presentation>
\usetheme{CDL}

% Titlul nu foloseşte Unicode pentru că e o problemă căreia nu i-am dat de
% cap.
\title[Diff si Patch]{Diff + Patch}
\subtitle{CDL - Cursul 3}
\institute{ROSEdu}
\author{Lucian Adrian Grijincu \\ \texttt{lucian.grijincu@rosedu.org}}

\begin{document}

% Slide-urile cu mai multe părţi sunt marcate cu textul (cont.)
\setbeamertemplate{frametitle continuation}[from second]
% Arătăm numărul frame-ului
\setbeamertemplate{footline}[frame number]

\frame{\titlepage}

\frame{\tableofcontents}

% NB: Secţiunile nu sunt marcate vizual, ci doar apar în cuprins.
\section{Diff}

% Pentru reamintirea periodică a cuprinsului şi unde ne aflăm:
\frame{\tableofcontents[currentsection]}

% Titlul unui frame se specifică fie în acolade, imediat după \begin{frame},
% fie folosind \frametitle
\begin{frame}{\textbf{diff} – Ce face?}
\begin{itemize} % Just like normal LaTeX
  \item calculează diferențele dintre două fișiere
  \item afișează diferențele într-un mod standard
    \begin{itemize}
    \item mai precis trei moduri standard
    \end{itemize}
  \item compară \texttt{linie cu linie} (util pentru comparat cod sursă)
\end{itemize}
\end{frame}


\begin{frame}{\textbf{diff} – Fișiere de exemplificare}
  \begin{columns}[t]
    \column{5cm}
    {\LARGE{f-original}}\\
    {\tt{\small \input{code/f_original}}}
    \column{5cm}
    {\LARGE{f-modificat}}\\
    {\tt \small \input{code/f_modificat}}
  \end{columns}
\end{frame}





\begin{frame}
  \frametitle{\textbf{diff} (fără parametri) – Modul normal}
  \begin{columns}

    \begin{column}{7cm}
      \tt \input{code/diff_no_params}
    \end{column}

    \begin{column}{10cm}
      \begin{itemize}
      \item start,stop\textbf{codOp}start,stop
        \begin{itemize}
        \item \textbf{\texttt{d}} - linii șterse
        \item \textbf{\texttt{c}} - linii modificate
        \item \textbf{\texttt{a}} - linii adăugate
        \end{itemize}
      \item ``\texttt{\textless}''  linii din fiș. original
      \item ``\texttt{\textgreater}'' linii din fiș. modificat
      \end{itemize}
    \end{column}

  \end{columns}
\end{frame}




\begin{frame}
  \frametitle{\textbf{diff -c} – Context copiat}
  \begin{columns}

    \begin{column}{6cm}
      \footnotesize \tt \input{code/diff_c}
    \end{column}

    \begin{column}{10cm}
      \begin{itemize}
      \item \texttt{\texttt{***}} – nume fișier original
      \item \texttt{\texttt{---}} – nume fișier modificat
      \item o serie de blocuri:
        \begin{itemize}
        \item \texttt{***} startOrig, stopOrig \texttt{***}
        \item \texttt{---} startModif, stopModif \texttt{---}
        \item „\texttt{ }” – linii \textit{nemodificate} (context)
        \item „\texttt{!}” – linii \textit{modificate}
        \item „\texttt{-}” – linii \textit{șterse}
        \item „\texttt{+}” – linii \textit{adăugate}
        \end{itemize}
      \end{itemize}
    \end{column}

  \end{columns}
\end{frame}


\begin{frame}
  \frametitle{\textbf{diff -u} – Context unificat}
  \begin{columns}

    \begin{column}{6cm}
      %      \footnotesize \tt \input{code/diff_u}
      \tt{\input{code/diff_u}}
    \end{column}

    \begin{column}{10cm}
      \begin{itemize}
      \item \texttt{\texttt{---}} – nume fișier original
      \item \texttt{\texttt{+++}} – nume fișier modificat
      \item o serie de blocuri:
        \begin{itemize}
          \item \footnotesize \texttt{-}start,stopOrig \texttt{+} start,stopModif
        \item „\texttt{ }” – linii \textit{nemodificate} (context)
        \item „\texttt{-}” – linii dispărute din original
        \item „\texttt{+}” – linii apărute în modificat
        \end{itemize}
      \end{itemize}
    \end{column}

  \end{columns}
\end{frame}



\begin{frame}
  \frametitle{\textbf Formate afișare \textit{diff}}
  \begin{itemize}
  \item \textbf{normal}
    \begin{itemize}
    \item varianta standard (suport pe o arie largă de mașini)
    \item nu are nici un fel de context
    \item greu de urmărit ce s-a modificat
    \end{itemize}

  \item \textbf{context copiat}
    \begin{itemize}
    \item din cauza contextului copiat modificările sunt distanțate
    \item greu de urmărit ce s-a modificat
    \end{itemize}

  \item \textbf{context unificat}
    \begin{itemize}
    \item \alert{cea mai folosită variantă în lumea open-source}
    \item modificările sunt apropiate și înconjurate de context
    \end{itemize}
  \end{itemize}
\end{frame}



\section{Patch}
\frame{\tableofcontents[currentsection]}


\begin{frame}
  \frametitle{\textit{diff \& patch} – model de lucru}
  \begin{itemize}[<+->]
  \item modific niște fișiere
    \begin{itemize}
      \item \textit{vim, emacs, etc.}
    \end{itemize}
  \item sumarizez modificările
    \begin{itemize}
    \item \scriptsize{\texttt{diff f\_original.txt f\_modificat.txt \textgreater diferențe.patch}}
    \end{itemize}
  \item trimit fișierul cu modificările la responsabilul de proiect
    \begin{itemize}
    \item email, http, etc.
    \end{itemize}
  \item responsabilul analizează .patch-ul și aplică modificările
    \begin{itemize}
    \item \scriptsize{\texttt{patch f\_original.txt -i diferențe.patch -o f\_actualizat.txt}}
    \end{itemize}

  \end{itemize}
\end{frame}


\begin{frame}
  \frametitle{\textit{patch}}
  \begin{itemize}[<+->]

  \item \textit{patch} poate deduce numele fișierelor de actualizat din fișierul cu diferențe
    \begin{itemize}
    \item folosește informațiile din antet \\
      \texttt{---} f\_orginal.txt \texttt{---} \\
      \texttt{+++} f\_modificat.txt \texttt{+++}
    \end{itemize}

  \item \textbf{ATENȚIE: } dacă \textit{patch} nu e invocat din același loc ca și diff, numele fișierelor nu vor fi identice
    \begin{itemize}
    \item Soluție: instructăm \textit{patch} să ignore o parte a căii
    \end{itemize}
  \end{itemize}
\end{frame}

\begin{frame}
  \frametitle{\textit{patch -pNUM \textless diferențe.patch}}
  \begin{itemize}[<+->]
  \item avem următoarele fișiere:                        \\
    \texttt{/home/cdl/proiect/fișier.txt} - original     \\
    \texttt{/home/lucian/proiect/fișier.txt} - modificat \\
  \item fișierul .patch conține calea \\
    \texttt{/home/lucian/proiect/fișier.txt}
  \item invocăm \textit{patch} din \\
    \texttt{/home/cdl/}
  \item \texttt{patch -p0} ––– \texttt{/home/lucian/proiect/fișier.txt}
  \item \texttt{patch -p1} ––– \texttt{home/lucian/proiect/fișier.txt}
  \item \texttt{patch -p2} ––– \texttt{lucian/proiect/fișier.txt}
  \item \texttt{patch -p3} ––– \texttt{proiect/fișier.txt}
  \end{itemize}
\end{frame}


\begin{frame}{\textit{patch} invers}
  \begin{itemize}[<+->]
  \item Problemă: am aplicat un patch și vrem să revenim la starea inițială
  \item Soluție: \textit{reverse patching}
    \begin{itemize}
    \item se folosește același fișier de intrare
    \item \textit{patch} inversează rolurile lui \texttt{+++}, \texttt{---} și \texttt{***}
    \item \texttt{patch -pNUM \textbf{-R} \textless diferențe.patch}
    \end{itemize}
  \end{itemize}
\end{frame}

\section{Exercițiu}
\frame{\tableofcontents[currentsection]}
\begin{frame}
  \begin{itemize}
  \item textul exercițiului și un schelet de cod găsiți în repo-ul \texttt{git} al \texttt{CDL}-ului
  \item \texttt{git clone http://git.rosedu.org/cdl.git}
  \item \texttt{cd cdl/curs3/diff+patch}
  \end{itemize}
\end{frame}

\end{document}
