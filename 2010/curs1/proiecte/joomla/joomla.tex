% vim: set tw=78 aw:
\documentclass{beamer}

\usepackage[utf8x]{inputenc} % diacritice
\usepackage[romanian]{babel}
\usepackage{color}			 % highlight
\usepackage{alltt}			 % highlight
\usepackage{hyperref}        % folositi \url{http://...}
\mode<presentation>
\usetheme{CDL}

\title[Autosave / Collaborative Editor]{Autosave / Collaborative Editor}
\subtitle{proiect CDL}
\institute[Joomla!]{Joomla!}
\author[Victor C]{Victor Cărbune (victor@rosedu.org)}

\begin{document}

\setbeamertemplate{frametitle continuation}[from second]
\setbeamertemplate{footline}[frame number]

\frame{\titlepage}

\section{Joomla! CMS}

\pgfdeclareimage[height=0.8cm, width=4.08cm]{jlogo}{jlogo}

\begin{frame}{World of USO}
\pgfputat{\pgfxy(7,.7)}{\pgfbox[left,base]{\pgfuseimage{jlogo}}}
\begin{itemize}
	\item Versiunea stabilă: 1.5
		\begin{itemize}
		\pause
		\item Printre cel mai popular CMS existent		
		\pause
		\item Peste 15 milioane de download-uri
 		\pause
		\item Comunitate de peste 200.000 membri
		\end{itemize}
 	\pause
	\item Versiunea curentă în dezvoltare: 1.6 Alpha 2
		\begin{itemize}	
		\pause	
		\item Cu toate că este încă în dezvoltare (de ~ 1 an), versiunea 1.6 aduce extrem de multe feature-uri noi 
		\pause	
		\item Probabil vom lucra direct pe această variantă, sau în paralel, pentru a face extensia utilizabilă în viitorul apropiat.
		\end{itemize}
\end{itemize}
\end{frame}

\section{Autosave Extension}

\begin{frame}{Autosave Extension}
  \begin{itemize}
    \pause	
    \item O extensie separată, instalabilă.
    \pause	
    \item Salvează starea curentă a utilizatorului, indiferent de ce face acesta (editeaza, modifică setări, etc.)
    \pause
    \item Reprezintă și primul pas către un editor colaborativ
    \pause
    \item A fost cerută explicit la un moment dat pe listele de discuții
    \pause
  \end{itemize}
\end{frame}

\section{Collaborative Editor}

\begin{frame}{Collaborative Editor}
  \begin{itemize}
   \item Joomla! are la baza un sistem de locking, iar atunci cand cineva editeaza un articol acesta nu mai poate fi editat si de altcineva 
   \pause
   \item Un editor colaborativ ar îmbunătăți considerabil experiența utilizatorului 
   \pause 
   \item Cum s-ar putea implementa?
   \pause
   \begin{itemize}
   \item Diff Match Patch API
   \pause
   \item Google Wave
   \pause
   \item ...
   \end{itemize}
  \end{itemize}
\end{frame}

\begin{frame}{Mai multe informații}
	\begin{itemize}
	\item \url{http://community.joomla.org/}
	\item \url{http://joomlacode.org/gf/project/vscontrol/}
	\end{itemize}
\end{frame}

\begin{frame}{Mai multe informații}
	\begin{itemize}
	\item \url{http://community.joomla.org/}
	\item \url{http://joomlacode.org/gf/project/vscontrol/}
	\end{itemize}
\end{frame}

\end{document}
