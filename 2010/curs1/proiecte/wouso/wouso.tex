% vim: set tw=78 aw:
\documentclass{beamer}

\usepackage[utf8x]{inputenc} % diacritice
\usepackage[romanian]{babel}
\usepackage{color}			 % highlight
\usepackage{alltt}			 % highlight
\usepackage{hyperref}        % folositi \url{http://...}
\mode<presentation>
\usetheme{CDL}

\title[WoUSO]{World of USO}
\subtitle{proiecte CDL}
\institute[ROSEdu]{ROSEdu}
\author[alex3f]{Alex Eftimie (alex@rosedu.org)}

\begin{document}

\setbeamertemplate{frametitle continuation}[from second]
\setbeamertemplate{footline}[frame number]

\frame{\titlepage}

\section{WoUSO Status}

\pgfdeclareimage[height=1.2cm, width=4.3cm]{wouso}{wousologo}

\begin{frame}{World of USO}
\pgfputat{\pgfxy(7,.3)}{\pgfbox[left,base]{\pgfuseimage{wouso}}}
\begin{itemize}
	\item Versiunea 3
	\begin{itemize}
	\item începută în 2007, de la zero
	\item peste 10.000 linii de cod PHP și HTML
	\item greu de întreținut
	\item 11 dezvoltatori
	\end{itemize}
	\pause
	\item Versiunea curentă (v4?)
	\begin{itemize}
	\item ianuarie 2010, Django
	\item aprox. 1.800 linii de cod Python și HTML
	\item 90\% din funcționalitatea v3
	\item 2 dezvoltatori
	\item sună promițător
	\end{itemize}
\end{itemize}
\end{frame}

\section{WoUSO Grandchallenge}

\begin{frame}{WoUSO GrandChallenge}
  \begin{itemize}
    \item Modul separat, la fel ca The Quest
    \item La finalul competiției, primii 32, joacă în sistem piramidal
    \item Design, modele, implementare
  \end{itemize}
\end{frame}

\section{Facebook Integration}

\begin{frame}{Facebook Integration}
  \begin{itemize}
   \item De ce? 
   \pause
   \begin{itemize}
   \item Mafia Wars, senior league
   \end{itemize}
   \pause 
   \item Cum?
   \pause
   \begin{itemize}
   \item pyfacebook, Facebook API for applications
   \end{itemize}
  \end{itemize}
\end{frame}

\begin{frame}{Mai multe informații}
	\begin{itemize}
	\item \url{http://dev.rosedu.org/wouso/wiki/CDL2010}
	\end{itemize}
\end{frame}
\end{document}
