% vim: set tw=78 aw sw=2 sts=2 noet:
\documentclass{beamer}

%\includeonlyframes{c} -- speeding up compilation speed during debug

\usepackage[utf8x]{inputenc} % diacritice
\usepackage[romanian]{babel}
\usepackage{hyperref}        % folositi \url{http://...}
\mode<presentation>
\usetheme{CDL}

\title[]{Versionarea codului}
\subtitle{CDL 2012 - Cursul 2}
\institute[]{ROSEdu}
\author[]{
  Mihai Maruseac \texttt{mihai@rosedu.org} \\
  Răzvan Deaconescu \texttt{razvan@rosedu.org}
}

\setbeamertemplate{frametitle continuation}[from second]
\setbeamertemplate{footline}[frame number]

%\pgfdeclareimage[height=3cm]{m1}{img/01}
%\pgfdeclareimage[height=3cm]{m2}{img/02}
%\pgfdeclareimage[height=6cm]{m3}{img/03}
%\pgfdeclareimage[height=6.5cm]{m4}{img/04}
\pgfdeclareimage[height=5cm]{m5}{img/05}
\pgfdeclareimage[height=5cm]{m6}{img/snapshots}
\pgfdeclareimage[height=5cm]{m7}{img/decentr}
\pgfdeclareimage[height=8cm]{m8}{img/08}

\begin{document}

\maketitle

\begin{frame}{Ce înseamnă versionarea codului?}
  \begin{itemize}
    \item menținerea informațiilor despre schimbări/modificări în cod
    \item crearea unei liste de modificări cu posibilitatea de revenire sau
      interogare
    \item un fel de istoric, cu format posibil arborescent (branch-uri,
      ramuri de dezvoltare)
  \end{itemize}
\end{frame}

\begin{frame}{De ce versionarea codului?}
  \begin{itemize}
    \item posibilitatea de revenire la o versiune anterioară
    \item dacă modifici ceva din greșeală poți restaura
    \item lucru colaborativ: distribuție (read-only) și unificare (read-write)
    \item nu e limitat doar la cod (fișiere de configurare sau alte fișiere text)
  \end{itemize}
\end{frame}

\begin{frame}{De ce Git?}
  \begin{itemize}
    \item e cool
    \item everyone's doing it
    \item distribuit (poți face commit-uri locale, poți edita commit-uri
      locale)
    \item merge și pentru dezvoltare/versionare locală și pentru lucru
      colaborativ
    \item facilități utile: stash, index
    \item lucru facil cu branch-uri
    \item documentație
    \item GitHub, Gitorious
    \item portabil
    \item dezvoltare continuă
    \item integrare cu tool-uri/aplicații third-party
  \end{itemize}
\end{frame}

\begin{frame}{Operații cu soluții de versionare}
  \begin{itemize}
    \item creare repository
    \item clonare repository
    \item creare de commit-uri
    \item actualizare/editare commit-uri (teh Git shit)
    \item sincronizare repository-uri (push/pull)
    \item tagging
    \item branching
    \item revenire la o versiune anterioară
  \end{itemize}
\end{frame}

\begin{frame}{Cuvinte cheie}
  \begin{itemize}
    \item repository
    \item clone
    \item commit
    \item tree
    \item tag
    \item HEAD
    \item remote
    \item push/pull
    \item branch
    \item merge
    \item patch
  \end{itemize}
\end{frame}

\begin{frame}{Flux uzual de lucru}
  \begin{itemize}
    \item local
      \begin{enumerate}
	\item creezi repository: \texttt{git init .}
	\item creezi fișiere/directoare: editoare, \texttt{mkdir}, \texttt{cp}
	\item adaugi \textbf{conținut} pentru a crea un commit: \texttt{git
	  add}
	\item creezi commit: \texttt{git commit -m "Mesaj"}
	\item mergi la pasul 2
      \end{enumerate}
    \item contribuitor
      \begin{enumerate}
	\item clonezi repository: \texttt{git clone URL}
	\item creezi fișiere/directoare: editoare, \texttt{mkdir}, \texttt{cp}
	\item adaugi \textbf{conținut} pentru a crea un commit: \texttt{git
	  add}
	\item creezi commit: \texttt{git commit -m "Mesaj"}
	\item dacă vrei să lucrezi în continuare local mergi la pasul 2
	\item actualizezi repository: \texttt{git pull --rebase}
	\item faci push la modificări: \texttt{git push origin master}
	\item mergi la pasul 2
      \end{enumerate}
  \end{itemize}
\end{frame}

\begin{frame}{Configurare inițială}
  \begin{itemize}
    \item \texttt{git config --global user.name "\textit{nume prenume}"}
    \item \texttt{git config --global user.email "\textit{nume@dom.com}"}
    \item \texttt{git config --global color.ui auto}
    \item \texttt{git config --global color.pager true}
    \item \texttt{git config --global core.editor \textit{editor}}
    \item \texttt{cat $\sim$/.gitconfig}
  \end{itemize}
\end{frame}

\begin{frame}{Ignorarea fișierelor}
  \begin{itemize}
    \item \texttt{.gitignore}
      \begin{itemize}
	\item \texttt{file.pdf}
	\item \texttt{/file.pdf}
	\item \texttt{/test/file.pdf}
	\item \texttt{*.pdf}
	\item \texttt{/*.pdf}
	\item \texttt{*.swp}, \texttt{*~}, \texttt{tags}, \texttt{TAGS},
	  \texttt{cscope.out}
      \end{itemize}
    \item \texttt{.git/info/exclude}
    \item \texttt{git update-index --assume-unchanged file.txt}
  \end{itemize}
\end{frame}

\begin{frame}{De ce (nu) fișiere binare?}
  \begin{itemize}
    \item TODO
  \end{itemize}
\end{frame}

\begin{frame}{Crearea unui commit}
  \begin{itemize}
    \item mici și dese
    \item \textit{Do one thing, do one thing well!}
    \item folosire \texttt{git add -i} sau \texttt{git commit --amend}
  \end{itemize}
\end{frame}

\begin{frame}{Mesajele de commit}
  \begin{itemize}
    \item descriptive
    \item propoziție, începe cu majusculă
    \item recomandare de limitare la 50 de caractere
    \item dacă e nevoie de mai mult mesaj nou
    \item
      \url{http://tbaggery.com/2008/04/19/a-note-about-git-commit-messages.html}
    \item \url{http://whatthecommit.com}
  \end{itemize}
\end{frame}

\begin{frame}{De ce branch-uri?}
  \begin{itemize}
    \item TODO
  \end{itemize}
\end{frame}

\begin{frame}{Curăță}
  \begin{itemize}
    \item TODO
  \end{itemize}
\end{frame}

\begin{frame}{Vizualizarea repository-ului}
  \begin{itemize}
    \item \texttt{git log}
    \item \texttt{git show}
    \item gitk, gitg, giggle, tig
  \end{itemize}
\end{frame}

\begin{frame}{Extra -- Rezolvarea bug-urilor}
  \begin{itemize}
    \item \texttt{git stash}
    \item \texttt{git bisect}
  \end{itemize}
\end{frame}

\begin{frame}{Extra -- Crearea și trimiterea patch-urilor}
  \begin{itemize}
    \item \texttt{git format-patch}
    \item \texttt{git send-email}
  \end{itemize}
\end{frame}

\begin{frame}{Extra -- Hook-uri}
  \begin{itemize}
    \item \texttt{.git/hooks} în repo
    \item \texttt{post-receive}
    \item publicarea de rezultate
    \item crearea de arhive
    \item trimitere de e-mail-uri
  \end{itemize}
\end{frame}

\begin{frame}[label=l]{Links}
  \begin{itemize}
    \item \href{http://en.wikipedia.org/wiki/Comparison_of_revision_control_software}{Comparație între SCM-uri}
    \item \href{http://talks.rosedu.org/prezentari/prezentarea03}{Prezentare
    Tech Talks Git (Mircea Bardac)}
    \item \href{http://nvie.com/posts/a-successful-git-branching-model/}{A
    branching model}
    \item \href{http://github.com}{GitHub}
    \item
    \href{http://www.eqqon.com/index.php/Collaborative_Github_Workflow}{Workflow
    GitHub}
    \item
    \href{http://techblog.rosedu.org/git-good-practices.html}{Git Tips and
    Good Practices}
    \href{http://gitready.com/}{git ready}
  \end{itemize}
\end{frame}

\end{document}
