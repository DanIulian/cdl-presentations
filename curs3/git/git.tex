% vim: set tw=78 aw:
\documentclass{beamer}

\usepackage[utf8x]{inputenc} % diacritice
\usepackage[romanian]{babel}
\usepackage{color}			 % highlight
\usepackage{alltt}			 % highlight
% \usepackage{code/highlight}	 % highlight
\usepackage{hyperref}        % folositi \url{http://...}
                             % sau \href{http://...}{Nume Link}
\mode<presentation>
{ \usetheme{Rochester} }		% TODO: settle this

% Titlul nu foloseşte Unicode pentru că e o problemă căreia nu i-am dat de
% cap.
\title[GIT]{GIT}
\subtitle{CDL - Cursul 3}
\institute{ROSEdu}
\author{Mircea Bardac \texttt{cs@mircea.bardac.net}}

\begin{document}

% Slide-urile cu mai multe părţi sunt marcate cu textul (cont.)
\setbeamertemplate{frametitle continuation}[from second]
% Arătăm numărul frame-ului
\setbeamertemplate{footline}[frame number]

\frame{\titlepage}

\frame{\tableofcontents}

% NB: Secţiunile nu sunt marcate vizual, ci doar apar în cuprins.
\section{Distribuit vs. centralizat}

% Pentru reamintirea periodică a cuprinsului şi unde ne aflăm:
\frame{\tableofcontents[currentsection]}

% Titlul unui frame se specifică fie în acolade, imediat după \begin{frame},
% fie folosind \frametitle
\begin{frame}{Distribuit vs. centralizat}
\begin{itemize} % Just like normal LaTeX
\item centralizat: SVN, CVS
\item distribuit: Git, Mercurial, Bazaar, darcs etc.
\end{itemize}
\end{frame}

% Titlul unui frame se specifică fie în acolade, imediat după \begin{frame},
% fie folosind \frametitle
\begin{frame}{Diferențe in utilizare}
\begin{itemize} % Just like normal LaTeX
\item mai multe repository-uri
\item adunarea codului se face in urma operațiilor de merge
\item locotenenti
\item exista un set nou de operații pentru trimiterea/descărcarea commit-urilor
\end{itemize}
\end{frame}



% Titlul unui frame se specifică fie în acolade, imediat după \begin{frame},
% fie folosind \frametitle
\begin{frame}{Avantaje model distribuit}
\begin{itemize} % Just like normal LaTeX
\item lucru deconectat (fără acces la repository-ul central)
\item operații mult mai rapide (totul e local)
\item libertate în dezvoltare (oricâte repository-uri)
\item permite lucrul privat (nu trebuie publicate modificările)
\item sistem distribuit $=>$ rezistență
\end{itemize}
\end{frame}

% Titlul unui frame se specifică fie în acolade, imediat după \begin{frame},
% fie folosind \frametitle
\begin{frame}{Dezavantaje model distribuit}
\begin{itemize} % Just like normal LaTeX
\item mai greu de inteles (that's why we are here!)
\item greu de schimbat istoria commit-urilor
\item management-ul accesului este mai dificil
\end{itemize}
\end{frame}

% Titlul unui frame se specifică fie în acolade, imediat după \begin{frame},
% fie folosind \frametitle
\begin{frame}{Git vs. SVN}
\begin{itemize} % Just like normal LaTeX
\item Git un singur director .git în repository
\item în Git se manipulează conținut, nu fișiere (operațiile de modificare a repository-ului se fac pe conținut)
\item Git păstrează toată istoria in repository (cu mici exceptii) - acces offline
\end{itemize}
\end{frame}

% Titlul unui frame se specifică fie în acolade, imediat după \begin{frame},
% fie folosind \frametitle
\begin{frame}{Concepte Git}
\begin{itemize} % Just like normal LaTeX
\item conținut (vs. fișiere) - adăugarea parțială a modificărilor
\pause\item repository
\begin{itemize}
	\item local 
	\item remote repository - se pot adauga la repo-ul local referințe către repository-ul remote
\end{itemize}
\pause \item branch (ramură) - lanț de commit-uri
\begin{itemize}
	\item local/remote
\end{itemize}
\pause \item commit - conținut modificat
\begin{itemize}
	\item base commit - primul commit
	\item HEAD - ultimul commit dintr-un branch
\end{itemize}
\pause \item tag - echivalent cu o etichetă
\\
\pause \item \textbf{working copy} vs. \textbf{branch}
\end{itemize}
\end{frame}

% Titlul unui frame se specifică fie în acolade, imediat după \begin{frame},
% fie folosind \frametitle
\begin{frame}{Getting started with Git}
Configurație globală:
\begin{itemize} % Just like normal LaTeX
\item \texttt{git config --global user.name "Your Namee"}
\item \texttt{git config --global user.email your@email.addr}
\\
\item \texttt{git config --global color.diff auto}
\item \texttt{git config --global color.status auto}
\item \texttt{git config --global color.branch auto}
\end{itemize}
Configurație locală (per repository):
\begin{itemize} % Just like normal LaTeX
\item în fișierul \texttt{repo-dir/.git/config}
\end{itemize}
\end{frame}

% Titlul unui frame se specifică fie în acolade, imediat după \begin{frame},
% fie folosind \frametitle
\begin{frame}{Getting started with a repository}
\begin{itemize} % Just like normal LaTeX
\item \texttt{git init}
\pause\item \texttt{git clone <repository-path> [<target-dir>]}
\begin{itemize}
	\item Git repo: \texttt{git://url}
	\item cale SSH: \texttt{ssh://username@computer:path}
	\item cale locală: absolută sau relativă
\end{itemize}
\pause\item \texttt{git svn clone <svn-repo-path> [<target-dir>]}
\begin{itemize}
	\item lucrul cu repository-ul extern SVN se face utilizând \texttt{git svn}
	\item acțiuni frecvente: \texttt{git svn dcommit}, \texttt{git svn rebase}
\end{itemize}
\end{itemize}
\end{frame}

% Titlul unui frame se specifică fie în acolade, imediat după \begin{frame},
% fie folosind \frametitle
\begin{frame}{Ce se găsește în repository (1)}
\begin{itemize} % Just like normal LaTeX
\item \texttt{repo overview: gitk --all}
\\
\pause\item ramuri (branches)
\begin{itemize}
	\item locale: \texttt{git branch}
	\item remote: \texttt{git branch -r}
	\item toate: \texttt{git branch -a}
	\item $*$ înaintea unui nume de ramură indică faptul ca ramura respectivă este reflectată în working copy
\end{itemize}
\end{itemize}
\end{frame}

% Titlul unui frame se specifică fie în acolade, imediat după \begin{frame},
% fie folosind \frametitle
\begin{frame}{Ce se găsește în repository (2)}
\begin{itemize} % Just like normal LaTeX
\item un repository remote cu numele \texttt{\textbf{origin}}
\item branch-uri remote cu numele \textbf{\texttt{origin/<nume-branch>}} (referințe la ramurile din repository-ul clonat) - se poate specifica în mod explicit la clonare doar o ramură
\\
\item informații despre repository-ul remote: \texttt{git remote show <repository-name>}
\item în cazul nostru: \texttt{git remote show origin}
\end{itemize}
\end{frame}

% Titlul unui frame se specifică fie în acolade, imediat după \begin{frame},
% fie folosind \frametitle
\begin{frame}{Ce se găsește în repository (3)}
\begin{itemize} % Just like normal LaTeX
\item multe multe commit-uri
\item \texttt{git log} - commit-urile din branch-ul curent
\item \texttt{git log <file-name>} - commit-urile care afectează doar fișier-ul \texttt{<file-name>}
\end{itemize}
\end{frame}

\begin{frame}{Starea repository-ului}
\begin{itemize} % Just like normal LaTeX
\item Exită două tipuri de fișiere în repository:
\begin{itemize} % Just like normal LaTeX
	\item tracked files - conținutul este urmărit de Git
	\item untracked files - fișierele nu sunt cunoscute
\end{itemize}
\pause\item Comanda \texttt{git status} afișează:
\begin{itemize} % Just like normal LaTeX
	\item modificările marcate pentru includere în viitorul commit
	\item modificările detectate în fișierele urmărite dar nemarcate pentru commit
	\item fișierele neurmărite
\end{itemize}
\item comenzile:
\begin{itemize} % Just like normal LaTeX
	\item \texttt{git diff} și
	\item \texttt{git diff --cached}
\end{itemize}
\end{itemize}
\end{frame}

% Titlul unui frame se specifică fie în acolade, imediat după \begin{frame},
% fie folosind \frametitle
\begin{frame}{Commit}
\begin{itemize} % Just like normal LaTeX
\item ID commit = SHA1
\item referire
\begin{itemize}
	\item HEAD = the last commit
	\item HEAD urmat de un sir de X caret-uri (shift+6) = al X-ulea commit anterior lui HEAD
	\item HEAD~1 = HEAD urmat de un caret
    \item HEAD~2 = HEAD urmat de două carent-uri
    \item HEAD~10 = 10 commit-uri inainte de HEAD
\end{itemize}
\pause\item adăugare de conținut la commit: \texttt{git add [-i] <path-to-file>}
\item salvarea commit-ului: \texttt{git commit -s -m mesaj}
\item undo: \texttt{git reset} ( cu $--hard$ si $--soft$)
\\
\item pentru a pune un tag: \texttt{git tag <tag-name>}
\end{itemize}
\end{frame}

% Titlul unui frame se specifică fie în acolade, imediat după \begin{frame},
% fie folosind \frametitle
\begin{frame}{Branch-uri (ramuri)}
\begin{itemize} % Just like normal LaTeX
\item listat branch-uri: \texttt{git checkout <branch-name>}
\item creat branch: \texttt{git branch <branch-name> [<start-point>] }
\item schimbat branch: \texttt{git checkout <branch-name>}
\begin{itemize}
	\item la schimbarea branch-ului, modificările din branch-ul curent trebuie să fie commit-uite.
	\item \texttt{git stash} poate fi utilizat pentru a realiza commit-uri temporare
\end{itemize}
\end{itemize}
\end{frame}


% Titlul unui frame se specifică fie în acolade, imediat după \begin{frame},
% fie folosind \frametitle
\begin{frame}{Colaborarea folosind Git}
\begin{itemize} % Just like normal LaTeX
\item repository-uri remote: setup, actualizare imagine locala repository, lucrul cu branch-uri din repository-uri remote
\item schimbul de patch-uri
\end{itemize}
\end{frame}

% Titlul unui frame se specifică fie în acolade, imediat după \begin{frame},
% fie folosind \frametitle
\begin{frame}{Setup remote repository}
\begin{itemize} % Just like normal LaTeX
\item \texttt{git remote add [-t <remote-branch>] <remote-name> <repository-path>}
\\
\pause\item afisare repository-uri remote: \texttt{git remote}
\item afisare informații repository remote: \texttt{git remote show <remote-name>}
\end{itemize}
\end{frame}

% Titlul unui frame se specifică fie în acolade, imediat după \begin{frame},
% fie folosind \frametitle
\begin{frame}{Cum aducem commit-urile din repository-ul remote în cel local}
\begin{itemize} % Just like normal LaTeX
\item \texttt{git fetch <remote-name>}
\begin{itemize} % Just like normal LaTeX
	\item aduce toate commit-urile asociate branch-urilor remote cunoscute pentru repository-ul remote dat prin \texttt{<remote-name>}
\end{itemize}
\pause \item \texttt{git pull [<remote-name>]}
\begin{itemize} % Just like normal LaTeX
	\item \texttt{pull = fetch + merge}
	\item fără niciun parametru, se execută \texttt{git pull origin}
\end{itemize}

\end{itemize}
\end{frame}

% Titlul unui frame se specifică fie în acolade, imediat după \begin{frame},
% fie folosind \frametitle
\begin{frame}{Lucrul cu commit-urile din repository-ul remote}
\begin{itemize} % Just like normal LaTeX
\item checkout direct al branch-ului remote (nu permite editări)
\begin{itemize} % Just like normal LaTeX
	\item \texttt{git checkout <remote-name>/<remote-branch>}
\end{itemize}
\pause\item \texttt{git pull [remote-name]}
\begin{itemize} % Just like normal LaTeX
	\item \texttt{pull = fetch + merge}
	\item fără niciun parametru, se execută \texttt{git pull origin}
\end{itemize}
\pause\item 'merge' cu un branch local:
\begin{itemize} % Just like normal LaTeX
	\item \texttt{git checkout <local-branch>}
	\item \texttt{git merge <remote-name>/<remote-branch>}
\end{itemize}
\pause \item \textbf{fast-forward merge}
\end{itemize}
\end{frame}

\end{document}
