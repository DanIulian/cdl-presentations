% vim: set tw=78 aw:
\documentclass{beamer}

\usepackage[utf8x]{inputenc} % diacritice
\usepackage[romanian]{babel}
\usepackage{color}			 % highlight
\usepackage{alltt}			 % highlight
\usepackage{code/highlight}	 % highlight
\usepackage{hyperref}        % folositi \url{http://...}
                             % sau \href{http://...}{Nume Link}
\mode<presentation>
{ \usetheme{Rochester} }		% TODO: settle this

% Titlul nu foloseşte Unicode pentru că e o problemă căreia nu i-am dat de
% cap.
\title[BugzillaLaunchpad]{Bug tracking, Project hosting} % todo: change
\subtitle{CDL - Cursul 3}
\institute{ROSEdu}
\author{Alex Eftimie \texttt{alex@eftimie.ro}}

\begin{document}

% Slide-urile cu mai multe părţi sunt marcate cu textul (cont.)
\setbeamertemplate{frametitle continuation}[from second]
% Arătăm numărul frame-ului
\setbeamertemplate{footline}[frame number]

\frame{\titlepage}

\frame{\tableofcontents}

% NB: Secţiunile nu sunt marcate vizual, ci doar apar în cuprins.
\section{Bug tracking}

% Pentru reamintirea periodică a cuprinsului şi unde ne aflăm:
\frame{\tableofcontents[currentsection]}

% Titlul unui frame se specifică fie în acolade, imediat după \begin{frame},
% fie folosind \frametitle
\begin{frame}{Ce este Bugzilla?}
\begin{itemize} % Just like normal LaTeX
\item Un sistem de gestiune a bug-urilor
\pause \item Ce e un bug?
\begin{itemize}
\pause \item tehnic, o problemă a unui program
\pause \item în terminologia Bugzilla poate fi și o îmbunătățire (enhancement)
\end{itemize}
\pause \item Bugzilla reține informațiile despre bug-uri și organizează dezvoltarea software
\pause \item Este și un utilitar de comunicare
\pause \item Sistemul folosit pentru administrarea bug-urilor de: Mozilla, Gnome, KDE, Apache, Open Office, Eclipse, NASA, Facebook ...
\end{itemize}
\end{frame}

\begin{frame}[allowframebreaks] % Spargem paginile mai mari automat
% Atenţie: allowframebreaks nu funcţionează cu overlay-uri
\frametitle{De ce am folosi un astfel de sistem?}
\begin{itemize}
\item Face totul pentru noi
\begin{itemize}
\item memoria este slabă; email greu de utilizat/distribuit
\end{itemize}
\item Reține modificările în timp
\begin{itemize}
\item comentariile arată discuțiile, problemele prin care au trecut dezvoltatorii până la repararea bug-ului
\end{itemize}
\item Organizează bug-urile pentru dezvoltatorii curenți și viitori
\begin{itemize}
\item cuvinte cheie, categorii, importanță, filtrări
\end{itemize}
\item Favorizează procesul de dezvoltare
\begin{itemize}
\item nu întrerupi lucrul doar pentru că a apărut o problemă; o amâni :)
\end{itemize}
\end{itemize}
\end{frame}

\begin{frame}
\frametitle{Ce intră în categoria bug?}
\begin{itemize}
\pause \item defecțiune a programului; cod prost
\pause \item sarcină pentru dezvoltatori
\pause \item cerință de la utilizatori
\pause \item listă personală TODO
\end{itemize}
\end{frame}

\begin{frame}
\frametitle{Raportarea unui bug}
\begin{itemize}
\item Cum gândim?
\begin{itemize}
\pause \item ce fel de bug este?
\pause \item cum a apărut problema?
\pause \item cum se poate reproduce?
\pause \item ce componentă afectează?
\pause \item cui i se atribuie acest bug? % e vorba de asign to
\end{itemize}
\pause \item Ce reține sistemul?
\begin{itemize}
\pause \item \textbf{ID} Ex: \#345529
\pause \item Component
\pause \item Assigned To
\pause \item Summary, Description
\pause \item Cc: persoane care ar putea fi interesate
\pause \item Priority, Severity
\pause \item Platform
\pause \item Keywords: pentru a facilita căutarea
\pause \item Depends on, Also affects
\end{itemize}
\end{itemize}
\end{frame}

\begin{frame}
\frametitle{Exercițiu}
\begin{itemize}
\pause \item Accesați \url{http://cdl.rosedu.org/cgi-bin/bugzilla3/}.
\pause \item Creați-vă cont.
\pause \item Gândiți-vă la o bucată de cod (funcție) care are o problemă.
\pause \item Implementați acea funcție și uploadați-o în directorul vostru din
repository-ul svn.
\pause \item Analizați funcția creată de persoana din stânga voastră, descoperiți
problema și transmiteți-i un bug report persoanei din dreapta.
\begin{itemize}
\item Folosiți proiectul asociat persoanei din stânga.
\end{itemize}
\pause \item Rezolvați în repository problema transmisă de persoana din stânga și marcați bug-ul ca fiind rezolvat.
\end{itemize}
\end{frame}

\begin{frame}
\frametitle{Ciclul de viață al unui bug}
\begin{itemize}
\item Un bug poate trece prin următoarele stări:
\begin{itemize}
\item NEW
\item ASSIGNED
\item RESOLVED
\item REOPENED
\item INVALID
\item WONTFIX
\item WORKSFORME
\item DUPLICATE
\item FIXED
\end{itemize}
\pause \item \emph{NB: în taxonomia Launchpad, avem:} New, Incomplete, Invalid, Confirmed, Triaged, In progress, Fix commited, Fix released
\end{itemize}
\end{frame}

\section{Launchpad}
\frame{\tableofcontents[currentsection]}

\begin{frame}
\frametitle{Ce este Launchpad?}
\begin{itemize}
\pause \item sistem de dezvoltare, promovare, publicare de proiecte
\pause \item aplicație proprietară \emph{(dar nu pentru mult timp)} dezvoltată de Canonical pentru Ubuntu
\pause \item nu numai Ubuntu, ci peste \emph{10917} proiecte, \emph{343397} bug-uri, \emph{27013} ramuri de cod, \emph{1144838} traduceri, \emph{64469} răspunsuri, \emph{15532} specificații
\begin{itemize}
\pause \item RD: Capacitatea unui om de a o lua pe ulei este invers
proporțională cu distanța dintre locul natal și Curtea de Argeș.
\end{itemize}
\pause \item sursă de \emph{karma} :-)
\end{itemize}
\end{frame} 

\begin{frame}
\frametitle{Caracteristici Launchpad}
\begin{itemize}
\pause \item Bug tracking
\begin{itemize}
\item sistem propriu, dar poate importa și din bugzilla, trac
\end{itemize}
\pause \item Găzduire cod 
\begin{itemize}
\item	\emph{Bazaar}, poate și git în curând
\end{itemize}
\pause \item Localizare
\begin{itemize}
\item traduceri, control al versiuni
\end{itemize}
\pause \item Specificații (blueprints)
\begin{itemize}
\item schițe, scheme, dependențe, mentori
\end{itemize}
\pause \item Suport
\begin{itemize}
\item sistem de întrebări și răspunsuri propuse și votate
\newline
\item folosit de multe aplicații și distribuții Linux % nu neapărat integral!
\end{itemize}
\end{itemize}
\end{frame}

\begin{frame}
\frametitle{La ce mai poate fi folosit Launchpad?}
\begin{itemize}
\item poate găzdui doar una dintre activitățile de mai sus (Ex: traducerile)
\item poate fi folosit pentru coordonarea activităților unei echipe
\begin{itemize}
\item Grupul pentru software liber \url{http://launchpad.net/~softwareliber}
\item Echipa Ubuntu România \url{http://launchpad.net/ubunturo}
% aceste echipe pun taskurile ca bug-uri și le rezolvă pe LP
\end{itemize}
\end{itemize}
\end{frame}

\begin{frame}
\frametitle{Cum funcționează Launchpad}
\begin{itemize}
\item două concepte de bază: persoană/echipă și proiect
\item un proiecte este \emph{menținut} de o persoană și \emph{condus} (driven) de o echipă
\end{itemize}
\end{frame}

\begin{frame}
\frametitle{Exercițiu}
\begin{itemize}
\item Enumerați 3 distribuții de Linux care folosesc Launchpad
\item Creați-vă cont pe \url{http://launchpad.net}
\item Configurați adresa la care locuiți
\item Găsiți echipa \texttt{ROSEdu Supporters} și aplicați pentru calitatea de membru
\item Găsiți proiectul \texttt{ROSEdu}. Adăugați un bug. Modificați un bug existent.
\item Sugerați o traducere pentru \texttt{GNOME Do} % use search
\end{itemize}
\end{frame}

\begin{frame}
\frametitle{Extra-features Launchpad}
\begin{itemize}
\item OpenID
\begin{itemize}
\item soluție modernă de gestiune a identității
\item identitate comună pe mai multe saituri
\item \texttt{http://launchpad.net/\~alexeftimie}
\end{itemize}
\item PPA - Personal Package Archive
\begin{itemize}
\item cel mai simplu mod de a crea \emph{.deb}-uri și de a le distribui
\item compilare automată pe mai multe platforme
\end{itemize}
\item Liste de discuții
\begin{itemize}
\item găzduiește liste de discuții mailman
\end{itemize}
\end{itemize}
\end{frame}

\section {Alternative și referințe}
\frame{\tableofcontents[currentsection]}

\begin{frame}
\frametitle{Alternative}
\begin{itemize}
\item Sisteme de bug tracking:
\begin{itemize}
\item Trac
\item Launchpad
\item MantisBT
\end{itemize}
\item Găzduire de proiecte:
\begin{itemize}
\item SourceForge \url{http://sourceforge.net}
\item Google Code \url{http://code.google.com}
\end{itemize}
\end{itemize}
\end{frame}

\begin{frame}
\frametitle{Referințe}
\begin{itemize}
\item \url{http://bugzilla.mozilla.org}
\item \url{http://launchpad.net/+tour}
% todo add some more
\end{itemize}
\end{frame}

\end{document}
