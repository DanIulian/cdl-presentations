\documentclass{beamer}

\usepackage[utf8x]{inputenc}		
\usepackage[romanian]{babel}
\usepackage{color}			
\usepackage{alltt}			
\mode<presentation>
{ \usetheme{Rochester} }	


\title[Programe libere]{Gestiunea facila a unui proiect. Trac}
\subtitle{CDL - Cursul 3}
\institute{ROSEdu}
\author{Marius Latu \\ {\footnotesize marius.latu@gmail.com}}

\begin{document}
\setbeamertemplate{frametitle continuation}[from second]
\setbeamertemplate{footline}[frame number]

\frame{\titlepage}
\frame{\tableofcontents}

\section{Introducere}

  \frame{\tableofcontents[currentsection]}
  
  \begin{frame}
  \frametitle{Mică istorie a Trac-ului}

  \begin{itemize}
  \pause \item Reimplementare a unui proiect mai vechi
  \pause \item CVSTrac \begin{itemize} \item Python (1991 - Guido van Rossum) \item SQLite integrată \end{itemize}
  \pause \item svntrac datorită abilității de interacționa cu Subversion
  \pause \item acum \begin{itemize} \item Subversion \item Git \item Bazaar \item Mercurial \item Darcs \end{itemize}
  \end{itemize}
  \end{frame}


  \begin{frame}
  \frametitle{Ce este de fapt Trac?}

  \begin{itemize}
  \pause \item Instrument web-based utilizat pentru 
	\begin{itemize}
	\item managementul proiectelor - PMS
	\item managementul codului sursă - SCM
	\item folosește: wiki,svn, tichete
	\end{itemize}
  \pause \item Folosit în momentul de față în aproximativ 450 proiecte importante
	\begin{itemize}
	\item Webkit 
	\item VirtualBox 
	\item OSI 
	\item Beryl 
	\item etc..(listă detaliată la bibliogafie)
	\end{itemize}
  \end{itemize}
  \end{frame}

\section{Utilizare}

  \frame{\tableofcontents[currentsection]}

  \begin{frame}
  \frametitle{De ce am folosi Trac?}

  \begin{itemize}
  \pause \item Licență - BSD
  \pause \item Support - comunitate de entuziaști
  \pause \item Adaptabilitate - Python 
	\begin{itemize}
	\item suport pentru integrare cu alte limbaje
	\item librării standard extensibile
	\item curbă de învățare
	\end{itemize}
  \pause \item instalare
  \pause \item ușurință de utilizare
  \pause \item ușurintă în administrare (sistem de tichete)
  \pause \item securitate
  \end{itemize}
  \end{frame}

  \begin{frame}
  \frametitle{Ce instrumente oferă Trac?}

  \begin{itemize}
  \pause \item multi-user acces
  \pause \item attachments
  \pause \item email integration
  \pause \item import/export capabilities 
  \pause \item custom fields
  \pause \item authentication
  \pause \item autorization
  \pause \item reporting
  \pause \item audit trails
  \end{itemize}
  \end{frame}

\section{Exerciții}

  \frame{\tableofcontents[currentsection]}

  \begin{frame}
  \frametitle{Wiki Basics}

  \begin{itemize}
  \pause \item Logațivă în mediul CdlTracTest folosind userul și parola de svn
  \pause \item Creați pagina propie de documentare pentru exerciții
	\begin{itemize}
	\item http://dev.rosedu.org/trac/rosedu/wiki/CdlTracTest
	\item Pagina va conține contine observațiile de debug
	\item se va folosi formatare Wiki (headings, intendare, legături)
	\item http://dev.rosedu.org/trac/rosedu/wiki/WikiFormatting
	\end{itemize}
  \pause \item Pagina va conține observațiile asupra celor 3 surse
  \end{itemize}
  \end{frame}

  \begin{frame}
  \frametitle{Tichet Basics 1}

  \begin{itemize}
  \pause \item Verificați poșta și acceptați tichetul pe care l-ați primit
(închideți tichetul la final).
  \pause \item Creați un tichet nou pe numele colegului din dreapta prin care
îi dați sarcina să verifice corectitudinea celor 3 surse create anterior
(sursă.nume - numele persoane care crează tichetul) și să documenteze în
pagina personală erorile constate.
  \pause \item Asignați tichetul primit colegului din dreapta.
  \pause \item Rezolvați ultimul tichet primit (instrucțiuni în slide-ul următor).
  \end{itemize}
  \end{frame}

  \begin{frame}
  \frametitle{Tichet Basics 2}

  \begin{itemize}
  \pause \item Se va copia pe rând surseleîntr-un fișier merge.NumePrenume
  \pause \item După fiecare copiere se vor comite modificările cu un mesaj specific
  \pause \item Se vor rezolva în ordinea copierii bug-urile din fișierul merge și
se vor comite cu un mesaj specific
  \pause \item Se vor nota observațiile în pagina de wiki și se va observa istoria
fiecărei surse din browser.
  \pause \item PS. Pentru editarea și comiterea fișierelor se va folosi doar editorul vim,
fără folosirea mouse-ului.
  \item Have fun!
  \end{itemize}
  \end{frame}

\section{Bibliografie}

  \frame{\tableofcontents[currentsection]}

  \begin{frame}
  \frametitle{Referințe}

  \begin{itemize}
 
  \item \href{http://en.wikipedia.org/wiki/Trac}{Trac - Wikipedia}
  \item \href{http://www.webfaction.com/demos/trac-svn}{Trac Subversion Demos}
  \item \href{http://www.virtualbox.org/}{VirtualBox}
  \item \href{http://webkit.org/}{Webkit}
  \item \href{http://trac.edgewall.org/wiki/TracUsers}{Who uses Trac?}
  \item \href{http://trac-hacks.org/}{Trac Hacks}
  \item \href{http://trac.edgewall.org/wiki/PluginList}{Plugings List}

  \end{itemize}
  \end{frame}


\end{document}
