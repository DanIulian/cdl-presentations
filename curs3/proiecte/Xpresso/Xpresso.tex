% vim: set tw=78 aw sts=2 noet sw=2:
\documentclass{beamer}

\usepackage[utf8x]{inputenc} % diacritice
\usepackage[romanian]{babel}
\mode<presentation>
{ \usetheme{Rochester} }		% TODO: settle this

% Titlul nu foloseşte Unicode pentru că e o problemă căreia nu i-am dat de
% cap.
\title{Xpresso}
\subtitle{Propunere proiect CDL}
\institute{ROSEdu}
\author{Mihai Maruseac \\ \texttt{mihai.maruseac@rosedu.org}}

\begin{document}

% Slide-urile cu mai multe părţi sunt marcate cu textul (cont.)
\setbeamertemplate{frametitle continuation}[from second]
% Arătăm numărul frame-ului
\setbeamertemplate{footline}[frame number]

\frame{\titlepage}

\begin{frame}{Întâmplări în anul 1}
\begin{center}\textit{If you do something twice make a script}\end{center}
\begin{itemize}
\item \pause PL (proiectare logică), temele 2 și 3
\item \pause Majoritatea problemelor erau de minimizare a unor expresii logice astfel încât ele să se încadreze pe anumite tehnologii.
\item \pause Problema nu era în minimizare, chiar dacă era repetitivă. \pause După minimizare trebuia să desenăm circuitul final.
\end{itemize}
\end{frame}

\begin{frame}{Ideea}
\begin{itemize}
\item Ne trebuia un program care primea la intrare orice fel de funcții logice și anumite restricții
\item \pause Și ne oferea la ieșire funcțiile minimizate și un layout al circuitului logic final.
\item \pause Știam de existența espresso (produs la Berkeley), doar că el funcționa în linie de comandă.
\item \pause Nu știu să existe vreun program care desenează un circuit logic pornind de la exprimarea lor algebrică \pause(e drept, problema e NP completă dar nu vrem ceva optimizat la extrem, pentru început ne mulțumim cu o soluție oarecare)
\item \pause Cu atât mai puțin un program care să îmbine ambele probleme.
\end{itemize}
\end{frame}

\begin{frame}{Puțină istorie}
\begin{itemize}
\item Cu o săptămână înainte de sesiunea de comunicări eu și 2 colegi ne-am apucat să lucrăm la proiect în C (standard).
\item \pause În ziua sesiunii de comunicări aveam doar parsarea expresiilor de la intrare. (Pentru că ne-am complicat prea mult acolo).
\item \pause Proiectul a picat.
\item \pause Până după stagiul de practică (hfall, pidgin, etc)
\item \pause În vacanță am reînceput lucrul la el. (altă echipă)
\item \pause Am decis să pornim cu interfața grafică întâi: wxWidgets + C++.
\item \pause O problemă cu alocarea și cu clasele din C++, adunată la faptul că am reînceput hfall-ul, a determinat proiectul să stagneze din nou.
\end{itemize}
\end{frame}

\begin{frame}{Situația actuală}
\begin{itemize}
\item Avem două capete care nu se prea leagă între ele.
\item \pause Pentru că în ambele variante ne-am complicat extrem de mult.
\end{itemize}\vspace{5mm}
\pause \begin{center}{\huge \bf De ce e propus acest proiect?}\end{center}
\begin{itemize}
\item \pause Pentru că ar ajuta mulți studenți (PL, CN, etc)
\item \pause Pentru că are guidelines foarte clare (după două eșecuri)
\item \pause Pentru că are atât o parte de idee (desenarea circuitului) cât și una muncitorească (parsarea funcțiilor, scrierea interfeței, etc)
\end{itemize}
\end{frame}

\begin{frame}{Guidelines}
\begin{itemize}
\item \textbf{Doar sugestii}
\item Funcția importantă: desenarea circuitelor logice
\item \pause Funcție secundară: minimizarea expresiilor din input
\item \pause Lines of Actions:
\begin{enumerate}
\item \pause Prima funcție nu este implementată (din ceea ce știu eu), aici va fi puțin de lucru
\item \pause Pentru partea a doua avem două variante, ambele la fel de acceptabile (din punctul meu de vedere)
\begin{enumerate}
\item \pause Construim doar un fork peste espresso existent (comunicăm cu el prin fișiere și apeluri simple ale funcțiilor din shell)
\item \pause Realizăm noi și minimizarea funcțiilor primite ca argument (diverși algoritmi euristici)
\end{enumerate}
\end{enumerate}
\end{itemize}
\end{frame}

\begin{frame}{Implementare}
\begin{itemize}
\item \textbf{doar sugestii}
\item \pause Python
\item \pause E foarte simplu și ne-am pierde prea mult capul în C
\item \pause Anyway, C is cooler, deci merge și C dacă asta vrea majoritatea (sau orice alt limbaj, desigur)
\item \pause GTK sau wxWidgets (clonă pe Python sau alte variante)
\end{itemize}
\end{frame}

\begin{frame}{Ce vom realiza?}
\begin{itemize}
\item Din câte știu vom face ceva ce nu mai există
\item \pause Puțin mai greu dar realizabil
\item \pause Va fi alt proiect al facultății promovat la cursurile facultății
\item \pause Dar va fi numele vostru pe el.
\end{itemize}
\end{frame}

\begin{frame}
  \begin{center}
    { \Huge \bf Întrebări? }
  \end{center}
\end{frame}
\end{document}
