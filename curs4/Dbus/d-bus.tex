% vim: set tw=78 aw:
\documentclass{beamer}

\usepackage[utf8x]{inputenc} % diacritice
\usepackage[romanian]{babel}
\usepackage{amsmath}
\usepackage{amssymb}
\usepackage{color}			 % highlight
\usepackage{alltt}			 % highlight
%\usepackage{code/highlight}	 % highlight
\usepackage{hyperref}        % folositi \url{http://...}
                             % sau \href{http://...}{Nume Link}
\mode<presentation>
{ \usetheme{Rochester} }		% TODO: settle this

% Titlul nu foloseşte Unicode pentru că e o problemă căreia nu i-am dat de
% cap.
\title[D-Bus]{D-Bus}
\subtitle{CDL - Cursul 4}
\institute{ROSEdu}
\author{Sergiu Iordache \\\texttt{sergiu.iordache@gmail.com}}

\begin{document}

% Slide-urile cu mai multe părţi sunt marcate cu textul (cont.)
\setbeamertemplate{frametitle continuation}[from second]
% Arătăm numărul frame-ului
\setbeamertemplate{footline}[frame number]

\frame{\titlepage}

\frame{\tableofcontents}

\section{D-Bus}
\frame{\tableofcontents[currentsection]}

\begin{frame}{D-Bus}
\begin{itemize}
\item Mecanism simplu de comunicare inter-procese(IPC)
\item cross-platform
\item Dezvoltat de comunitatea Red Hat prin freedesktop.org
\item Influențat de DCOP(KDE) pe care l-a și înlocuit în versiunea 4 a KDE 
\item \pause D-Bus are mai multe bindinguri disponibile:
	\begin{itemize}
		\item C(GLib)
		\item C++(dbus-c++)
		\item Python(dbus-pyhton)
		\item Java
		\item Qt4
		\item PHP
		\item etc.
	\end{itemize}
\end{itemize}
\end{frame}

\begin{frame}{Arhitectura}
\begin{itemize}
\item 2 componente 
	\begin{itemize}
		\item o biblioteca pentru comunicare punct la punct 
		\item un daemon care se ocupă de administrarea magistralei
	\end{itemize}
\item \pause Procesele folosesc functiile din bibiloteca pentru a se conecta la daemon
\item \pause Oricâte procese se pot conecta la o magistrală în același timp
\item \pause Pot exista mai multe magistrale în sistem în același timp
\end{itemize}
\end{frame}

\begin{frame}{Adrese}
  \begin{tabular}{ | l | l | l | }
    \hline
    Obiect & Identificat prin & Exemplu \\ \hline
    magistrala & adresa & unix:path=/var/run/dbus/\\
    & & system\_bus\_socket \\ \hline
    conexiune & nume magistrala & com.mycompany.TextEditor \\ \hline
    obiect & cale & /com/mycompany/TextFileManager \\ \hline
    interfață & nume interfață & org.freedesktop.Hal.Manager \\ \hline
    \hline
  \end{tabular}
\end{frame}

\begin{frame}{Exemplu GNOME}
\begin{itemize}
\item 2 tipuri de bus-uri:
	\begin{itemize}
		\item o magistrală de system care se ocupa cu comunicarea intre diferitele componente ale sistemului(de exemplu adăugarea unei noi componente hardware)
		\item \pause o magistrală pentru fiecare sesiune activă a unui utilizator
	\end{itemize}
\end{itemize}
\end{frame}

\begin{frame}{Exemplu notificări}
\begin{itemize}
	\item Deschideți python în consolă (interactiv)
	\item \pause {\bfseries import dbus}
	\item \pause {\bfseries bus = dbus.SessionBus()}
	\item \pause {\bfseries notify\_service = bus.get\_object('org.freedesktop.Notifications',\linebreak '/org/freedesktop/Notifications')}
	\item \pause {\bfseries notify\_interface = dbus.Interface(notify\_service, \linebreak'org.freedesktop.Notifications')}
	\item \pause {\bfseries notify\_interface.Notify('program name', 0, '', 'CDL', \linebreak'Testăm notificări în GNOME', [], {}, -1)}
	\item \pause \href{http://www.galago-project.org/specs/notification/0.9/x408.html}{Explicația parametriilor}
	\item \pause Nu am reușit să setez icon-ul :(
\end{itemize}
\end{frame}

\begin{frame}{Concluzii}
\begin{itemize}
\item Plusuri
	\begin{itemize}
		\item interfață standardizată
		\item multe binding-uri
	\end{itemize}
\item \pause Minusuri
	\begin{itemize}
		\item documentația destul de slabă
		\item puține exemple pe internet(probabil mai multe în surse)
	\end{itemize}
\end{itemize}
\end{frame}


\section{Linkuri utile}
\frame{\tableofcontents[currentsection]}
\begin{frame}{Linkuri utile}
\begin{itemize}
\item \href{http://www.freedesktop.org/wiki/Software/dbus}{Pagina proiect D-Bus}
\item \href{http://www.freedesktop.org/wiki/IntroductionToDBus}{Introducere în D-Bus}
\item \href{http://dbus.freedesktop.org/doc/dbus-tutorial.html}{D-Bus tutorial}
\end{itemize}
\end{frame}

\end{document}
