% vim: set tw=78 aw:
\documentclass{beamer}

\usepackage[utf8x]{inputenc} % diacritice
\usepackage[romanian]{babel}
\usepackage{color}       % highlight
\usepackage{alltt}       % highlight
\usepackage{code/highlight}   % highlight
\usepackage{hyperref}        % folositi \url{http://...}
% sau \href{http://...}{Nume Link}
\mode<presentation>
\usetheme{CDL}

% Titlul nu foloseşte Unicode pentru că e o problemă căreia nu i-am dat de
% cap.
\title[Eclipse IDE]{Eclipse IDE}
\subtitle{CDL - Cursul 4}
\institute[ROSEdu]{ROSEdu}
\author[Adrian|Victor]{Victor Cărbune - Adrian Scoică\\(\small victor@rosedu.org - adrian.sc@rosedu.org)}

\begin{document}

% Slide-urile cu mai multe părţi sunt marcate cu textul (cont.)
\setbeamertemplate{frametitle continuation}[from second]
% Arătăm numărul frame-ului
\setbeamertemplate{footline}[frame number]

\frame{\titlepage}

\begin{frame}
\tableofcontents
\end{frame}

% NB: Secţiunile nu sunt marcate vizual, ci doar apar în cuprins.
\section{Introducere}

\pgfdeclareimage[height=4cm, width=4cm]{eclipselogo}{eclipselogo}

\begin{frame}{Istorie}
\pgfputat{\pgfxy(7,1)}{\pgfbox[left,top]{\pgfuseimage{eclipselogo}}}

\begin{itemize}
  \item IBM
     \begin{itemize}
    \pause  
    \item Noiembrie, 2001
    \end{itemize}
  \pause
  \vspace{4mm} 
  \item Fundația Eclipse
    \begin{itemize}  
    \pause  
    \item Ianuarie, 2004
    \pause
    \item Se ocupă de management-ul, infrastructura și licențierea proiectelor 
    \pause  
    \item Dezvoltarea proiectelor este asigurată de "commiters"
      \begin{itemize}  
      \pause  
      \item angajați ai unor terțe organizații din industrie
      \pause  
      \item voluntari
      \end{itemize}
    \end{itemize}
  \vspace{4mm} 
  \pause  
  \item Release-uri: Callisto, Europa, Ganymede (2006 - 2008), Galileo(2009), Helios(2010), Indigo(2011)
\end{itemize}
\end{frame}

\begin{frame}{De ce să folosim un IDE?}
  \begin{itemize}
  \pause
  \item Productivitate sporită, în special în cazul proiectelor mari
  \pause
  \item \textbf{I}ntegrated \textbf{D}evelopment \textbf{E}nvironment:
    \begin{itemize} 
    \pause
    \item Editor
    \pause
    \item Compilator, interpretor
    \pause
    \item Debugger
    \pause
    \item Manager de fișiere
    \end{itemize}
  \end{itemize}
\end{frame}

\section{De ce Eclipse?}

\begin{frame}{Aspecte pozitive}
  \begin{itemize}
  \pause
  \item Cross platform (scris în Java)
  \pause
  \item Un proiect reușit, de anvergură, foarte bine dezvoltat
  \pause
  \item Este o platformă în sine, iar tool-urile pentru development sunt adaugate ca extensii
  \pause
  \item Același GUI pentru o varietate de limbaje (folosirea mai multor perspective):
    \begin{itemize}
      \pause
      \item Object Oriented Programming - Java, C/C++, Python, Perl, PHP
      \pause
      \item Functional Programming -  Scheme, Haskell
      \pause
      \item Modeling - UML și altele
      \pause
      \item XML, Latex, etc.
      \pause
    \end{itemize}
  \item Multe IDE-uri comerciale se bazează pe Eclipse (dezvoltate de Adobe, Freescale, IBM și alții)
\end{itemize}
\end{frame}

\begin{frame}{Aspecte negative}
  \begin{itemize}
  \pause
  \item RAM
  \pause
  \item RAM
  \pause
  \item RAM
  \pause
  \item Interfața grafică nu este foarte user friendly 
  \pause 
  \item Doar pentru proiecte mari
  \pause
\end{itemize}
\end{frame}

\begin{frame}{Funcționalități de bază}
    \begin{itemize}
    \pause
    \item Autocomplete (variabile, nume de funcții, etc.)
    \pause
    \item Refactoring
    \pause
    \item Căutarea prin surse a unei metode după semnătura ei
    \pause
    \item Recuperarea unui fișier șters din greșeală
    \pause
    \item Compararea rapidă a două versiuni ale unor fișiere ale proiectului 
    \pause
    \item Integrarea rapidă a proiectul cu SCM
    \end{itemize}
\end{frame}

\section{Instalarea CDT}

\begin{frame}{Lab}
  \begin{itemize}
  \item Instalați Eclipse
  \pause
  \item Instalați CDT
    \begin{itemize} 
    \pause
    \item Help ... Install New Software
    \pause
    \item Available Software ... Add Site
    \item http://download.eclipse.org/tools/cdt/releases/galileo
    \end{itemize}
  \item Creați un proiect C++ nou (HelloWorld)
  \end{itemize}
\end{frame}

\section{GIT/SVN}

\begin{frame}{SCM și Eclipse}
  \begin{itemize}
  \item Similar cu instalarea CDT, instalați GIT și SVN
  \pause
    \begin{itemize}
    \item GIT http://download.eclipse.org/egit/updates/
    \pause
    \item SVN http://subclipse.tigris.org/update\_1.6.x
    \pause
    \end{itemize}
  \item Importați un proiect CDL utilizând GIT
    \begin{itemize}
    \item http://ext3grep.googlecode.com/svn/trunk/ (svn backup plan)
    \end{itemize}
  \end{itemize}
\end{frame}

\section{Type Hierarchy și Task-uri}

\begin{frame}{Type Hierarchy și Task-uri }
  \begin{itemize}
  \item Permite vizualizarea proiectului orientat obiect (ierarhii de clase)
  \pause
  \item Click dreapta pe o metodă și selectați Quick Type Hierarchy / Open Type Hierarchy
  \pause
  \item Tab-ul Tasks este folosit pentru intepretarea diverselor comentarii din cod (spre exemplu, //TODO, //FIXME)
  \end{itemize}
\end{frame}

\section{Local History}

\begin{frame}{Compararea a două fișiere}
  \begin{itemize}
  \item Prin intermediul Local History, Eclipse reține și versiunile modificate 
  \pause
  \item Efectuați câteva modificări asupra unuia dintre fișierele sursă
  \pause
  \item Click dreapta pe fișier și selectați Compare from Local History
  \end{itemize}
\end{frame}

\begin{frame}{Ștergerea unui fișier}
  \begin{itemize}
  \item Eclipse are un sistem propriu de supraveghere a fișierelor
  \pause
  \item Ștergeți sursa proiectului nou creat
  \pause
  \item Click dreapta pe proiect și selectați Restore From Local History
  \end{itemize}
\end{frame}

\section{Eclipse debugger}

\begin{frame}{Debug mode}
  \begin{itemize}
  \item Rulați proiectul helloworld în debug mode (F11) 
  \pause
  \item Variabilele sunt afișate în partea dreaptă sus
  \pause
  \item Inclusiv registrele 
  \pause
  \item Setați un breakpoint, în partea stângă a sursei 
  \end{itemize}
\end{frame}

\section{Resurse}

\begin{frame}{Resurse}
  \begin{itemize}
  \item http://www.eclipse.org
  \pause
  \item http://www.eclipseplugincentral.org  
  \pause
  \vspace{10mm}
  \item Întrebări?
  \end{itemize}
\end{frame}

\begin{frame}{}
\end{frame}

\end{document}
