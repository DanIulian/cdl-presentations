% vim: set tw=78 aw:
\documentclass{beamer}

\usepackage[utf8x]{inputenc} % diacritice
\usepackage[romanian]{babel}
\usepackage{amsmath}
\usepackage{amssymb}
\usepackage{color}			 % highlight
\usepackage{alltt}			 % highlight
\usepackage{code/highlight}	 % highlight
\usepackage{hyperref}        % folositi \url{http://...}
                             % sau \href{http://...}{Nume Link}
\mode<presentation>
{ \usetheme{Rochester} }		% TODO: settle this

% Titlul nu foloseşte Unicode pentru că e o problemă căreia nu i-am dat de
% cap.
\title[GTK]{GTK}
\subtitle{CDL - Cursul 4}
\institute{ROSEdu}
\author{Mihai Maruseac \\\texttt{mihai.maruseac@gmail.com}}

\begin{document}

% Slide-urile cu mai multe părţi sunt marcate cu textul (cont.)
\setbeamertemplate{frametitle continuation}[from second]
% Arătăm numărul frame-ului
\setbeamertemplate{footline}[frame number]

\frame{\titlepage}

\frame{\tableofcontents}

\section{GTK}
\frame{\tableofcontents[currentsection]}

\begin{frame}{GTK}
\begin{itemize}
\item GIMP Toolkit
\item GTK+
\item cross-platform
\item Una dintre cele mai populare interfețe grafice din X Window System, împreună cu Qt.
\item \pause Scris în C dar cu bindings pentru multe limbaje:
	\begin{itemize}
		\item C++
		\item Perl
		\item Ruby
		\item Java (?, fără Windows)
		\item C\#
		\item Python
	\end{itemize}
\end{itemize}
\end{frame}

\begin{frame}{PyGTK}
\begin{itemize}
	\item Binding $Python + GTK = PyGTK$
	\vspace{2cm}
	\item\pause\#pygtk
\end{itemize}
\end{frame}

\begin{frame}{Hello World, 1}
\begin{itemize}
	\item Importați pygtk și gtk
	\item \pause Folosiți\linebreak
		{\bfseries pygtk.require('2.0')}\linebreak
		pentru a cere explicit folosirea funcționalităților din GTK2.0
	\item \pause Construiți o clasă oarecare care va avea o singură metodă, {\bfseries \_\_init\_\_}
	\item \pause Construiți o fereastră prin instanțierea clasei Window\linebreak{\bfseries self.w = gtk.Window(gtk.WINDOW\_TOPLEVEL)}
	\item \pause Afișați fereastra ({\bfseries show})
	\item \pause Rulați
\end{itemize}
\end{frame}

\begin{frame}{Hello World, 1 - cod complet}
	\noindent
\ttfamily
\hlstd{}\hlline{\ \ \ \ 1\ }\hlslc{\#!/bin/bash}\\
\hlline{\ \ \ \ 2\ }\hlstd{\\
\hlline{\ \ \ \ 3\ }MAX}\hlsym{=}\hlstd{}\hlnum{10}\\
\hlline{\ \ \ \ 4\ }\hlstd{}\\
\hlline{\ \ \ \ 5\ }\hlkwa{for\ }\hlstd{i\ }\hlkwa{in\ }\hlstd{\$}\hlsym{(}\hlstd{}\hlkwc{seq\ }\hlstd{}\hlnum{1\ }\hlstd{}\hlkwd{\$\{MAX\}}\hlstd{}\hlsym{);\ }\hlstd{}\hlkwa{do}\\
\hlline{\ \ \ \ 6\ }\hlstd{}\hlstd{\ \ \ \ \ \ \ \ }\hlstd{}\hlkwb{echo\ }\hlstd{}\hlstr{"Testing\ \$\{i\}"}\hlstd{}\\
\hlline{\ \ \ \ 7\ }\hlkwa{done}\\
\hlline{\ \ \ \ 8\ }\hlstd{}\\
\hlline{\ \ \ \ 9\ }\hlkwb{exit\ }\hlstd{}\hlnum{0}\hlstd{}\\
\mbox{}
\normalfont

\end{frame}

\begin{frame}{Hello World, 2}
\begin{itemize}
	\item Unde e fereastra?
	\item \pause Construiți o funcție main în cadrul clasei voastre, funcție care doar apelează gtk.main() (orice aplicație GTK necesită această funcție)
\end{itemize}
\end{frame}

\begin{frame}{Hello World, 2 - cod complet}
	\noindent
\ttfamily
\hlstd{}\hlline{\ \ \ \ 1\ }\hldir{\#include\ $<$stdio.h$>$}\\
\hlline{\ \ \ \ 2\ }\hlstd{}\hldir{\#include\ $<$stdlib.h$>$}\\
\hlline{\ \ \ \ 3\ }\hlstd{}\\
\hlline{\ \ \ \ 4\ }\hldir{\#define\ WHATEVER\ 1}\\
\hlline{\ \ \ \ 5\ }\hlstd{}\\
\hlline{\ \ \ \ 6\ }\hlkwb{int}\\
\hlline{\ \ \ \ 7\ }\hlstd{}\hlkwd{main}\hlstd{}\hlsym{(}\hlstd{}\hlkwb{int\ }\hlstd{argc}\hlsym{,\ }\hlstd{}\hlkwb{char\ }\hlstd{}\hlsym{{*}{*}}\hlstd{argv}\hlsym{)}\\
\hlline{\ \ \ \ 8\ }\hlstd{}\hlsym{\{}\\
\hlline{\ \ \ \ 9\ }\hlstd{}\hlstd{\ \ \ \ \ \ \ \ }\hlstd{}\hlkwd{puts}\hlstd{}\hlsym{(}\hlstd{}\hlstr{"Hello,\ World!}\hlesc{$\backslash$n}\hlstr{"}\hlstd{}\hlsym{);}\\
\hlline{\ \ \ 10\ }\hlstd{\\
\hlline{\ \ \ 11\ }}\hlstd{\ \ \ \ \ \ \ \ }\hlstd{}\hlkwa{return\ }\hlstd{}\hlnum{0}\hlstd{}\hlsym{;}\\
\hlline{\ \ \ 12\ }\hlstd{}\hlsym{\}}\hlstd{}\\
\mbox{}
\normalfont

\end{frame}

\begin{frame}{Hello World, 2}
\begin{itemize}
	\item \textbf{Ce avem?}
	\item \pause O fereastră
	\item \pause care nu oprește aplicația la închiderea ei
	\item \pause \textbf{Ce ne trebuie?}
	\item \pause O metodă prin care oprim aplicația când trebuie
	\item \pause Afișarea textului !!one!!
\end{itemize}
\end{frame}

\begin{frame}{Callbacks}
\begin{itemize}
	\item GTK este condus prin evenimente
	\item \pause Controlul va sta în main până la apariția unui eveniment, caz în care se va executa o funcție potrivită
	\item \pause Semnale: când se produce un eveniment (click mouse buton), elementul care a primit evenimentul va produce un semnal, care va fi capturat de un handler ce va apela o funcție corespunzătoare.
	\item \pause \textbf{\textit{object.connect(signal\_name, func, data)}}
	\item \pause \textit{def func(widget, callback\_data)}
	\item \pause \textit{def func(widget, event, callback\_data)}
\end{itemize}
\end{frame}

\begin{frame}{Implementarea unei închideri funcționale}
\begin{itemize}
	\item Avem două evenimente de tratat:
		\begin{itemize}
			\item delete\_event - managerul de ferestre decide închiderea ferestrei (dacă se returnează False, fereastra se va închide)
			\item destroy - se capturează semnalul destroy
		\end{itemize}
	\item \pause Ieșirea se va face cu gtk.main\_quit().
\end{itemize}
\end{frame}

\begin{frame}{Hello World, 3}
\begin{itemize}
	\item Adăugați cele două evenimente pentru fereastra creată
	\item Construiți cele două funcții necesare
	\item Rulați programul.
\end{itemize}
\end{frame}

\begin{frame}{Hello World, 3 - cod complet}
	\noindent
\ttfamily
\footnotesize
\hlstd{}\hlline{\ \ \ \ 1\ }\hlslc{\#!/usr/bin/env\ python}\\
\hlline{\ \ \ \ 2\ }\hlstd{}\hlkwa{import\ }\hlstd{pygtk\\
\hlline{\ \ \ \ 3\ }pygtk}\hlsym{.}\hlstd{}\hlkwd{require}\hlstd{}\hlsym{(}\hlstd{}\hlstr{'2.0'}\hlstd{}\hlsym{)}\\
\hlline{\ \ \ \ 4\ }\hlstd{}\hlkwa{import\ }\hlstd{gtk}\\
\hlline{\ \ \ \ 5\ }\hlkwa{class\ }\hlstd{Hello}\hlsym{:}\\
\hlline{\ \ \ \ 6\ }\hlstd{}\hlstd{\ \ \ \ }\hlstd{}\hlkwa{def\ }\hlstd{}\hlkwd{del\textunderscore event\textunderscore f}\hlstd{}\hlsym{(}\hlstd{self}\hlsym{,\ }\hlstd{widget}\hlsym{,\ }\hlstd{event}\hlsym{,\ }\hlstd{data\ }\hlsym{=\ }\hlstd{}\hlkwa{None}\hlstd{}\hlsym{):}\\
\hlline{\ \ \ \ 7\ }\hlstd{}\hlstd{\ \ \ \ \ \ \ \ }\hlstd{}\hlkwa{return\ False}\\
\hlline{\ \ \ \ 8\ }\hlstd{}\hlstd{\ \ \ \ }\hlstd{}\hlkwa{def\ }\hlstd{}\hlkwd{destroy\textunderscore sgn\textunderscore f}\hlstd{}\hlsym{(}\hlstd{self}\hlsym{,\ }\hlstd{widget}\hlsym{,\ }\hlstd{data\ }\hlsym{=\ }\hlstd{}\hlkwa{None}\hlstd{}\hlsym{):}\\
\hlline{\ \ \ \ 9\ }\hlstd{}\hlstd{\ \ \ \ \ \ \ \ }\hlstd{gtk}\hlsym{.}\hlstd{}\hlkwd{main\textunderscore quit}\hlstd{}\hlsym{()}\\
\hlline{\ \ \ 10\ }\hlstd{}\hlstd{\ \ \ \ }\hlstd{}\hlkwa{def\ }\hlstd{}\hlkwd{\textunderscore \textunderscore init\textunderscore \textunderscore }\hlstd{}\hlsym{(}\hlstd{self}\hlsym{):}\\
\hlline{\ \ \ 11\ }\hlstd{}\hlstd{\ \ \ \ \ \ \ \ }\hlstd{self}\hlsym{.}\hlstd{w\ }\hlsym{=\ }\hlstd{gtk}\hlsym{.}\hlstd{}\hlkwd{Window}\hlstd{}\hlsym{(}\hlstd{gtk}\hlsym{.}\hlstd{WINDOW\textunderscore TOPLEVEL}\hlsym{)}\\
\hlline{\ \ \ 12\ }\hlstd{}\hlstd{\ \ \ \ \ \ \ \ }\hlstd{self}\hlsym{.}\hlstd{w}\hlsym{.}\hlstd{}\hlkwd{show}\hlstd{}\hlsym{()}\\
\hlline{\ \ \ 13\ }\hlstd{}\hlstd{\ \ \ \ \ \ \ \ }\hlstd{self}\hlsym{.}\hlstd{w}\hlsym{.}\hlstd{}\hlkwd{connect}\hlstd{}\hlsym{(}\hlstd{}\hlstr{"delete\textunderscore event"}\hlstd{}\hlsym{,\ }\hlstd{self}\hlsym{.}\hlstd{del\textunderscore event\textunderscore f}\hlsym{)}\\
\hlline{\ \ \ 14\ }\hlstd{}\hlstd{\ \ \ \ \ \ \ \ }\hlstd{self}\hlsym{.}\hlstd{w}\hlsym{.}\hlstd{}\hlkwd{connect}\hlstd{}\hlsym{(}\hlstd{}\hlstr{"destroy"}\hlstd{}\hlsym{,\ }\hlstd{self}\hlsym{.}\hlstd{destroy\textunderscore sgn\textunderscore f}\hlsym{)}\\
\hlline{\ \ \ 15\ }\hlstd{}\hlstd{\ \ \ \ }\hlstd{}\hlkwa{def\ }\hlstd{}\hlkwd{main}\hlstd{}\hlsym{(}\hlstd{self}\hlsym{):}\\
\hlline{\ \ \ 16\ }\hlstd{}\hlstd{\ \ \ \ \ \ \ \ }\hlstd{gtk}\hlsym{.}\hlstd{}\hlkwd{main}\hlstd{}\hlsym{()}\\
\hlline{\ \ \ 17\ }\hlstd{hello\ }\hlsym{=\ }\hlstd{}\hlkwd{Hello}\hlstd{}\hlsym{()}\\
\hlline{\ \ \ 18\ }\hlstd{hello}\hlsym{.}\hlstd{}\hlkwd{main}\hlstd{}\hlsym{()}\hlstd{}\\
\mbox{}
\normalfont
\normalsize

\end{frame}
\begin{frame}{Mesajul??}
\begin{itemize}
	\item Unde avem afișarea?
	\item Vom folosi un buton care să ne permită să afișăm text la consolă.
	\item Widget-ul button: gtk.Button(text)
	\item Eveniment: clicked
\end{itemize}
\end{frame}

\begin{frame}{Hello World, 4 \& 5}
\begin{itemize}
	\item Modificați programul anterior prin adăugarea unui buton astfel încât la afișarea lui să se afișeze în terminal mesajul "Salut, $<$nume$>$", unde $<$nume$>$ este numele vostru.
	\item \pause Adăugați un alt buton, astfel încât la apăsarea lui să se afișeze în consolă mesajul "GTK is nice". Nu definiți alte funcții de callback.
\end{itemize}
\end{frame}

\begin{frame}{Aranjarea în pagină}
\begin{itemize}
	\item Ce se întâmplă când punem mai multe controale?
	\item De unde știe cum să le amplaseze convenabil?
	\item \pause Soluția: packing
	\item \pause Se construiesc cutii invizibile în care se plasează componentele.
	\item \pause Două tipuri de cutii: orizontale (HBox) și verticale (VBox)
	\item \pause pack\_start; pack\_end: \pause direcția de împachetare
\end{itemize}
\end{frame}

\begin{frame}{Hello World, 6}
\begin{itemize}
	\item Adăugați aplicației încă 4 butoane și apoi testați diverse mdouri de amplasare
	\item Încercați atât cutii verticale cât și cutii orizontale, încercați și combinații de cutii și cutii în cutii
	\item Testați ambele variante de pack
\end{itemize}
\end{frame}

\begin{frame}{Hello World, 7}
\begin{itemize}
\item Încercați și o aplasare utilizând un Table
\item \pause table.attach(child, left, rigth, top, bottom)
\end{itemize}
\end{frame}

\begin{frame}{Hello World, 8}
\begin{itemize}
\item Folosiți un container fix (Fixed)
\item fixed.put(widget, x, y)
\item \pause fixed.move(widget, x, y)
\end{itemize}
\end{frame}

\section{Linkuri utile}
\frame{\tableofcontents[currentsection]}
\begin{frame}{Linkuri utile}
\begin{itemize}
\item \href{http://www.gtk.org/}{The GTK+ Project}
\item \href{http://www.gtk.org/documentation.html}{GTK+ Documentation}
\item \href{http://www.pygtk.org/index.html}{PyGTK}
\end{itemize}
\end{frame}

\end{document}
