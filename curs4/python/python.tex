% vim: set tw=78 aw:
\documentclass{beamer}

\usepackage[utf8x]{inputenc}      % diacritice
\usepackage[romanian]{babel}
\usepackage{color}                % highlight
\usepackage{alltt}                % highlight
\usepackage{code/highlight}       % highlight
\usepackage{hyperref}             % folositi \url{http://...}
                                  % sau \href{http://...}{Nume Link}
\mode<presentation>
{ \usetheme{Rochester} }          % TODO: settle this

% Titlul nu foloseşte Unicode pentru că e o problemă 
% căreia nu i-am dat de cap.
\title[Python]{Python}
\subtitle{CDL - Cursul 4}
\institute{ROSEdu}
\author{Lucian Adrian Grijincu \\ 
  \texttt{lucian.grijincu@rosedu.org}}

\begin{document}

% Slide-urile cu mai multe părţi sunt marcate cu textul (cont.)
\setbeamertemplate{frametitle continuation}[from second]
% Arătăm numărul frame-ului
\setbeamertemplate{footline}[frame number]

\frame{\titlepage}

\frame{\tableofcontents}

% NB: Secţiunile nu sunt marcate vizual, ci doar apar în cuprins.
\section{Inca un limbaj?}

% Pentru reamintirea periodică a cuprinsului şi unde ne aflăm:
\frame{\tableofcontents[currentsection]}


\begin{frame}{Încă unul?}
  \begin{itemize}
  \item ASM
  \item FORTRAN %(\textbf{încă} folosit în HPC)
  \item Pascal
  \item C
  \item C++
  \item Java \& C\#
  \item JavaScript
  \item ???
  \end{itemize}
\end{frame}


\begin{frame}{Dacă am scrie un limbaj nou am vrea ...}
  \begin{itemize}[<+->]
  \item să fie ușor de învățat
  \item compilator portabil (să rulezeze pe cât mai multe platforme)
  \item programele scrise în el să fie portabile
  \item scalabil (bun și pentru proiecte mici și mari)
  \item stabil, matur, simplu și elegant
  \item bibliotecă standard generoasă
  \item multe biblioteci terțe
  \item să fie orientat obiect (că era la modă acum câțiva ani)
  \item să fie funcțional (că e la modă acum)
  \item dezvoltare rapidă
  \item să fie emebedable (să poată fi inclus în alte programe
    pentru scripting – de ex. în jocuri)
  \item self-documenting (vedem mai târziu ce înseamnă)
  \end{itemize}
\end{frame}


\section{Interpretor interactiv}
\frame{\tableofcontents[currentsection]}


\begin{frame}{}
  \begin{itemize}
  \item permite evaluarea expresiilor fără compilare \\
    \texttt{
      >>> 1 + 1 \\ 
      1         \\
      >>> x = 1 \\ 
      >>> y = 2 \\ 
      >>> x+y   \\
      3
    }
  \item referințele nu au tip\\
    \texttt{
      >>> x = 1        \\
      >>> x            \\
      1                \\
      >>> x = 'asdf'   \\
      >>> x            \\
      'asdf'
    }
  \end{itemize}
\end{frame}

\begin{frame}{help}
  \begin{itemize}
  \item informații de ajutor pentru fiecare obiect
  \item executați \textbf{help(1)} – veți obține help-ul 
    clasei \textbf{int} asdf
  \item \textbf{\_\_add\_\_, \_\_and\_\_, \_\_div\_\_, \_\_sub\_\_} 
    sunt nume pentru \\
    \textbf{+, \&, /, -}
  \item executați \textbf{help(str)}
  \end{itemize}
\end{frame}



\section{Sintaxa}
\frame{\tableofcontents[currentsection]}

\begin{frame}{if - elif - else}
  \begin{itemize}[<+->]
  \item ce observați în codul următor? \\
    \input{code/if}
  \item lipsesc \{\} și ().
  \item blocurile sunt separate prin identare
  \item simplu, curat, arată a pseudocod
  \end{itemize}
\end{frame}

\begin{frame}{funcții}
  \begin{itemize}
  \item funcțiile sunt documentate prin \textit{docstrings} \\
   \small \input{code/fun}
  \item \textbf{help(factorial)}                            \\
    \texttt{
      Help on function f in module \_\_main\_\_\:           \\
      factorial(n)                                          \\
      Computes the factorial of n.                          \\
      Returns n!.
    }
  \end{itemize}
\end{frame}


\end{document}
