\documentclass{beamer}

\usepackage[utf8x]{inputenc} % diacritice
\usepackage[romanian]{babel}
\usepackage{color}			 % highlight
\usepackage{alltt}			 % highlight
\usepackage{code/highlight}	 	 % highlight
\usepackage{hyperref}        % folositi \url{http://...}
                             % sau \href{http://...}{Nume Link}
\mode<presentation>
{\usetheme{CDL}}

\title[Template-uri si STL]{Template-uri si STL}
\subtitle{CDL - Cursul 6}
\institute{ROSEdu}
\author{Adrian Scoica \\ {\footnotesize adrian.scoica@gmail.com}}

\begin{document}
\setbeamertemplate{frametitle continuation}[from second]
\setbeamertemplate{footline}[frame number]

\frame{\titlepage}

\frame{\tableofcontents}


\section{Introducere}

  \frame{\tableofcontents[currentsection]}

  \begin{frame}{Ce este un template?}
  Pe scurt, un template class (Ro: clasa sablon) este un mecanism care permite rezolvarea unei probleme generale implementand o singura solutie, care este valabila pentru toate tipurile de date.
  \end{frame}
 
  \begin{frame}{Exemplu concret}
  Vrem sa implementam o clasa care:\\
  \vspace{0.2cm}
  \begin{itemize}
  \setlength{\itemsep}{0.2cm}
  \pause \item Contine un membru privat de tip intreg 
  \pause \item Are un constructor care initializeaza acel membru
  \pause \item Are o metoda care afiseaza acel membru la iesirea standard
  \end{itemize}
  \pause
  \input{code/SimpleInt}
  \end{frame}

  \begin{frame}{It may work, BUT...}
  Ce facem daca vrem o clasa similara si pt char, float, short int, unsigned int, etc...?
  \begin{itemize}
  \pause \item Rescriem inutil codul si facem putine modificari
  \pause \item Cum ramane cu tipurile definite de utilizator?
  \pause \item De cate ori trebuie modificam codul daca am depistat un bug?
  \end{itemize}
  \vspace{0.6cm}
  \pause
  \rightline{$\Rightarrow$ ... it's WRONG!}
  \end{frame}

\section{Template-uri in C++}

  \frame{\tableofcontents[currentsection]}

  \begin{frame}{Solutia - folosim un template}
  Avantaje:
  \begin{itemize}
  \pause \item Implementam o singura data - efort redus :)
  \pause \item Scapam de sute de "cast"-uri ambigue (hint: void*)
  \pause \item Merge cu orice tip de date (simplu sau definit de utilizator)
  \pause \item Codul se intretine foarte usor. 
  \end{itemize}
  \pause Dezavantaje:
  \begin{itemize}
  \pause \item Compileaza [mai] lent.
  \pause \item Code bloating (more on that later)
  \end{itemize}
  \end{frame}

  \begin{frame}{Sintaxa unui template}
  \input{code/SimpleClass}
  \end{frame}
 
  \begin{frame}{Exemplu de instantiere}
  \input{code/SimpleClassTester}
  \end{frame}

  \begin{frame}{Exercitiu:}
  Scrieti un template pentru o clasa Echipa care contine:
  \begin{itemize}
  \item Doua variabile membru de tipuri neprecizate $<$T$>$ si $<$V$>$
  \item O metoda de tip "void display()" care sa afiseze continutul celor doua variabile membru sub forma ( a , b )
  \item Definiti o structura Persoana cu doi membri de tip std::string : \begin{itemize} \item nume \item prenume \end{itemize}
  \item Instantiati un obiect de tip Echipa$<$Persoana,Persoana$>$
  \end{itemize}
  ATENTIE! Atunci cand folositi tipuri de date definite de utilizator, va trebui sa definiti modul in care functioneaza asupra lor operatorii folositi in template.
  \end{frame}

  \begin{frame}{Solutie: Template-ul}
  \input{code/Echipa}
  \end{frame}

  \begin{frame}{Solutie: Definirea tipului compatibil cu Template-ul}
  \input{code/EchipaTester}
  \end{frame}

\section{STL}

  \frame{\tableofcontents[currentsection]}

  \begin{frame}
  \frametitle{Ce este STL?}
  STL (Standard Template Library) este o biblioteca(!) de template-uri foarte utile, a caror folosire creste siguranta si viteza dezvoltarii de programe.
  \pause \\Din biblioteca STL vom mentiona doua tipuri de template-uri:
  \pause 
  \begin{itemize}
  \item Containere
  \item Algoritmi
  \end{itemize}
  \end{frame}

\section{STL-Containere}

  \frame{\tableofcontents[currentsection]}

  \begin{frame}
  \frametitle{Containerele din STL}
  Containerele sunt implementari ale unor structuri de date foarte uzuale.
  \\ \pause Pentru ca o structura de date sa se comporte asemeni unui container, este foarte important ca datele sa fie opace.
  \\ \pause Avantaje:
  \begin{itemize}
  \pause \item Au in general implementari eficiente
  \pause \item Gestiunea memoriei este sigura \pause (well, sort of...)
  \pause \item Pun la dispozitie multe metode gata implementate
  \end{itemize}
  \pause
  Dezavantaje:
  \pause
  \begin{itemize}
  \item Set relativ limitat de structuri de date (desi suficient pentru aplicatii practice)
  \item Adesea sunt mai lente decat implementarile specializate 
  \end{itemize}
  \end{frame}	

  \begin{frame}
  \frametitle{Principalele structuri de date din STL}
  Containere:
  \begin{itemize}
  \item vector - Array-uri
  \item list - Liste dublu inlantuite
  \item slist - Liste simplu inlantuite
  \item deque - Double-ended queue
  \item set - Multimi (in sens matematic)
  \end{itemize}
  Adaptori pentru containere:
  \begin{itemize}
  \item queue - Structura FIFO (coada)
  \item stack - Structuri LIFO
  \item priority\_queue - Coada de prioritati
  \end{itemize}
  Toate template-urile sunt definite in namespace-ul "std" si au fisiere-antet de forma $<$vector$>$, $<$stack$>$, etc
  \end{frame}

  \begin{frame}
  \frametitle{Metode generale comune}
  \begin{itemize}
  \pause \item push\_front / pop\_front (nu si pentru "vector")
  \item push\_back / pop\_back
  \pause \item empty / size / clear 
  \item front / back (referinta la primul / ultimul element)
  \end{itemize}
  \pause Pentru "vector" si "deque" sunt definite si metode de acces prin subscripting:
  \begin{itemize}	
  \item $[]$ -  Acces fara verificarea limitelor 
  \item at  -  Acces cu verificarea limitelor
  \end{itemize}
  \pause De ce $[]$ nu verifica limitele?
  \end{frame}

  \begin{frame}
  \frametitle{Parcurgerea elementelor unui container}
  La nivel filozofic, pentru a parcurge elementele unui container ar fi suficient sa stii:
  \begin{itemize}
  \pause \item Unde sa incepi
  \pause \item Cum sa treci la urmatorul element
  \pause \item Unde sa te opresti
  \end{itemize}
  \pause
  Important: valabil INDIFERENT de modul in care e organizat intern containerul.
  \pause
  \\Solutia: iteratori 
  \\Iteratorii sunt implementati sa se comporte precum niste pointeri la date.
  \\ \pause Exista de doua tipuri:
  \begin{itemize}
  \pause \item iterator - permite sa modifici datele pe masura ce parcurgi
  \pause \item const\_iterator - nu da voie sa modifici datele
  \end{itemize}
  \end{frame}

  \begin{frame}
  \frametitle{Cum se folosesc iteratorii?}
  In primul rand, containerele trebuie sa puna la dispozitie:
  \begin{itemize}
  \pause \item Tipul iterator / const\_iterator
  \pause \item begin() - primul element (exista)
  \pause \item end() - sfarsitul (dupa ultimul element - NU exista!!)
  \pause \item Optional:
	\begin{itemize}
	\item reverse\_iterator / reverse\_const\_iterator
        \item rbegin() / rend()
        \item rend() - sfarsitul (inainte de primul elem - NU exista!!)
        \end{itemize}
  \end{itemize}
  \end{frame}

  \begin{frame}
  \frametitle{Exemplu de parcurgere a unui vector}
  \input{code/ParcurgeVector}
  \pause Ce trebuie sa schimbam pentru o lista? \pause Dar pentru o multime?
  \end{frame}

  \begin{frame}
  \frametitle{Practice}
  1. Scrieti un program simplu care sa adauge intr-o multime(= set) restul impartirii primelor 100 numere la 15
  \\2. Afisati in ordine inversa acele numere din multime divizibile la 3.
  \end{frame}

  \begin{frame}
  \frametitle{Solutie:}
  \input{code/ExempluSet}
  \end{frame}

\section{STL-Algoritmi}

  \frame{\tableofcontents[currentsection]}

  \begin{frame}
  \frametitle{Algoritmii din STL}
  \begin{itemize}
  \pause \item Sunt generici (nu depind de tipul de date)
  \pause \item In general, implementati optim
  \pause \item Again... mai lenti decat o implementare custom-made
  \pause \item Sunt definiti in antetul $<$algorithm$>$
  \end{itemize}
  Exemplu: pentru a sorta un vector de elemente, este suficient sa stim cum sa comparam elementele intre ele. \\Algoritmul in sine este acelasi.
  \end{frame}

  \begin{frame}
  \frametitle{Algoritmul sort}
  \begin{itemize}
  \pause \item Este o implementare a lui Quick Sort ($O(n*log(n))$ mediu, dar $O(n^2)$ maxim)
  \pause \item Cea mai uzuala definitie este: sort(RandomAccessIterator first, RandomAccessIterator last)
  \pause \item Nu uitati, exista si alte definitii (specificara criteriului de comparatie, de ex.)
  \pause \item Implicit, se foloseste pt comparatie operatorul "$<$". Trebuie definit daca nu exista.
  \end{itemize}
  \end{frame}

  \begin{frame}
  \frametitle{Exemplu de sortare}
  \input{code/ExempluSort}
  \end{frame}

  \begin{frame}
  \frametitle{Sortarea dupa alte criterii}
  Am vazut ca daca vrem sa sortam structuri, trebuie sa supraincarcam operatorul "$<$". 
  \vspace{0.01cm}
  \pause
  \\Ce se intampla daca vrem sa sortam intai crescator si apoi descrescator?
  \pause
  \vspace{0.01cm}
  \\Solutie: Folosim comparatori! Un comparator poate sa fie:
  \pause
  \begin{itemize}
  \item Un pointer la o functie (C-style)
  \item Un function object (numit si Functor)
  \end{itemize}
  \pause
  Observatie: un Functor este un obiect pentru care este definit operatorul "()".
  \end{frame}

  \begin{frame}
  \frametitle{Sortarea dupa alte criterii: exemplu}
  \input{code/ExempluSortCriteriu}
  \end{frame}

  \begin{frame}
  \frametitle{Cautarea unui obiect intr-o colectie}
  Pentru gasire, se poate folosi cel mai rapid algoritmul STL "find". Acest algoritm primeste ca parametru:
  \begin{itemize}
  \pause \item Doi iteratori din colectie: first si last
  \pause \item Un obiect egal (NU identic!) cu obiectul cautat.
  \end{itemize}
  \pause
  Si returneaza un iterator:
  \begin{itemize}
  \pause \item Care indica la elementul cautat din colectie, daca a fost gasit un obiect egal.
  \pause \item Egal cu "last" daca nu a fost gasit un obiect egal.
  \end{itemize}
  \pause Observatie: Cautarea se face in intervalul [first, last). Asta pentru ca in mod normal, "last" trebuie sa fie prima pozitie imediat de \_dupa\_ ultima pozitie din container.
  \end{frame}

  \begin{frame}
  \frametitle{Cautarea binara a unui element intr-o colectie}
  Observatie: Cautarea binara are sens doar pentru colectii sortate (hint: sort()).
  \pause
  Exista mai multe variante ale cautarii binare:
  \begin{itemize}
  \pause \item binary\_seach(ForwardIterator first, ForwardIterator last, const T\& value); 
  \end{itemize}
  \pause
  Sau cu un criteriu de comparatie custom-made:
  \begin{itemize}
  \pause \item binary\_seach(ForwardIterator first, ForwardIterator last, const T\& value, Compare comp); 
  \end{itemize}
  \pause Observatie: tipul intors este "bool" (deci putem afla doar daca exista sau nu, nu putem avea acces la acel element gasit).
  \end{frame}

  \begin{frame}
  \frametitle{Alte template-uri de functii utile}
  \begin{itemize}
  \item replace(ForwardIterator first, ForwardIterator last, const T\& replacethis, const T\& withthis)
  \item replace\_if(ForwardIterator first, ForwardIterator last, Predicate condition, const T\& withthis)
  \item count(ForwardIterator first, ForwardIterator last, const T\& countthis)
  \item swap(ForwardIterator a, ForwardIterator b)
  \end{itemize}
  etc...
  \end{frame}

\section{Tips \& Tricks}

  \frame{\tableofcontents[currentsection]}

  \begin{frame}
  \frametitle{Facts about $<$vector$>$}
  Acest template este probabil cel mai folosit template din STL.
  \begin{itemize}
  \pause \item vector$<$bool$>$ este o specializare pe biti $\Rightarrow$ eficienta timp / spatiu
  \pause \item Folositi vector::swap pentru a interschimba doi vectori (Complexitate $O(1)$ !)
  \pause \item Daca adaugati la vector in timp ce il parcurgeti cu un iteratori, riscati $Segfault$. Posibila solutie: vector::reserve
  \end{itemize}
  \end{frame}

  \begin{frame}
  \frametitle{Facts about container adapters}
  Atentie: \begin{itemize} \item stack \item queue \item priority\_queue \end{itemize} NU sunt containere!
  \pause Ele sunt adaptori de containere ("container adapters"). Rolul lor in limitarea accesului.
  \pause De exemplu:
  \begin{itemize}
  \pause \item stack$<$int$>$ - declaratie implicita. Se instantiaza un deque$<$int$>$! (+ adaptare)
  \pause \item stack$<$ int, vector$<$int$>$ $>$ - declaratie explicita. Se instantiaza un vector$<$int$>$! (+ adaptare)
  \end{itemize}
  Observatie!!
  \begin{itemize}
  \pause \item Atunci cand instantiati template-uri, este OBLIGATORIU sa puneti spatii intre simboluri $<$ sau $>$ succesive. 
  \pause \item $<<$ si $>>$ sunt operatori! $\Rightarrow$ eroare de compilare.
  \end{itemize}
  \end{frame}

  \begin{frame}
  \frametitle{Template-urile si mostenirea}
  Este important atunci cand lucram cu template-urile sa retinem urmatoarele ecuatii: 
  \pause \centerline{Template Class + Tipuri = Class}
  \pause \centerline{Class + "imbunatatiri" = Shortcut $>$:)}
  \begin{itemize}
  \pause \item De exemplu, daca dorim sa implementam o structura de date care retine o versiune "arhivata" a datelor, suntem tentati sa derivam o specializare a unui container din STL. (Java-style)
  \pause \item Desi posibil, acest lucru este in general RAU!!  
  \pause \item Motivul? \pause Destructorii containerelor STL NU sunt virtuali. (ce inseamna asta?)
  \pause \item Solutia corecta? \pause Facem un adaptor (un wrapper pentru containerul care ne intereseaza).
  \end{itemize}
  \pause P.S. Care e diferenta intre un destructor si o alta functie membru supradefinita?
  \end{frame}

  \begin{frame}
  \frametitle{O stiva care retine datele in format arhivat - part 1}
  \input{code/DerivareContainer1}
  \end{frame}

  \begin{frame}
  \frametitle{O stiva care retine datele in format arhivat - part 2}
  \input{code/DerivareContainer2}
  \end{frame}

\section{Concluzie}
  
  \frame{\tableofcontents[currentsection]}
  
  \begin{frame}{Concluzie}
  Cand folosim template-uri ?
  \begin{itemize}
  \pause \item Cand avem de scris cod care nu tine cont de tipurile de date inglobate SI altfel am avea de rescris
  \pause \item Cand vrem sa folosim cod care sigur merge
  \pause \item Cand nu vrem/stim sa implementam de mana :D
  \end{itemize}
  \pause 
  Cand nu folosim template-uri?
  \begin{itemize}
  \pause \item Cand putem scoate implementari mai bune daca tinem cont de tipul de date. \pause (exemple?)
  \pause \item Cand nu vom avea nevoie sa refolosim aceeasi clasa, dar pentru alt tip de date.
  \pause \item Cand vrem sa evitam "code bloating".
  \pause \item Cand proiectului nostru ii trebuiesc deja 12 minute sa compileze si folosirea unor template-uri nu se justifica in mod deosebit :D
  \end{itemize}
  \end{frame}

\section{Further Reading}

  \frame{\tableofcontents[currentsection]}

  \begin{frame}{Further Reading}
  Link-uri
  \begin{itemize}
  \setlength{\itemsep}{0.5cm}
  \item \href{http://www.cplusplus.com/reference}{Sursa oficiala de documentatie pt C++}
  \item \href{http://www.cs.brown.edu/\~jak/proglang/cpp/stltut/tut.html}{Un tutorial de baza pentru STL}
  \item \href{http://www.sgi.com/tech/stl/}{Un tutorial ceva mai serios despre STL}
  \item \href{http://www.boost.org/}{Site-ul oficial al bibliotecii BOOST}
  \end{itemize}
  Carti
  \begin{itemize}
  \item "The C++ Programming Language - Special Edition", Bjarne Stroustrup
  \item "C++", Danny Kalev, Michael J. Tobler, Jan Walter
  \end{itemize}
  \end{frame}

\end{document}
