% vim: set tw=78 aw:
\documentclass{beamer}

\usepackage[utf8x]{inputenc} % diacritice
\usepackage[romanian]{babel}
\usepackage{hyperref}        % folositi \url{http://...}
% sau \href{http://...}{Nume Link}
\mode<presentation>
{ \usetheme{Rochester} }		% TODO: settle this

% Titlul nu foloseşte Unicode pentru că e o problemă căreia nu i-am dat de
% cap.
\title[Eclipse IDE]{Eclipse IDE}
\subtitle{CDL - Cursul 6}
\institute{Ceata}
\author{Dascalu Laurentiu - dascalu.laurentziu@gmail.com}

\begin{document}

% Slide-urile cu mai multe părţi sunt marcate cu textul (cont.)
\setbeamertemplate{frametitle continuation}[from second]
% Arătăm numărul frame-ului
\setbeamertemplate{footline}[frame number]

\frame{\titlepage}

\frame{\tableofcontents}

% NB: Secţiunile nu sunt marcate vizual, ci doar apar în cuprins.
\section{Introducere}

% Pentru reamintirea periodică a cuprinsului şi unde ne aflăm:
\frame{\tableofcontents[currentsection]}

% Titlul unui frame se specifică fie în acolade, imediat după \begin{frame},
% fie folosind \frametitle
\begin{frame}{Introducere}
  \begin{itemize}
   \item Site oficial : eclipse.org
   \item IDE - Integrated development enviroment; plug-in system
   \item Eclipse Public License; free and open source software; dezvoltat de industrie : Adobe, Freescale, IBM, Atmel etc
   \item Competitori : Visual Studio, Netbeans
  \end{itemize}
\end{frame}

\begin{frame}{Caracteristici generale I}
  \begin{itemize}
    \item Principalele aspecte \textbf{pozitive}:
    \begin{itemize}
      \item acelasi GUI, indiferent de limbajul de programare; comportamentul este definit de plug-in-uri, prin wrapping peste utilitare: compilator, debugger, profiler etc
      \item foarte bine scris; proiect open-source reusit
      \item independent de platforma (SO); dependent de platforma $<=>$ Java este platforma
      \item se poate integra cu orice tip de serviciu : de la e-mail pana la modelare UML
    \end{itemize}
  \end{itemize}
\end{frame}

\begin{frame}{Caracteristici generale II}
  \begin{itemize}
    \item Principalele aspecte \textbf{negative}:
    \begin{itemize}
      \item interfata grafica complicata
      \item scris in java si necesita JRE pentru a porni
      \item necesita mult RAM; plug-in developmentul necesita hardware de ultima generatie
      \item project based; nu se preteaza pentru programe mici ( ${'wc -l'}$ $<$ 300)
    \end{itemize}
  \end{itemize}
\end{frame}

\begin{frame}{Ce vom face in Eclipse?}
  \begin{itemize}
  \item Instalam plugin-uri : git/svn, CDT
  \item Print-o aplicatie, exploram feature-urile IDE-ului :
    \begin{itemize}
      \item Menutar, toolbar
      \item Editor : autocomplete(variabile, nume de functii, membri ai structurilor), macro-expansion, refactoring, on the fly spell check
      \item Help $->$ Software Updates $->$ online plugin install; instalarea offline a plugin-urilor este putin mai complicata si nu se va trata in acest curs
      \item Perspectiva grafica; strans legata de limbajul de programare in care lucram
    \end{itemize}
  \end{itemize}
\end{frame}

\section{Plugin install}
\frame{\tableofcontents[currentsection]}

\begin{frame}{Plugin install}
\begin{itemize}
  \item Instalam plugin-ul de SVN
  \item Help $->$ Software Updates $->$ Available Software $->$ Add site $->$ $http://subclipse.tigris.org/update\_1.4.x$
  \item Expandam noua inregistrare pentru a vedea ce contine; selectam toate cele 3 module si apoi apasam Install
  \item Next, Next, Accept la licenta si restart la Eclipse
  \item Am terminat deinstalat plugin-ul; instalarea online a altor plugin-uri este foarte asemanatoare
\end{itemize}
\end{frame}

\section{Aplicatie}
\frame{\tableofcontents[currentsection]}

\begin{frame}{Aplicatie Part I}
\begin{itemize}
  \item New $->$ Other $->$ Checkout Projects from SVN
  \item Create new repository location $->$ svn://cdl.rosedu.org/cdl-lab7 $->$ Next, Next
  \item Expandati eclipse $->$ Selectati eclipse-code $->$ Next, Next pana s-a creat inregistrarea in Project explorer
  \item Analizati continutul fisierelor sursa si header. Rezolvati micile probleme folosind cat mai multe feature-uri prezentate.
\end{itemize}
\end{frame}

\begin{frame}{Aplicatie Part II}
\begin{itemize}
  \item Incercati sa folositi cat mai multe din urmatoarele :
    \begin{itemize}
      \item CTRL+B pentru build (compileaza sursele modificate de la ultimul build)
      \item CTRL+SPACE pentru autocompletarea numelor - scrierea malloc-ului, accesarea unui membru dintr-o structura
      \item click stanga pe nume $->$ click dreapta $->$ open declaration - vedeti continutul structurii FILE
      \item plimbati mouse-ul pe :
        \begin{itemize}
	  \item numele functiei $->$ descriere
	  \item apelul macro-ului $->$ expandare
	  \item numele variabilei $->$ tip
	\end{itemize}
      \item CTRL + SHIFT + F pentru indentarea codului
    \end{itemize}
\end{itemize}
\end{frame}

\begin{frame}{Aplicatie Part III}
\begin{itemize}
  \item Schimbati numele .c si .h pentru a nu aparea coliziuni de nume
  \item Click dreapta pe numele proiectului $->$ Team $->$ Commit....
  \item Completati cerintele editorului pentru a comite modificarile pe SVN
\end{itemize}
\end{frame}

\begin{frame}{Aplicatie Part IV}
\begin{itemize}
  \item Creati un proiect nou, hello world! Project $->$ New Project $->$ C project
  \item Comiteti proiectul pe svn :
  \begin{itemize}
    \item Click dreapta pe numele proiectului
    \item Team $->$ Share project... $->$ SVN $->$ Use existing repository $->$ Next, next
  \end{itemize}
\end{itemize}
\end{frame}

\section{Link-uri utile}
\frame{\tableofcontents[currentsection]}
\begin{frame}{Link-uri utile}
\begin{itemize}
  \item $eclipse.org$
  \item $www.eclipseplugincentral.com$
\end{itemize}
\end{frame}

\end{document}

