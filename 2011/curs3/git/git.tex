% vim: set tw=78 aw sw=2 sts=2 noet:
\documentclass{beamer}

%\includeonlyframes{c} -- speeding up compilation speed during debug

\usepackage[utf8x]{inputenc} % diacritice
\usepackage[romanian]{babel}
\usepackage{hyperref}        % folositi \url{http://...}
\mode<presentation>
\usetheme{CDL}

\title[]{Git}
\subtitle{CDL 2011 - Cursul 3}
\institute[]{ROSEdu}
\author[]{Mihai Maruseac \\ \texttt{mihai@rosedu.org} \\ (with content from Mircea Bardac)}

\setbeamertemplate{frametitle continuation}[from second]
\setbeamertemplate{footline}[frame number]

\pgfdeclareimage[height=3cm]{m1}{img/01}
\pgfdeclareimage[height=3cm]{m2}{img/02}
\pgfdeclareimage[height=6cm]{m3}{img/03}
\pgfdeclareimage[height=6.5cm]{m4}{img/04}
\pgfdeclareimage[height=5cm]{m5}{img/05}

\begin{document}

\maketitle

\section{Source Control Management}

\begin{frame}{Intro}
  \begin{alertblock}{Quote}
    \texttt{\textit{Perilous to us all are the devices of an art deeper than
    we posses ourselves.}}
    \flushright{(Tolkien)}
  \end{alertblock}
  \pause
  \begin{alertblock}{Quote}
    \texttt{\textit{Keep your code safe. Tattoo that sentence backward across
    your forehead and stare in the mirror for 10 minutes every morning.}}
    \flushright{(adapted from Pete Goodliffe)}
  \end{alertblock}
\end{frame}

\begin{frame}{Source Control Management}
  \begin{itemize}[<+->]
    \item aka Revision Control
    \item urmărire/stocare modificări
      \begin{alertblock}{Careful!}
        NU copy-paste în directoare separate
      \end{alertblock}
    \item colaborare, proiecte mari
      \begin{alertblock}{Careful!}
        NU mail / dropbox
      \end{alertblock}
  \end{itemize}
\end{frame}

\begin{frame}{Concepte}
  \begin{description}[<+->]
    \item[repository] = ...
    \item[checkout,clone] = make local copy
    \item[working copy] = copie locală, \textbf{sandbox}
    \item[change] = modificare
    \item[commit] = submit changes
    \item[update] = sync
    \item[merge] = aplicare a 2+ commits
    \item[conflict] = schimbări în același document
    \item[version] = revision
    \item[HEAD] = cel mai recent commit (tip)
    \item[branch] = alternativă development
    \item[label] = punct important din development (tag)
  \end{description}
\end{frame}

\begin{frame}{Unde e repository-ul?}
  \begin{itemize}
    \item centralizat
      \begin{itemize}
        \item server: DB
        \item client: working copy
      \end{itemize}
      \pause
      \begin{description}
        \item[+] totul la un loc
        \item[--] bottleneck, \textit{single point of failure}
      \end{description}
    \pause
    \item descentralizat
      \begin{itemize}
        \item P2P
        \item operații/concepte noi
        \item repo si working-tree simultan
        \item istoric si lucru complet descentralizat
      \end{itemize}
  \end{itemize}
\end{frame}

\begin{frame}{SCM Descentralizat}
  \begin{itemize}
    \item avantaje
      \begin{itemize}
        \item fără \textit{single point of failure}
        \item lucru deconectat
        \item operații rapide
        \item modele noi de development
      \end{itemize}
    \pause
    \item dezavantaje
      \begin{itemize}
        \item mai greu de înteles
        \item cine are acces?
        \item istoria e permanentă (almost)
      \end{itemize}
    \pause
    \item operatii noi
      \begin{description}[<+->]
        \item[push] - sync eu $\rightarrow$ remote
        \item[fetch,pull] - sync eu $\leftarrow$ remote
        \item[remote] - ...
      \end{description}
  \end{itemize}
\end{frame}

\begin{frame}{SCMs}
  \begin{itemize}
    \item \textbf{SVN} (Subversion), centralizat, C
    \item \textbf{Git} descentralizat, Perl, C, shell
    \item \textbf{Mercurial} descentralizat, $<$ Git (see wiki ref), Python, C, shell
    \item \textbf{darcs} patch (see Darcs ref), Haskell, lazy
  \end{itemize}
  \pause
  \begin{alertblock}{Hint}
    Toate sunt folosite. Încercați-le și voi :)
  \end{alertblock}
\end{frame}

\section{Git}

\begin{frame}{Git history}
  \begin{itemize}
    \item Linus Torvalds
    \item Linux Kernel (moved from BitKeeper)
    \item design:
      \begin{itemize}
        \item \textit{CVS as example of what \textbf{not} to do}
        \item distribuit, workflow a la BitKeeper
        \item protecție contra distrugerilor/erorilor
        \item perfomanță
      \end{itemize}
    \item development start: 3 aprilie 2005
    \item 29 aprilie 2005: 6.7 patch-uri/s în kernel via Git
    \item 16 iunie 2005: 2.6.12 pe Git
    \item 31 ianuarie 2011: git 1.7.4
    \item C, Perl, shell
  \end{itemize}
\end{frame}

\begin{frame}{Git mits}
  \begin{itemize}[<+->]
    \item git is hard to learn
    \item git is slow
    \item git nu e matur
    \item nimeni nu folosește git
    \item proiectele care folosesc git au un SVN in spate
    \item nu poti muta proiecte de pe SVN pe Git
  \end{itemize}
\end{frame}

\begin{frame}{Git metadata}
  \begin{itemize}[<+->]
    \item director \texttt{.git} \textit{doar} în rădăcină
    \item conținut:
      \begin{itemize}
        \item \texttt{.git/config} fișier configurare
        \item \texttt{.git/hooks} scripturi pentru evenimente
        \item \texttt{.git/refs} referințe (la ce?)
      \end{itemize}
  \end{itemize}
\end{frame}

\begin{frame}{Git config}
  \begin{itemize}
    \item local în \texttt{.git/config}
    \item global în \texttt{\$HOME/.gitconfig}
  \end{itemize}
  \pause
  \begin{alertblock}{Task 1}
    \begin{itemize}
      \item install git
      \item \texttt{git config --global user.name "\textit{nume prenume}"}
      \item \texttt{git config --global user.email "\textit{nume@dom.com}"}
      \item \texttt{git config --global color.ui auto}
      \item \texttt{git config --global color.pager true}
      \item \texttt{git config --global core.editor \textit{editor}}
    \end{itemize}
  \end{alertblock}
\end{frame}

\begin{frame}{Git commands}
  \begin{itemize}
    \item \texttt{git \textit{cmd args}}
    \item \texttt{git help}
    \item \texttt{git help \textit{cmd}}
    \item \texttt{git \textit{cmd} help}
    \item \texttt{man git-\texttt{cmd}}
  \end{itemize}
  \pause
  \begin{alertblock}{Task 2}
    \begin{itemize}
      \item Ce reprezenta opțiunea \texttt{color.pager} de la \texttt{git config}?
    \end{itemize}
  \end{alertblock}
\end{frame}

\begin{frame}{Primul repo}
  \begin{itemize}
    \item \texttt{git init} într-un director cu surse
    \item \texttt{git clone \textit{url}}
  \end{itemize}
  \pause
  \begin{alertblock}{Task 3}
    \begin{itemize}
      \item Obțineți repository-ul de la adresa TODO
      \item listați conținutul directorului creat
    \end{itemize}
  \end{alertblock}
\end{frame}

\begin{frame}[label=Content]{Conținut}
  \begin{alertblock}{Careful!}
    Git urmărește conținut, nu fișiere!!
  \end{alertblock}
  \begin{itemize}
    \item work tree (working dir): local
    \item index: pregătit de commit
    \item repository: commit salvat
  \end{itemize}
  \pgfuseimage{m5}
\end{frame}

\begin{frame}{Staging}
  \begin{itemize}
    \item work tree $\rightarrow$ index
    \item \texttt{git add \textit{filename}}
    \item \texttt{git rm \textit{filename}}
    \item \texttt{git mv \textit{filename}}
  \end{itemize}
  \pause
  \begin{alertblock}{Task 4}
    \begin{itemize}
      \item Modificați sursa \texttt{main.c} din director, adăugând numele vostru pe linia
      corespunzătoare.
      \item Mutați în index fișierul modificat
    \end{itemize}
  \end{alertblock}
\end{frame}

\begin{frame}{Commiting}
  \begin{itemize}
    \item index $\rightarrow$ repository \textbf{local}
    \item \texttt{git commit \textbf{-m "\textit{mesaj}"}}
  \end{itemize}
  \begin{alertblock}{Careful!}
    Nu folosiți mesaje goale!
  \end{alertblock}
\end{frame}

\begin{frame}{Shortcuts}
  \begin{itemize}
    \item \texttt{git commit -am "..."}
    \item \texttt{git add .}
    \pause
    \begin{alertblock}{Tip}
      Nu rulați \texttt{git commit -am..}.
    \end{alertblock}
    \begin{alertblock}{Careful!}
      Nu rulați \texttt{git add .} dacă nu aveți totul setat ok.
    \end{alertblock}
    \pause
    \item probleme:
      \begin{itemize}
        \item fișiere executabile (\texttt{a.out}), obiect (\texttt{a.o})
        \item fișiere swap (\texttt{.git.tex.swp})
        \item fișiere locale, personale
        \item etc.
      \end{itemize}
  \end{itemize}
\end{frame}

\begin{frame}{Ignore}
  \begin{itemize}
    \item \texttt{.gitignore} - \textbf{global!!}
    \item \texttt{.git/info/exclude} - local
  \end{itemize}
  \pause
  \begin{alertblock}{Task 5}
    \begin{itemize}
      \item Ignorați \textbf{local} fișierele obiect și executabilul generat.
    \end{itemize}
  \end{alertblock}
\end{frame}

\begin{frame}{Commit (2)}
  \begin{itemize}
    \item fiecare commit are un ID: SHA1
    \item fiecare commit are cel puțin un părinte (cu excepția primului)
    \item graf aciclic de commit-uri
    \pause
    \item undo: \texttt{git reset}
    \item type: \texttt{git commit --amend}
    \item revenire la un working tree anterior: \texttt{git revert \textit{id}}
    (nerecomandat)
    \item doar pt un fișier: \texttt{git checkout \textit{file}}
    \pause
    \item cum referim un commit?
  \end{itemize}
\end{frame}

\begin{frame}{Commit (3)}
  \begin{description}
    \item[\texttt{HEAD}] = ultimul commit
    \item[\texttt{HEAD\textasciicircum}] = penultimul
    \item[\texttt{HEAD\textasciicircum\textasciicircum}] = antepenultimul
    \item[...] = etc
    \pause
    \item[\texttt{HEAD\textasciitilde5}] = \texttt{HEAD\textasciicircum\textasciicircum\textasciicircum\textasciicircum\textasciicircum}
    \pause
    \item[a946e644d5b4c26aaa4e73338805e207bbfd78b0] = full hash
    \pause
    \item[a946e6] = short hash
  \end{description}
\end{frame}

\begin{frame}{Status}
  \begin{itemize}
    \item \texttt{git status}
    \item prezintă:
      \begin{itemize}
        \item conținut din index {staged}
        \item conținut care nu e în index {non-staged}
        \item conținut neurmărit (non-tracked)
        \item other info
      \end{itemize}
  \end{itemize}
  \pause
  \begin{alertblock}{Task 6}
    \begin{itemize}
      \item Modificați fișierul \texttt{README}, adăugând numele și grupa pe linia
      corespunzătoare.
      \item creați un fișier numit după voi în working dir
      \item vizualizați starea working dir.
      \item faceți working dir clean
    \end{itemize}
  \end{alertblock}
\end{frame}

\begin{frame}{Diff}
  \begin{itemize}
    \item \texttt{git diff} - diferențe index - working tree
    \item \texttt{git diff --cached} - conținut index
    \item \texttt{git diff \textit{idcommit1}} - commit și working dir
    \item \texttt{git diff \textit{idcommit1} \textit{idcommit2}} - commit și
    commit
  \end{itemize}
  \pause
  \begin{alertblock}{Task 7}
    \begin{itemize}
      \item obtineți un diff între HEAD și starea inițială a repo-ului
    \end{itemize}
  \end{alertblock}
\end{frame}

\begin{frame}{Remote Repository}
  \begin{itemize}
    \item clone = copie repo remote
    \item commit /= update goes to repo
    \item \texttt{git push origin master} - trimite commit-urile în branch-ul
    master din repository-ul referit de origin
    \item dacă nu pot exista confuzii, merge și \texttt{git push}
  \end{itemize}
  \pause
  \begin{alertblock}{Task 8}
    \begin{itemize}
      \item Transmiteți commit-urile voastre repository-ului central (de unde
      ați clonat).
    \end{itemize}
  \end{alertblock}
\end{frame}

\begin{frame}{Remote Repository(2)}
  \begin{itemize}
    \item \texttt{git pull}
    \item \texttt{git fetch ...}
    \item obțin commit de la un remote
    \item actualizează repository-ul local
  \end{itemize}
  \pause
  \begin{alertblock}{Task 9}
    \begin{itemize}
      \item Actualizați repository-ul local cu cel remote.
    \end{itemize}
  \end{alertblock}
\end{frame}

\begin{frame}{Remote Repository(3)}
  \begin{itemize}
    \item cum adăugăm un alt remote?
    \item \texttt{git remote add \textit{nume} \textit{url}}
    \pause
    \item cum vedem ce remote-uri există?
    \item \texttt{git remote show}
  \end{itemize}
\end{frame}

\begin{frame}{Branches}
  \begin{itemize}[<+->]
    \item \texttt{git push origin master} - trimite commit-urile în branch-ul
    master din repository-ul referit de origin
    \item ramuri de dezvoltare
    \item locale: \texttt{git branch}
    \item remote: \texttt{git branch -r}
    \item toate: \texttt{git branch -a}
    \item *\textit{numebranch}: branch curent
  \end{itemize}
  \pause
  \begin{alertblock}{Task 10}
    \begin{itemize}
      \item Listați \textbf{toate} branch-urile din repository.
    \end{itemize}
  \end{alertblock}
\end{frame}

\begin{frame}{Branches (2)}
  \begin{itemize}
    \item creare:
      \begin{itemize}
        \item \texttt{git branch \textit{nume} \textit{startref}}
        \item \texttt{git branch \textit{nume}}
        \item \texttt{git checkout -b \textit{nume}}
      \end{itemize}
    \pause
    \item ștergere:
      \begin{itemize}
        \item \texttt{git branch -d \textit{bname}}
        \item \texttt{git branch -D \textit{bname}}
      \end{itemize}
    \pause
    \item schimbare branch:
      \begin{itemize}
        \item \texttt{git checkout \textit{bname}}
      \end{itemize}
  \end{itemize}
  \pause
  \begin{alertblock}{Task 11}
    \begin{itemize}
      \item Schimbați în branch-ul corespunzător numelui vostru.
      \item Listați conținutul directorului
      \item Ce s-a întâmplat cu sursele vechi?
    \end{itemize}
  \end{alertblock}
\end{frame}

\againframe{Content}

\begin{frame}{Branches (3)}
  \begin{alertblock}{Task 12}
    \begin{itemize}
      \item Completați fișierul \texttt{info} cu ce vreți voi
      \item reveniți în branch-ul master și faceți pull (folosiți argumentul
      --all)
    \end{itemize}
  \end{alertblock}
\end{frame}

\begin{frame}{Best Practices}
  \begin{itemize}
    \item repository: mereu compilabil
    \item mesaje de commit relevante
    \item commit-uri mici: un singur bugfix/feature/patch
  \end{itemize}
\end{frame}

\section{More git}

\begin{frame}{Log}
  \begin{itemize}
    \item istoria din repo
    \item \texttt{git log}: toată istoria
    \item \texttt{git log \textit{ref1}..\textit{ref2}}: între refs
    \item \texttt{git log \textit{file}}: doar file
  \end{itemize}
\end{frame}

\begin{frame}{Tools}
  \begin{itemize}
    \item repo-git human read-able
    \begin{itemize}
      \item gitk - Tcl/Tk, cel mai cunoscut
      \item giggle - Gnome, GTK+
      \item tig - command line
    \end{itemize}
  \end{itemize}
  \begin{alertblock}{Hint}
    Toate sunt folosite. Încercați-le și voi :)
  \end{alertblock}
\end{frame}

\begin{frame}{Tools(3)}
  \pgfuseimage{m3}
\end{frame}

\begin{frame}{Tools(4)}
  \pgfuseimage{m4}
\end{frame}

\begin{frame}{Tools(5)}
  \begin{alertblock}{Task 13}
    \begin{itemize}
      \item Instalați gitk și vizualizați starea repo-ului
    \end{itemize}
  \end{alertblock}
\end{frame}

\begin{frame}{Merge vs Rebase}
  \begin{itemize}
    \item Ce se întâmplă la pull când avem și local commit-uri noi?
    \pause
    \item merge - combină commit-urile
    \item rebase - reordonează istoria
  \end{itemize}
  \pgfuseimage{m1}
  \pgfuseimage{m2}
\end{frame}

\begin{frame}{Tags}
  \begin{itemize}
    \item momente importante din development (release, etc)
    \item creat cu \texttt{git tag \textit{nume} \textit{ref}}
    \item șters cu \texttt{git tag -d \textit{nume}}
    \item Putem referi commit-uri cu tag-uri prin numele tagurilor
  \end{itemize}
\end{frame}

\begin{frame}[label=l]{Links}
  \begin{itemize}
    \item \href{http://en.wikipedia.org/wiki/Comparison_of_revision_control_software}{Comparație între SCM-uri}
    \item \href{http://en.wikibooks.org/wiki/Understanding_Darcs/Patch_theory}{Darcs Patch Theory}
    \item \href{http://talks.rosedu.org/prezentari/prezentarea03}{Prezentare
    Tech Talks Git (Mircea Bardac)}
    \item \href{http://github.com}{GitHub}
    \item
    \href{http://www.eqqon.com/index.php/Collaborative_Github_Workflow}{Workflof
    GitHub}
  \end{itemize}
\end{frame}

\section{Extra git}

\begin{frame}{Git bisect}
  \begin{itemize}
    \item identifică commit-ul care a introdus un bug
    \item \texttt{git bisect start}
    \item \texttt{git bisect bad}
    \item \texttt{git bisect good \textit{commit\_ok}}
    \item view, test, repeat
    \pause
    \item \texttt{git blame}
  \end{itemize}
  \pause
  \begin{alertblock}{Task 14}
    \begin{itemize}
      \item branch-ul \texttt{fault} conține un bug. Identificați commit-ul
      care l-a generat.
    \end{itemize}
  \end{alertblock}
\end{frame}

\begin{frame}{GitHub}
  \begin{itemize}
    \item web-based repo for git
    \item repos și gists
    \item pages
    \pause
    \item 1. fork
    \item 2. clone your own repo
    \item 3. commit early and often
    \item 4. fork queue / pull requests
    \item 5. repeat / profit :D
  \end{itemize}
\end{frame}

\section{Even more extra git}

\begin{frame}{Cherry pick}
  \begin{itemize}
    \item cod dintr-un branch în altul
    \item util când branch-urile diverg foarte tare
    \item se `culeg' commit-urile relevante
    \item \texttt{git cherry-pick \textit{IDcommit}}
  \end{itemize}
\end{frame}

\againframe{l}

\end{document}
