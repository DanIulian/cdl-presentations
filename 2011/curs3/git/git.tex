% vim: set tw=78 aw sw=2 sts=2 noet:
\documentclass{beamer}

%\includeonlyframes{c} -- speeding up compilation speed during debug

\usepackage[utf8x]{inputenc} % diacritice
\usepackage[romanian]{babel}
\usepackage{hyperref}        % folositi \url{http://...}
\mode<presentation>
\usetheme{CDL}

\title[]{Git}
\subtitle{CDL 2011 - Cursul 2}
\institute[]{ROSEdu}
\author[]{
  Mihai Maruseac \texttt{mihai@rosedu.org} \\
  Razvan Deaconescu \texttt{razvan@rosedu.org}
}

\setbeamertemplate{frametitle continuation}[from second]
\setbeamertemplate{footline}[frame number]

%\pgfdeclareimage[height=3cm]{m1}{img/01}
%\pgfdeclareimage[height=3cm]{m2}{img/02}
%\pgfdeclareimage[height=6cm]{m3}{img/03}
%\pgfdeclareimage[height=6.5cm]{m4}{img/04}
%\pgfdeclareimage[height=5cm]{m5}{img/05}

\begin{document}

\maketitle

\begin{frame}{Before start}
  \begin{alertblock}{Homework}
    Parcurgeți \url{http://gitimmersion.com/} înaintea cursului.
  \end{alertblock}
\end{frame}

\begin{frame}{Git history}
  \begin{itemize}
    \item Linus Torvalds
    \item Linux Kernel (moved from BitKeeper)
    \item design:
      \begin{itemize}
        \item \textit{CVS as example of what \textbf{not} to do}
        \item distribuit, workflow a la BitKeeper
        \item protecție contra distrugerilor/erorilor
        \item perfomanță
      \end{itemize}
    \item development start: 3 aprilie 2005
    \item 29 aprilie 2005: 6.7 patch-uri/s în kernel via Git
    \item 16 iunie 2005: 2.6.12 pe Git (acum și pe GitHub)
    \item 31 ianuarie 2011: git 1.7.4
    \item C, Perl, shell
  \end{itemize}
\end{frame}

\section{Live demo}

\begin{frame}{Config}
  \begin{itemize}
    \item \texttt{git config --global user.name "\textit{nume prenume}"}
    \item \texttt{git config --global user.email "\textit{nume@dom.com}"}
    \item \texttt{git config --global color.ui auto}
    \item \texttt{git config --global color.pager true}
    \item \texttt{git config --global core.editor \textit{editor}}
    \item \texttt{cat $\sim$/.gitconfig}
  \end{itemize}
\end{frame}

% Vom scrie un program care va primi un număr ca argument și-l va afișa
\begin{frame}{First repo}
  \begin{itemize}
    \item \texttt{git init}
    \item \texttt{git status}
    \item ignore % binar + executabil
    \item \texttt{git add}
    \item \texttt{git commit}
  \end{itemize}
  % TODO: poza cu cele 3 zone: work tree, index, repo
\end{frame}

\begin{frame}{More about Commits}
  \begin{itemize}
    \item date
    \item metadate
    \pause
    \begin{itemize}
      \item ID (SHA1)
      \item parent(s)
      \item tree
      \item commit msg
    \end{itemize}
  \end{itemize}
  % TODO: poza cu un singur commit
\end{frame}

\begin{frame}{Commit Reference}
  \begin{description}
    \item[\texttt{HEAD}] = ultimul commit
    \item[\texttt{HEAD\textasciicircum}] = penultimul
    \item[\texttt{HEAD\textasciicircum\textasciicircum}] = antepenultimul
    \item[...] = etc
    \pause
    \item[\texttt{HEAD\textasciitilde5}] = \texttt{HEAD\textasciicircum\textasciicircum\textasciicircum\textasciicircum\textasciicircum}
    \pause
    \item[a946e644d5b4c26aaa4e73338805e207bbfd78b0] = full hash
    \item[a946e6] = short hash
  \end{description}
\end{frame}

\begin{frame}{Going public}
  \begin{itemize}
    \item \texttt{git remote}
    \item \texttt{git push}
  \end{itemize}
  % TODO poza cu cele 2 repo-uri de acum
\end{frame}

\begin{frame}{Contributing}
  \begin{itemize}
    \item \texttt{git clone}
    \item small change
    \item \texttt{git diff}
    \item \texttt{git commit}
    \item \texttt{git push}
    \item \texttt{git pull}
  \end{itemize}
  % TODO poza cu cele 3 repo-uri
\end{frame}

\begin{frame}{Repo status}
TODO pic - detaliat unul din repo-uri în stilul
\url{http://learn.github.com/p/intro.html\#snapshots\_not\_changesets}
\end{frame}

\begin{frame}{Working in Parallel}
  \begin{itemize}
    \item \texttt{git branch}
    \item \texttt{git pull}
    \item \texttt{git pull --rebase}
  \end{itemize}
\end{frame}

\begin{frame}{Repository Viewing}
  \begin{itemize}
    \item \texttt{git log}
    \item \texttt{git show}
    \item gitk, gitg, tig
    \item tags
  \end{itemize}
\end{frame}

\begin{frame}{Fix Wrong Commit}
  \begin{itemize}
    \item \texttt{git rebase}
    \item \texttt{git ammend}
  \end{itemize}
\end{frame}

\begin{frame}{Bug Fixing}
  \begin{itemize}
    \item \texttt{git stash}
    \item \texttt{git bisect}
  \end{itemize}
\end{frame}

\begin{frame}{Extra - Mailing Patches}
  \begin{itemize}
    \item \texttt{git format-patch}
    \item \texttt{git send-email}
  \end{itemize}
\end{frame}

\begin{frame}[label=l]{Links}
  \begin{itemize}
    \item \href{http://en.wikipedia.org/wiki/Comparison_of_revision_control_software}{Comparație între SCM-uri}
    \item \href{http://talks.rosedu.org/prezentari/prezentarea03}{Prezentare
    Tech Talks Git (Mircea Bardac)}
    \item \href{http://nvie.com/posts/a-successful-git-branching-model/}{A
    branching model}
    \item \href{http://github.com}{GitHub}
    \item
    \href{http://www.eqqon.com/index.php/Collaborative_Github_Workflow}{Workflof
    GitHub}
  \end{itemize}
\end{frame}

\end{document}
