% vim: set tw=78 aw:
\documentclass{beamer}

\usepackage[utf8x]{inputenc} % diacritice
\usepackage[romanian]{babel}
\usepackage{hyperref}        % folositi \url{http://...}
% sau \href{http://...}{Nume Link}
\mode<presentation>
\usetheme{CDL}

% Titlul nu foloseşte Unicode pentru că e o problemă căreia nu i-am dat de
% cap.
\title{Cursul de Dezvoltare Liberă la final}
\subtitle{CDL - Cursul 9}
\institute[ROSEdu]{ROSEdu}
\author[Victor]{Victor Cărbune \and Laura Vasilescu \\ {\small victor@rosedu.org \and laura@rosedu.org }}

\begin{document}

% Slide-urile cu mai multe părţi sunt marcate cu textul (cont.)
\setbeamertemplate{frametitle continuation}[from second]
% Arătăm numărul frame-ului
\setbeamertemplate{footline}[frame number]

\frame{\titlepage}

\begin{frame}
\tableofcontents
\end{frame}

% NB: Secţiunile nu sunt marcate vizual, ci doar apar în cuprins.
\section{CDL}

% Pentru reamintirea periodică a cuprinsului şi unde ne aflăm:
% \frame{\tableofcontents[currentsection]} - nu mai sunt utile în template-ul
% nou

% Titlul unui frame se specifică fie în acolade, imediat după \begin{frame},
% fie folosind \frametitle
\begin{frame}{Overview}
  \begin{itemize} % Just like normal LaTeX
  \pause
  \item 9 cursuri, 18 studenti
  \pause
  \item Prezentări, laboratoare, prezentări, proiecte, prezentări, liste, IRC,
  prezentări, prezentări, prezentări
  \pause
  \item New stuff! (OOP, SVN, GIT, Wiki, Design Patterns, Redmine, Unit testing,
  STL)
  \pause
  \item Prezentări Adrian Ber, Octavian Purdilă, Mircea Mitu, Iulian
  Șerbănoiu, Andrei Pitiș, Bogdan Iancu, Radu Rendec, Dragoș Mănac
  \end{itemize}
\end{frame}

\begin{frame}{Proiecte}
  \begin{itemize} % Just like normal LaTeX
  \pause
  \item Proiecte actuale în industrie 
  \pause
  \item Dezvoltate, având comunități cu experiență în spate 
  \pause
  \item Experiență, inovație și rezultate, credem noi, deosebite!
  \pause
  \item Succes?
  \begin{itemize}
      \pause
      \item 1 Mentor sometimes MIA (missing in action). În rest, OK!
      \item Feedback
  \end{itemize}
  \end{itemize}
\end{frame}

\section{What next?}

\begin{frame}{EOC(End of CDL)}
  \begin{itemize} % Just like normal LaTeX
  \pause
  \item Is it over?
  \pause
  \item Really?
  \pause
  \item Implicați-vă în comunități!
  \pause
  \end{itemize}
\end{frame}

\begin{frame}{Why?}
  \begin{itemize} % Just like normal LaTeX
  \pause
  \item Acum credem că ne puteți spune chiar voi de ce!
  \pause
  \item Comunități!
  \item Social networking
  \item Evenimente și proiecte
  \item Oameni deosebiți
  \item Income! (Training, proiecte open source în cadrul companiilor)
  \item Summer Coding finanțat pentru studenți contributori în OS: Ruby, Red Hat, Google, etc.
  \end{itemize}
\end{frame}

\begin{frame}{Where?}
  \begin{itemize} % Just like normal LaTeX
  \pause
  \item ROSEdu (ROSEdu Summer of Code 2010, CDL 2011)
  \pause
  \item Începând, eventual, cu proiecte mai mici 
  \pause
  \item Conferințe (ROSDev, eLiberatica)
  \pause
  \item Concursuri (Best Linux Application, ProInfo / LVLE și altele, worldwide)
  \pause
  \item Proiecte proprii?
  \end{itemize}
\end{frame}

\section {Echipa CDL}
\begin{frame} Echipa
  \begin{itemize}
  \pause
  \item Chief MIA: Razvan Deaconescu!
  \pause 
  \item Vlad Dogaru
  \item Alex Eftimie
  \item Andrada Georgescu
  \item Alex Juncu
  \item Andrei Maruseac
  \item Mihai Maruseac
  \item Adrian Scoică
  \item Laura Vasilescu
  \item Vlad Voicu
  \pause
  \end{itemize}
\end{frame}
\end{document}
