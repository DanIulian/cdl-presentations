% vim: set tw=78 aw sw=2 sts=2 noet:
\documentclass{beamer}

%\includeonlyframes{c} -- speeding up compilation speed during debug

\usepackage[utf8x]{inputenc} % diacritice
\usepackage[romanian]{babel}
\usepackage{hyperref}        % folositi \url{http://...}
\mode<presentation>
\usetheme{CDL}

\title[]{Pidgin}
\subtitle{Proiect CDL 2012}
\institute[]{ROSEdu}
\author[]{
  Mihai Maruseac \texttt{mihai@rosedu.org} \\
  Valentina Manea \texttt{valentina@rosedu.org}
}

\setbeamertemplate{frametitle continuation}[from second]
\setbeamertemplate{footline}[frame number]

\pgfdeclareimage[height=1cm, width=1cm]{yah}{img/yah}
\pgfdeclareimage[height=1cm, width=1cm]{pig}{img/pig}

\begin{document}

\maketitle

\begin{frame}{Popular IM Clients}
  \begin{columns}[c]
    \column{1.5in}
      \pgfputat{\pgfxy(1,.5)}{\pgfbox[left,base]{\pgfuseimage{yah}}}
      \uncover<2->{
      \begin{itemize}
        \item Windows
        \item web - limited
        \uncover<3->{\item only Yahoo}
        \uncover<4->{\item closed source}
        \uncover<5->{\item not extensible}
      \end{itemize}}
    \column{1.5in}
      \pgfputat{\pgfxy(1,.5)}{\pgfbox[left,base]{\pgfuseimage{pig}}}
      \uncover<2->{
      \begin{itemize}
        \item cross-platform
        \uncover<3->{\item multiple protocoale}
        \uncover<4->{\item open-source}
        \uncover<5->{\item plugins}
      \end{itemize}}
  \end{columns}
\end{frame}

\begin{frame}{Pidgin Sketch}
  \begin{itemize}
    \item Ce nu are Pidgin și are Yahoo?
    \item <2-> Photo share
    \item <0> 
    \item <0>
    \item <3->Plugin pidgin - image sharing
    \item <4->vectorial images - SVG
  \end{itemize}
\end{frame}

\begin{frame}{Detalii proiect}
  \begin{itemize}
    \item C
    \item template standard
    \item comentarii, documentare puțină
    \item cod ușor de înțeles
    \item <2-> Yahoo, GTalk, ...
    \item <3-> protocol separat sau integrat în protocoale existente
    \item <4-> grafică vectorială - SVG sau derivate
    \item <5-> grafice funcții, desene simple
    \item <6-> extindere posibilă pentru orice imagine (orice format)
  \end{itemize}
\end{frame}

\begin{frame}{Cerințe și așteptări}
  \begin{itemize}
    \item C
    \item implementare protocoale comunicații
    \item unit testing
    \item team work
    \item coding style
  \end{itemize}
\end{frame}

\begin{frame}{Timeline aproximativ}
  \begin{itemize}[<+->]
    \item curs 3 - documentare trimitere mesaje prin plugins (gRIM, Psychic
    Mode)
    \item curs 4 - plugin de test - trimitere mesaje simple; design format
    imagine
    \item curs 5 - transmitere imagini simple; schiță UI
    \item curs 6 - complicare imagini; terminare UI
    \item curs 7 - adăugare expresii și funcții
    \item curs 8 - testare pe diverse protocoale, diverse platforme
    \item curs 9 - plugin complet funcțional; trimitere spre comunitate
    \item curs 10 - rezolvare buguri semnalate de comunitate
  \end{itemize}
\end{frame}

\end{document}
