% vim: set tw=78 aw sw=2 sts=2 noet:
\documentclass{beamer}

\usepackage[utf8x]{inputenc} % diacritice
\usepackage[romanian]{babel}

\usepackage{color}			 % highlight
\usepackage{alltt}			 % highlight
\usepackage{code/highlight}

\usepackage{hyperref}        % folositi \url{http://...}
% sau \href{http://...}{Nume Link}
\mode<presentation>
\usetheme{CDL}

% Titlul nu foloseşte Unicode pentru că e o problemă căreia nu i-am dat de
% cap.
\title[]{Django}
\subtitle{CDL - Cursul 3}
\institute[]{ROSEdu}
\author[]{Sergiu Iordache \\ \texttt{sergiu@rosedu.org}}

\setbeamertemplate{frametitle continuation}[from second]
\setbeamertemplate{footline}[frame number]

\begin{document}

\maketitle

\tableofcontents

\begin{frame}{Introducere in Django}
  \begin{itemize}
    \item 
    \begin{quote}
    The web framework for perfectionists with deadlines.
    \end{quote} 
    \pause
    \item Framework web de nivel inalt
    \pause
    \item Excelent pentru un tip de site-uri, dar versatil
    \pause
    \item Python \pause (rules)
    \pause
    \item Documentatie excelenta
    \pause
    \item Organizare mai buna
    \pause
    \item Mai putine sarcini plictisitoare
    \pause
    \item Initial site-uri pentru stiri
  \end{itemize}
\end{frame}

\section{Structura}
\begin{frame}{Structura}
  \begin{itemize}
    \item Model View Controller (MVC)
    \pause
    \item De fapt Model View Template
    \pause
    \item Model - ORM propriu Django \pause - pluggable, se poate folosi altceva
    \pause
    \item Template - Template engine propriu \pause - corespunde cu view din MVC
    \pause
    \item View - cod, primeste request-uri, scoate response-uri
	\pause
    \item legate de definitii url-uri - expresii regulate 
  \end{itemize}
\end{frame}

\section{Cum lucram}
\begin{frame}{Cum lucram}
  \begin{itemize}
    \item \texttt{django-admin.py startproject proiect}
    \pause
    \item \texttt{./manage.py startapp aplicatie}
    \pause
    \item Facem modelul
    \pause
    \item \texttt{./manage.py syncdb}
  \end{itemize}
\end{frame}

\section{Model}
\begin{frame}{Model}
  \input{code/model}
\end{frame}

\section{View}
\begin{frame}{View}
  \input{code/view}
\end{frame}

\section{Template}
\begin{frame}{Template}
  % vim: set tw=78 aw:
\documentclass{beamer}

\usepackage[utf8x]{inputenc} % diacritice
\usepackage[romanian]{babel}
\usepackage{color}			 % highlight
\usepackage{alltt}			 % highlight
\usepackage{code/highlight}	 % highlight
\usepackage{hyperref}        % folositi \url{http://...}
                             % sau \href{http://...}{Nume Link}
\mode<presentation>
{ \usetheme{Rochester} }		% TODO: settle this

% Titlul nu foloseşte Unicode pentru că e o problemă căreia nu i-am dat de
% cap.
\title[Introducere \^in FLOSS]{Introducere \^in Free/Libre Open Source Software}
\subtitle{CDL - Cursul 1}
\institute{ROSEdu}
\author{Tibi Turbureanu \and Another Author}

\begin{document}

% Slide-urile cu mai multe părţi sunt marcate cu textul (cont.)
\setbeamertemplate{frametitle continuation}[from second]
% Arătăm numărul frame-ului
\setbeamertemplate{footline}[frame number]

\frame{\titlepage}

\frame{\tableofcontents}

% NB: Secţiunile nu sunt marcate vizual, ci doar apar în cuprins.
\section{Istorie}

% Pentru reamintirea periodică a cuprinsului şi unde ne aflăm:
\frame{\tableofcontents[currentsection]}

% Titlul unui frame se specifică fie în acolade, imediat după \begin{frame},
% fie folosind \frametitle
\begin{frame}{Preambul}
\begin{itemize} % Just like normal LaTeX
\item Prezentarea încearcă să atragă persoanele implicate în CDL spre a folosi
\LaTeX Beamer.
\item Am copiat cu neruşinare schiţele lui Tibi pentru prima parte.
\item Be sure to check the code: lots of comments and TODOs.
\item Spam the list, IRC, or Vlad (nu la telefon) for more info.
\end{itemize}
\end{frame}

\begin{frame}[allowframebreaks] % Spargem paginile mai mari automat
% Atenţie: allowframebreaks nu funcţionează cu overlay-uri
\frametitle{Câte ceva despre istoria FLOSS}
\begin{itemize}
\item Punctul de plecare: noua imprimantă de la MIT
\item Proiectul GNU, activismul lui Richard Stallman (1983)
\item Cele patru drepturi. Programele libere nu sunt neapărat gratuite
\item Minix, sistem de operare UNIX educațional. Micronucleu (1987)
\item GNU: un sistem de operare liber dar incomplet. Hurd nu este gata (1990)
\item Linux: nucleul monolitic devenit liber (1991). Primele distribuții
GNU/Linux
\item Afaceri. Companiile multinaționale devin și ele contribuitori la
programele libere
\item OSI și definiția Open Source (1998). Diferențe de perspectivă și
aplicabilitate
\item Patentele pe programe din SUA amenință programele libere. Procese (2004
- prezent)
\item Microsot, Adobe și Nokia încep să dezvolte programe Open Source, nu
toate libere (2008)
\item Echipa Debian se alătură efortului internațional de a termina
revoluționarul GNU Hurd
\end{itemize}
\end{frame}

\section{Cod}
\frame{\tableofcontents[currentsection]}

\begin{frame}
\frametitle{Exemplu de cod}
Acesta este un script bash:\\
\noindent
\ttfamily
\hlstd{}\hlline{\ \ \ \ 1\ }\hlslc{\#!/bin/bash}\\
\hlline{\ \ \ \ 2\ }\hlstd{\\
\hlline{\ \ \ \ 3\ }MAX}\hlsym{=}\hlstd{}\hlnum{10}\\
\hlline{\ \ \ \ 4\ }\hlstd{}\\
\hlline{\ \ \ \ 5\ }\hlkwa{for\ }\hlstd{i\ }\hlkwa{in\ }\hlstd{\$}\hlsym{(}\hlstd{}\hlkwc{seq\ }\hlstd{}\hlnum{1\ }\hlstd{}\hlkwd{\$\{MAX\}}\hlstd{}\hlsym{);\ }\hlstd{}\hlkwa{do}\\
\hlline{\ \ \ \ 6\ }\hlstd{}\hlstd{\ \ \ \ \ \ \ \ }\hlstd{}\hlkwb{echo\ }\hlstd{}\hlstr{"Testing\ \$\{i\}"}\hlstd{}\\
\hlline{\ \ \ \ 7\ }\hlkwa{done}\\
\hlline{\ \ \ \ 8\ }\hlstd{}\\
\hlline{\ \ \ \ 9\ }\hlkwb{exit\ }\hlstd{}\hlnum{0}\hlstd{}\\
\mbox{}
\normalfont
 % includem codul
\end{frame}

\begin{frame}{Încă un exemplu de cod}
Acesta este un program C:

\noindent
\ttfamily
\hlstd{}\hlline{\ \ \ \ 1\ }\hldir{\#include\ $<$stdio.h$>$}\\
\hlline{\ \ \ \ 2\ }\hlstd{}\hldir{\#include\ $<$stdlib.h$>$}\\
\hlline{\ \ \ \ 3\ }\hlstd{}\\
\hlline{\ \ \ \ 4\ }\hldir{\#define\ WHATEVER\ 1}\\
\hlline{\ \ \ \ 5\ }\hlstd{}\\
\hlline{\ \ \ \ 6\ }\hlkwb{int}\\
\hlline{\ \ \ \ 7\ }\hlstd{}\hlkwd{main}\hlstd{}\hlsym{(}\hlstd{}\hlkwb{int\ }\hlstd{argc}\hlsym{,\ }\hlstd{}\hlkwb{char\ }\hlstd{}\hlsym{{*}{*}}\hlstd{argv}\hlsym{)}\\
\hlline{\ \ \ \ 8\ }\hlstd{}\hlsym{\{}\\
\hlline{\ \ \ \ 9\ }\hlstd{}\hlstd{\ \ \ \ \ \ \ \ }\hlstd{}\hlkwd{puts}\hlstd{}\hlsym{(}\hlstd{}\hlstr{"Hello,\ World!}\hlesc{$\backslash$n}\hlstr{"}\hlstd{}\hlsym{);}\\
\hlline{\ \ \ 10\ }\hlstd{\\
\hlline{\ \ \ 11\ }}\hlstd{\ \ \ \ \ \ \ \ }\hlstd{}\hlkwa{return\ }\hlstd{}\hlnum{0}\hlstd{}\hlsym{;}\\
\hlline{\ \ \ 12\ }\hlstd{}\hlsym{\}}\hlstd{}\\
\mbox{}
\normalfont

\end{frame}

\section{Overlay-uri}
\frame{\tableofcontents[currentsection]}

\begin{frame}{Comanda `pause'}
`Pause' ne lasă să arătăm textul `gradual'.

\pause Putem să folosim pause cu itemize sau orice altă facilitate.
\begin{itemize}
\pause \item De exemplu.
\pause \item Încă un exemplu.
\pause \item Dar nu e foarte flexibil...
\end{itemize}
\end{frame}

\begin{frame}{Overlay-uri mai avansate}
\only<1>{Cue chapter 9.3 from the Beamer manual.}
\only<2>{Facem o mare ciorbă de overlay-uri. \textbf{Notă:} E vorba de un
singur slide -- numărul şi titlul se păstrează.}
\begin{itemize}
\item<3-> orice comandă poate fi urmată (fără spaţiu) de un specificator de
overlay;
\item<4-> specificatoarele sunt marcate cu \texttt{<} şi \texttt{>};
\item<5-> \texttt{<4>} înseamnă ``doar pe overlay-ul 4'';
\item<5-> \texttt{<1-3>} înseamnă ``doar pe overlay-urile de la 1 până la 3,
inclusiv'';
\item<6-> \texttt{<5->} înseamnă ``de la overlay-ul 5 până la sfârşit'' (de
fapt, până la slide-ul următor, fireşte);
\item<6-> \texttt{<-4>} înseamnă ``de la început până la overlay-ul 4'';
\item<7-> Toate exemplele de mai sus au avut comanda \texttt{item} urmată de
un overlay. Trebuie să ţinem cont de ordine manual, dar nu cred că vom avea
nevoie de cine-ştie-ce efecte.
\end{itemize}
\end{frame}


\end{document}

\end{frame}

\section{CRUD}
\begin{frame}{CRUD}
  \begin{itemize}
    \item Create Read Update Delete
    \pause
    \item \textbf{booooring}
    \pause
    \item Generat automat din modele prin introspectie 
    \pause
    \item Se poate modifica pentru a se potrivi nevoilor noastre
  \end{itemize}
\end{frame}

\section{Altele}
\begin{frame}{Baterii incluse}
  \begin{itemize}
    \item Autenitifcare
    \item i18n
    \item sesiuni
    \item pagini statice
    \item cacheing
    \item protectie Cross Site Forgery
    \item si multe altele\ldots 
  \end{itemize}
\end{frame}

\section{Links}
\begin{frame}{Links}
  \begin{itemize}
    \item \url{http://djangoproject.com/}
    \item \url{http://docs.djangoproject.com/en/1.1/}
    \item \url{http://docs.djangoproject.com/en/1.1/intro/tutorial01/}
    \item Google
  \end{itemize}
\end{frame}

\end{document}
