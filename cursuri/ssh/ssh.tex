% vim: set tw=78 aw:
\documentclass{beamer}

\usepackage[utf8x]{inputenc} % diacritice
\usepackage[romanian]{babel}
\usepackage{hyperref}        % folositi \url{http://...}
% sau \href{http://...}{Nume Link}
\mode<presentation>
{ \usetheme{Rochester} }		% TODO: settle this

% Titlul nu foloseşte Unicode pentru că e o problemă căreia nu i-am dat de
% cap.
\title[Secure Shell]{Secure Shell}
\subtitle{CDL - Cursul 5}
\institute{ROSEdu}
\author{Răzvan Deaconescu}

\begin{document}

% Slide-urile cu mai multe părţi sunt marcate cu textul (cont.)
\setbeamertemplate{frametitle continuation}[from second]
% Arătăm numărul frame-ului
\setbeamertemplate{footline}[frame number]

\frame{\titlepage}

\frame{\tableofcontents}

% NB: Secţiunile nu sunt marcate vizual, ci doar apar în cuprins.
\section{No\c{t}iuni generale}

% Pentru reamintirea periodică a cuprinsului şi unde ne aflăm:
\frame{\tableofcontents[currentsection]}

% Titlul unui frame se specifică fie în acolade, imediat după \begin{frame},
% fie folosind \frametitle
\begin{frame}{SSH}
  \begin{itemize}
    \item \textbf{protocol} de rețea (nivel aplicație)
    \item canal securizat de comunicație
    \item comandă de la distanță, transfer de fișiere, tunelare
    \item client (ssh), server (sshd, portul 22)
    \item servere pe Unix-uri, Mac OS X, Windows (cygwin)
    \item \href{http://www.openssh.com/}{OpenSSH} este practic singura
implementare de SSH
  \end{itemize}
\end{frame}

\begin{frame}{Criptare}
  \begin{itemize}
    \item două tipuri
      \begin{itemize}
        \item chei simetrice - rapidă
        \item chei asimetrice - lentă dar potențial mai sigură
      \end{itemize}
    \item algoritmi de criptare
       \begin{itemize}
         \item alg(mesaj) $\rightarrow$ mesaj\_criptat
         \item AES, IDEA, Twofish
         \item RSA, DSA
       \end{itemize}
  \end{itemize}
\end{frame}

\begin{frame}{Chei publice/private}
  \begin{itemize}
    \item legătură matematică
    \item generate în același timp
    \item pub(priv(M)) = M (semnătură digitală)
    \item priv(pub(M)) = M (criptare)
    \item autentificare (certificate)
    \item criptare folosită în conjuncție cu chei simetrice
  \end{itemize}
\end{frame}

\section{Comenzi de bază}
\frame{\tableofcontents[currentsection]}

\begin{frame}{Conectare la distanță}
  \begin{itemize}
    \item ssh ana@cdl.rosedu.org
    \item ssh -l bogdan cdl.rosedu.org
    \item ssh -p 2222 -l corina cdl.rosedu.org
    \item ssh -l dan cdl.rosedu.org "ls \~{}/projects"
    \item ssh -i .ssh/servers.priv -l root cdl.rosedu.org
    \item man ssh
  \end{itemize}
\end{frame}

\begin{frame}{Copiere la/de la distanță}
  \begin{itemize}
    \item scp file.txt elena@cdl.rosedu.org:
    \item scp file.txt florin@cdl.rosedu.org:projects/elena/tmp/
    \item scp -P 2222 file.txt gabi@cdl.rosedu.org:
    \item scp horia@cdl.rosedu.org:test.txt .
    \item scp -r horia@cdl.rosedu.org:docs/ \~{}/stuff/
    \item man scp
  \end{itemize}
\end{frame}

\begin{frame}{Generare pereche de chei}
  \begin{itemize}
    \item ssh-keygen -t rsa
    \item ssh-keygen -t dsa -N My$|$P422)(
    \item implicit \{\~{}/.ssh/id\_dsa, \~{}/.ssh/id\_dsa.pub\}
    \item ssh-keygen -t rsa -f my\_key
    \item \{my\_key, my\_key.pub\}
    \item man ssh-keygen
  \end{itemize}
\end{frame}

\begin{frame}{Autentificare folosind chei publice}
  \begin{itemize}
    \pause \item copiere cheie publică la distanță (ssh cu parolă sau altfel)
      \begin{itemize}
        \pause \item scp \~{}/.ssh/id\_rsa.pub irina@cdl.rosedu.org
      \end{itemize}
    \pause \item configurare cont la distanță (irina@cdl.rosedu.org)
      \begin{itemize}
        \pause \item cat id\_rsa.pub $>>$ \~{}/.ssh/authorized\_keys
      \end{itemize}
    \pause \item conectare la distanță
      \begin{itemize}
        \pause \item ssh -l irina cdl.rosedu.org
      \end{itemize}
    \pause \item cat \~{}/.ssh/id\_rsa.pub $|$ ssh irina@cdl.rosedu.org "cat
$>>$ \~{}/.ssh/authorized\_keys"
  \end{itemize}
\end{frame}

\section{No\c{t}iuni avansate}
\frame{\tableofcontents[currentsection]}

\begin{frame}{Instalare și interacțiune server SSH}
  \begin{itemize}
    \item apt-get install openssh-server
    \item /etc/init.d/ssh \{stop, start, restart\}
  \end{itemize}
\end{frame}

\begin{frame}{Configurări de bază server SSH}
  \begin{itemize}
    \item /etc/ssh/sshd\_config
    \item Port 2222
    \item PasswordAuthentication no
    \item X11Forwarding yes
    \item man sshd
  \end{itemize}
\end{frame}

\begin{frame}{Folosire ssh-agent}
  \begin{itemize}
    \item mai multe perechi de chei $\rightarrow$ folosire opțiune -i la ssh/scp
    \item chei cu passphrase $\rightarrow$ introducerea passphrase-ului la fiecare conectare
    \item booooring (lene, pierdere de timp, rutină)
    \item ssh-add    ; se adaugă cheia implicită
    \item ssh-add \~{}/.ssh/my\_key    ; se adaugă cheia specificată
    \item la autentificare nu se mai cere passphrase; nu mai e nevoie de -i
    \item pornit automat cu interfața grafică
    \item \url{http://www.gentoo.org/proj/en/keychain/}
  \end{itemize}
\end{frame}

\begin{frame}{TCP Port Forwarding}
  \begin{itemize}
    \item ssh -L 8080:anaconda.cs.pub.ro:80 -l razvan anaconda.cs.pub.ro
      \begin{itemize}
        \item conexiune pe localhost:8080 sunt dirijate către
anaconda.cs.pub.ro:80
      \end{itemize}
    \item ssh -R 8022:localhost:22 -l razvan anaconda.cs.pub.ro
       \begin{itemize}
         \item ssh -p 8022 razvan@localhost ; de pe anaconda.cs.pub.ro
         \item conexiunile ajung la sistemul care a inițiat reverse port
       \end{itemize}
forward
  \end{itemize}
\end{frame}

\begin{frame}{X-Forwarding cu SSH}
  \begin{itemize}
    \item X11Fowarding on ; /etc/ssh/sshd\_config
    \item ssh -X -l lucian cdl.rosedu.org
    \item xeyes, firefox, gvim etc.
    \item rulare la distanță, afișare locală
  \end{itemize}
\end{frame}

\section{Utilizare SSH \^{i}n alte aplicații}
\frame{\tableofcontents[currentsection]}

\begin{frame}{SCM}
  \begin{itemize}
    \item git clone ssh://cdl.rosedu.org/projects/repo.git/
    \item svn checkout svn+ssh//cdl.rosedu.org/projects/repo.svn/
  \end{itemize}
\end{frame}

\begin{frame}{rsync}
  \begin{itemize}
    \item rsync -avz -e ssh monica@cdl.rosedu.org:/home/cdl/projects
\~{}/rsync/
  \end{itemize}
\end{frame}

\begin{frame}{Bibliografie}
  \begin{itemize}
    \item SSH: The Secure Shell (The Definitive Guide), O'Reilly 2005 (2nd
Edition)
    \item \url{http://www.linuxjournal.com/article/4412}
    \item \url{http://suso.org/docs/shell/ssh.sdf}
edition)
  \end{itemize}
\end{frame}

\end{document}
