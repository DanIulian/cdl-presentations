% vim: set tw=78 sts=2 sw=2 ts=8 aw et:
\documentclass{beamer}

\usepackage[utf8x]{inputenc}		% diacritice
\usepackage[romanian]{babel}
%\usepackage{color}			% highlight
%\usepackage{alltt}			% highlight
%\usepackage{code/highlight}		% highlight
\usepackage{hyperref}			% folosiți \url{http://...}
                                        % sau \href{http://...}{Nume Link}
\usepackage{verbatim}
\usepackage{tabularx}
\usepackage{booktabs}

\mode<presentation>
\usetheme{CDL}

% Show contents at every section beginning. Ripped off from manual.
\AtBeginSection[] % Do nothing for \section*
{
  \begin{frame}<beamer>
    \frametitle{Outline}
  \tableofcontents[currentsection]
    \end{frame}
}

\renewcommand{\arraystretch}{1.3}

% Titlul nu foloseşte Unicode pentru că e o problemă căreia nu i-am dat de
% cap.
\title[Redmine]{Redmine -- Web-based Software Project Management}
\subtitle{Wiki, Issues, Repository Access}
\institute[ROSEdu]{ROSEdu}
\date{19 februarie 2011}
\author{Răzvan Deaconescu \\ \texttt{razvan@rosedu.org}}

\begin{document}

% Slide-urile cu mai multe părți sunt marcate cu textul (cont.)
\setbeamertemplate{frametitle continuation}[from second]
% Arătăm numărul frame-ului
\setbeamertemplate{footline}[frame number]

\frame{\titlepage}

\frame{\tableofcontents}

\section{Redmine}

\begin{frame}{Web-base Software Project Management}
  \begin{itemize}
    \item gestiunea proiectelor sofware
    \item de obicei wiki, issue/ticket tracker, roadmap, acces la repository
    \item colaborare, organizare, history
    \item client-server: Trac, Redmine
    \item hosted: SourceForge, BerliOS, Savannah, Google Code
    \item GitHub, Gitorious
  \end{itemize}
\end{frame}

\begin{frame}{De ce Redmine?}
  \begin{itemize}
    \item web-based software project management system
    \item repository, wiki, issues, files, documents, forum, activity,
calendar etc.
    \item proiecte multiple în cadrul aceleiași instanțe
    \item gestiunea utilizatorilor
    \item plugin-uri
  \end{itemize}
\end{frame}

\section{Wiki-uri}

\begin{frame}{De ce wiki-uri?}
  \begin{itemize}
    \item editare colaborativă
    \item formatare facilă, link-uri facile
    \item easy to install, use, edit (no training)
    \item revision control
    \item knowledge base (tutoriale, informații utile, documentare)
  \end{itemize}
\end{frame}

\begin{frame}{Ce este un wiki?}
  \begin{itemize}
    \item site web, editare facilă
    \item limbaj markup simplificat, vizualizare rapidă a informațiilor
    \item wiki software (wiki engine) -- aplicație ce rulează wiki
    \item suport pentru colaborare
    \item istoric, revizie, autentificare, ACL
    \item Ward Cunningham: ``the simplest online database that could possibly
work''
  \end{itemize}
\end{frame}

\begin{frame}{Wiki versus \ldots}
  \begin{itemize}
    \item MS Office / OpenOffice
      \begin{itemize}
        \item (wiki) lucru colaborativ, interfață facilă, web
        \item (office) printer friendly, complex, acces controlat
    (non-professional)
      \end{itemize}
    \item fișiere LaTeX într-un repository
      \begin{itemize}
        \item (wiki) ușor de editat, feedback imediat, WYIWYG, web
        \item (office) printer friendly, profesionist
      \end{itemize}
    \item CMS
      \begin{itemize}
        \item (wiki) ușor de editat, lucru colaborativ facil
        \item (CMS) aspect și prezentare
      \end{itemize}
    \item Google Docs
      \begin{itemize}
        \item (wiki) open to public, web-friendly (link-uri etc.)
        \item (gdocs) gestiune mai bună a accesului, apropiat Office
      \end{itemize}
  \end{itemize}
\end{frame}

\begin{frame}{Cazuri de utilizare}
  \begin{itemize}
    \item knowledge base (comunitate, companie, proiect)
    \item colaborare, editare colaborativă
    \item tutoriale
    \item publicare rapidă de conținut (curbă de învățare redusă)
  \end{itemize}
\end{frame}

\begin{frame}{Exemple cunoscute}
  \begin{itemize}
    \item \url{http://www.wikimatrix.org/}
    \item MediaWiki
    \item DokuWiki
    \item TWiki
    \item MoinMoin
    \item PhpWiki
    \item PmWiki
    \item integrate în alte aplicații/site-uri
  \end{itemize}
\end{frame}

\section{Redmine Wiki}

\begin{frame}{Sintaxă Redmine wiki}
  \begin{itemize}
    \item \url{http://www.redmine.org/wiki/1/RedmineTextFormatting}
    \item pentru crearea unei pagini se creează un link către o pagină și apoi
se accesează
    \item există o pagină de ajutor cu sintaxa wiki-ului disponibilă în
momentul editării
    \item pe instanța CDL se folosește un plugin Creole pentru sintaxa Creole
      \begin{itemize}
        \item \url{http://www.wikicreole.org/wiki/Creole1.0}
      \end{itemize}
  \end{itemize}
\end{frame}

\begin{frame}{Comparison chart}
  \begin{center}
    \begin{tabular}{@{}lcc@{}}
      \toprule
      \textbf{Format} & \textbf{Creole} & \textbf{Redmine nativ (Textile)} \\
      \midrule
      heading 1 & = Nume & .h1 Nume \\
      bold & **text** & *text* \\
      italic & //text// & \_text\_ \\
      link la pagină & [[PageName$|$Nume link]] & [[PageName$|$Nume link]] \\
      URL-uri externe & [[URL$|$Nume link]] & "Nume link":URL \\
      liste neordonate & *, ** & *, ** \\
      liste ordonate & \#, \#\# & \#, \#\# \\
      nowiki & \{\{\{ \ldots \}\}\} & .bq \ldots \\
      \bottomrule
    \end{tabular}
  \end{center}
\end{frame}

\section{Issues}

\begin{frame}{Issues}
  \begin{itemize}
    \item probleme, solicitări de rezolvare a unor probleme
    \item tip problemă, autor, asignat, stare, deadline, prioritate,
descriere
    \item tichete -- help desk, call center
    \item issue tracking system -- gestiunea issue-urilor unei
organizații, unui proiect
      \begin{itemize}
        \item aplicație software
        \item interfață web, bază de date
        \item asemănător cu un bug tracking system (bugtracker)
        \item autentificare -- în proiectele open-source submiterea de
bug-uri e deschisă
      \end{itemize}
  \end{itemize}
\end{frame}

\begin{frame}{Issue tracking systems}
  \begin{itemize}
    \item Bugzilla, MantisBT -- single purpose
    \item Trac, Redmine -- multi purpose
    \item SourceForge, Launchpad, Google Code, GitHub -- hosted
  \end{itemize}
\end{frame}

\section{Redmine Issues}

\begin{frame}{New issues}
  \begin{itemize}
    \item tracker -- tipul issue-ului (configurabil)
    \item subject
    \item description
    \item category (de configurat)
    \item due date
    \item priority
    \item assigned to
    \item attached files
  \end{itemize}
\end{frame}

\begin{frame}{View/edit issues}
  \begin{itemize}
    \item list
    \item summary
    \item Gantt chart
    \item calendar
    \item filters
    \item update
    \item delete, move, copy (admin rights)
    \item personalizare mod de afișare
  \end{itemize}
\end{frame}

\begin{frame}{Issue workflow}
  \begin{enumerate}
    \item autor/inițiator
      \begin{itemize}
        \item creare -- marcat issue ca \textbf{New}
        \item descriere
        \item (eventual) asignare -- marcat issue ca \textbf{Assigned}
      \end{itemize}
    \item assignee
      \begin{itemize}
        \item (eventual) preluare -- marcat issue ca \textbf{Assigned}
        \item lucru
        \item rezolvare -- marcat issue ca \textbf{Resolved} sau,
        \texttt{Feedback}
      \end{itemize}
    \item inițiator sau coordonator
      \begin{itemize}
        \item închidere -- marcat issue \textbf{Closed})
      \end{itemize}
    \item viewer
      \begin{itemize}
        \item listare
        \item referire (în wiki, secțiuni de descriere) -- caracterul \# (diez) urmat de numărul issue-ului
      \end{itemize}
  \end{enumerate}
\end{frame}

\begin{frame}{Alte facilități Redmine}
  \begin{itemize}
    \item repository browser
    \item news
    \item \textbf{RSS Feeds}
    \item Gantt charts, calendar
    \item plugin-uri (code review, Doodle)
    \item forumuri
  \end{itemize}
\end{frame}

\section{Concluzii}

\begin{frame}{Cuvinte cheie}
  \begin{columns}
    \begin{column}[l]{0.5\textwidth}
      \begin{itemize}
        \item wiki
        \item editare colaborativă
        \item easy to install, use, customize
        \item markup language
        \item wiki engine
        \item WikiMatrix
        \item DokuWiki
        \item sintaxă Creole
        \item Redmine wiki
      \end{itemize}
    \end{column}
    \begin{column}[l]{0.5\textwidth}
      \begin{itemize}
        \item issues
        \item issue tracking
        \item bug tracking
        \item Redmine
        \item create, view, update
      \end{itemize}
    \end{column}
  \end{columns}
\end{frame}

\begin{frame}{Link-uri utile}
  \begin{itemize}
    {\footnotesize
    \item \url{http://www.wikimatrix.org/}
    \item \url{http://www.dokuwiki.org/syntax}
    \item \url{http://www.wikicreole.org/wiki/Creole1.0}
    \item \url{http://www.redmine.org/wiki/1/RedmineTextFormatting}
    \item \url{http://en.wikipedia.org/wiki/Issue\_tracking\_system}
    \item \url{http://www.redmine.org/wiki/redmine/RedmineIssues}
    \item \url{http://www.bugzilla.org/}
    \item
\url{http://en.wikipedia.org/wiki/Comparison\_of\_issue\_tracking\_systems}
    \item \url{http://www.chiark.greenend.org.uk/~sgtatham/bugs.html}
    }
  \end{itemize}
\end{frame}

\begin{frame}{Exerciții}
  \begin{itemize}
    {\small
    \item \url{https://projects.rosedu.org/projects/cdl-2011/wiki}
    }
  \end{itemize}
\end{frame}

\section{Intrebări}

\end{document}
