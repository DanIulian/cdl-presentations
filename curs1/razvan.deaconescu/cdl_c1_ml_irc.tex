\documentclass{beamer}

\usepackage[utf8x]{inputenc}
\usepackage[romanian]{babel}
\usepackage{url}
\usepackage{color}
\usepackage{alltt}
%\usepackage{code/highlight}
\mode<presentation>
\usetheme{Rochester}

\title[Liste de discu\c{t}ii \c{s}i IRC]{Liste de discu\c{t}ii \c{s}i IRC}
\institute{ROSEdu}
\author{R\u{a}zvan Deaconescu}

\begin{document}

\setbeamertemplate{frametitle continuation}[from second]
\setbeamertemplate{footline}[framenumber]

\frame{\titlepage}

\frame{\tableofcontents}

\section{Comunicare}
    \frame{\tableofcontents[currentsection]}
    
    \begin{frame}{Comunicare}
    \begin{itemize}
    \setlength{\itemsep}{0.4cm}
    \item Informații, anunțuri, opinii, feedback, întrebări
    \item Liste de discuții
    \item Forumuri
    \item IRC
    \item Chat
    \item Conferințe (video, audio, chat)
    \item Întâlniri (live)
        \begin{itemize}
        \item formale: conferințe, workshop-uri, meeting-uri
        \item informale: hackathon, code sprint, bere :-)
        \end{itemize}
    \end{itemize}
    \end{frame}

\section{Liste de discuții}
    \frame{\tableofcontents[currentsection]}

    \begin{frame}{Liste de discuții}
    \begin{itemize}
    \setlength{\itemsep}{0.8cm}
    \item În general, principala formă de comunicare în proiectele FOSS
    \item Exemple de liste de discuții?
        \begin{itemize}
        \setlength{\itemsep}{0.4cm}
        \item<2-> Linux kernel mailing lists
        \item<3-> Debian mailing lists
        \item<4-> RedHat mailing lists
        \item<5-> Gnome, LDE, gcc, glibc, pidgin, OpenOffice, Mozilla etc.
        \end{itemize}
    \end{itemize}
    \end{frame}

    \begin{frame}{Funcționare listă de discuții}
    \begin{itemize}
    \setlength{\itemsep}{0.4cm}
    \item Înscriere (subscribe) (adresă de e-mail)
    \item Confirmare
    \item Citire/scriere de mesaje
    \item Flaming :-)
    \item Deînscriere
    \item Funcționare one-to-many
    \item Aplicații de mailing list
        \begin{itemize}
            \item Mailman (\url{http://www.gnu.org/software/mailman/index.html})
            \item Ecartis (\url{http://www.ecartis.org/})
        \end{itemize}
    \end{itemize}
    \end{frame}

    \begin{frame}{Exercițiu}
    \begin{itemize}
    \setlength{\itemsep}{0.5cm}
    \item Înscrieți-vă pe lista de discuții \url{cdl-studenti@rosedu.org}
        \begin{itemize}
        \item \url{http://rosedu.org/cgi-bin/mailman/listinfo/cdl-studenti}
        \end{itemize}
    \item Creați un filtru în clientul vostru web de e-mail pentru \\
        mesajele de la această listă
    \item Trimiteți un mesaj de salut pe listă
    \item Înscrieți-vă pe lista kernelnewbies și creați un filtru
        \begin{itemize}
        \item \url{http://kernelnewbies.org/MailingList}
        \end{itemize}
    \end{itemize}
    \end{frame}

    \begin{frame}{Liste de discuții vs. forumuri}
    \begin{itemize}
    \setlength{\itemsep}{0.6cm}
    \item Forumuri
        \begin{itemize}
        \setlength{\itemsep}{0.3cm}
        \item autentificare
        \item organizarea informației/thread-uri de discuție
        \item citești doar ce vrei
        \end{itemize}
    \item Liste de discuții
        \begin{itemize}
        \setlength{\itemsep}{0.3cm}
        \item fără operații "suplimentare"
        \item se garantează că mesajul "ajunge" la destinație
        \item se pot transmite ușor atașamente și alte fișiere
        \end{itemize}
    \end{itemize}
    \end{frame}

    \begin{frame}{Tipuri de liste de discuție}
    \begin{itemize}
    \setlength{\itemsep}{0.3cm}
    \item Development (-devel)
    \item Utilizatori (-user)
    \item Anunturi (-announce)
    \item Specializate (linux-net, debian-mips)
    \item Offtopic
    \item Humor
    \item Unele liste nu sunt asociate unui proiect; sunt folosite pentru \\
        întrebări generale
        \begin{itemize}
        \item rlug (\url{rlug@lists.lug.ro})
        \item kernelnewbies (\url{kernelnewbies@nl.linux.org})
        \end{itemize}
    \end{itemize}
    \end{frame}

    \begin{frame}{Exerciții}
    \begin{itemize}
    \setlength{\itemsep}{0.5cm}
    \item Identificați 5 liste de discuții folosite de dezvoltatorii Linux \\
        și rolul acestora
    \item Identificați 5 liste de discuții folosite de dezvoltatorii Debian \\
        și rolul acestora
    \item Identificați 5 liste de discuții folosite de dezvoltatorii Gnome \\
        și rolul acestora
    \item Identificați 5 liste de discuții folosite în diferite proiecte \\
        sau organizații din România
    \item Hint: folosiți Google
    \end{itemize}
    \end{frame}

    \begin{frame}{Netiquette}
    \begin{itemize}
    \setlength{\itemsep}{0.5cm}
    \item<2-> Scurt și la obiect
    \item<3-> Don't double-post!
    \item<4-> Lurk before you leap!
    \item<5-> Vorbește corect
    \item<6-> Fii amabil
    \item<7-> Răspunde când știi
    \item<8-> Precizează răspunsul la întrebarea proprie dacă l-ai găsit
    \end{itemize}
    \end{frame}

    \begin{frame}{Netiquette (2)}
    \begin{itemize}
    \setlength{\itemsep}{0.5cm}
    \item<2-> Folosește interleaved-posting
    \item<3-> Semnează mesajul
    \item<4-> Taie semnătura mesajului anterior
    \item<5-> Încearcă să nu provoci flame-uri (nu pune paie pe foc)
    \item<6-> Folosește mesaje text (ASCII/non-HTML)
    \item<7-> Limitează liniile mesajului la 72-80 de caractere
    \end{itemize}
    \end{frame}

\section{Internet Relay Chat}
    \begin{frame}{IRC}
    \begin{itemize}
    \setlength{\itemsep}{0.5cm}
    \item Internet Relay Chat
    \item More than "asl pls"
    \item<2-> De ce? De ce nu liste de discuții sau forumuri?
        \begin{itemize}
        \setlength{\itemsep}{0.2cm}
        \item<3-> răspunsuri în timp real
        \end{itemize}
    \item<4-> Se folosește?
        \begin{itemize}
        \setlength{\itemsep}{0.2cm}
        \item<5-> \#bash, \#linux, \#\#c, \#workingset
        \item<5-> \#rosedu, \#cdl :-)
        \end{itemize}
    \end{itemize}
    \end{frame}

    \begin{frame}{Funcționare IRC}
    \begin{itemize}
    \setlength{\itemsep}{0.5cm}
    \item IRC itself is a teleconferencing system, which (through the \\
        use of the client-server model) is well-suited to running on \\
        many machines in a distributed fashion. A typical setup involves \\
        a single process (the server) forming a central point for clients \\
        (or other servers) to connect to, performing the required message \\
        delivery/multiplexing and other functions.
    \item J. Oikarinen, D. Reed; Internet Relay Chat Protocol; RFC 1459; May 1993
    \end{itemize}
    \end{frame}

    \begin{frame}{Funcționare IRC (2)}
    \begin{itemize}
    \setlength{\itemsep}{0.5cm}
    \item Server IRC
        \begin{itemize}
        \setlength{\itemsep}{0.2cm}
        \item retransmite mesajele în cadrul unei camere de chat tuturor utilizatorilor conectați
        \item oferă participanților iluzia existenței unei camere virtuale de chat proprii
        \item lem.freenode.net, calvino.freenode.net etc.
        \end{itemize}
    \item Client IRC
        \begin{itemize}
        \setlength{\itemsep}{0.2cm}
        \item A client is anything connecting to a server that is not a server
        \item connect, disconnect, send, receive
        \end{itemize}
    \end{itemize}
    \end{frame}

    \begin{frame}{Funcționare IRC (3)}
    \begin{itemize}
    \setlength{\itemsep}{0.5cm}
    \item Canal IRC
        \begin{itemize}
        \setlength{\itemsep}{0.2cm}
        \item locație virtuală de întâlnire
        \item textbased
        \item "chatroom"
        \end{itemize}
    \item Operații
        \begin{itemize}
        \setlength{\itemsep}{0.2cm}
        \item join
        \item leave
        \item create
        \item find
        \end{itemize}
    \end{itemize}
    \end{frame}

    \begin{frame}{Funcționare IRC (4)}
    \begin{itemize}
    \setlength{\itemsep}{0.5cm}
    \item Rețele IRC (IRC networks)
        \begin{itemize}
        \item mai multe servere conectate
        \item serverele partajează canalele de IRC ale rețelei
        \item toate serverele primesc mesajele din toate camerele de chat
        \end{itemize}
    \item The Big Four
        \begin{itemize}
        \item EFnet
        \item IRCnet
        \item QuakNet
        \item Undernet
        \end{itemize}
    \item Freenode
    \end{itemize}
    \end{frame}

    \begin{frame}{Clienți IRC}
    \begin{itemize}
    \setlength{\itemsep}{0.3cm}
    \item mIRC
    \item ircll
    \item Pidgin
    \item XChat
    \item irssi
    \item Konversation
    \item ChatZilla
    \item bots - automated clients
    \end{itemize}
    \end{frame}

    \begin{frame}{Folosirea unui client IRC}
    \begin{itemize}
    \setlength{\itemsep}{0.5cm}
    \item Rulare client
    \item Alegere rețea IRC
    \item Creare cont de utilizare
    \item Conectare la server IRC din rețea
    \item Join pe diverse canale din rețea
    \item chat :-)
    \end{itemize}
    \end{frame}

    \begin{frame}{Comenzi IRC de bază}
    \begin{itemize}
    \setlength{\itemsep}{0.3cm}
    \item /msg NickServ REGISTER myshinypass my@shiny.email
    \item /msg NickServ IDENTIFY $<$password$>$
    \item /join \#\#c
    \item /part
    \item /me
    \item /ping
    \item /quit
    \item /help
    \end{itemize}
    \end{frame}

    \begin{frame}{Exerciții}
    \begin{itemize}
    \setlength{\itemsep}{0.4cm}
    \item Instalați și folosiți XChat pentru conectare la
        \begin{itemize}
        \item rețeaua Freenode
        \item serverul lem.freenode.net
        \item canalele \#rosedu, \#cdl, \#linux, \#bash
        \end{itemize}
    \item Nu uitați să vă autentificați
    \item Daca nu aveți cont creați-vă unul (/msg ...)
    \item Pe canalul \#cdl salutați pe ceilalți și spuneți-vă numele
    \item Pe canalul \#cdl transmiteți un mesaj de forma \\
          "*user vorbește mult", unde user este nick-ul vostru
    \item Configurați XChat pentru a se conecta automat la canalele \\
        de mai sus la fiecare pornire
    \end{itemize}
    \end{frame}

    \begin{frame}{Exerciții (2) :}
    \begin{itemize}
    \setlength{\itemsep}{0.5cm}
    \item Folosire ajutor
        \begin{itemize}
        \item folosiți comanda /help pentru a naviga printre comenzile \\
            posibile IRC
        \item Pe canalul \#bash folosiți bot-ul de ajutor pentru a obține \\
            ajutor referitor la redirectare
        \end{itemize}
    \item Folosire pastebin
        \begin{itemize}
        \item folosiți un site de paste în care să scrieți un program "Hello, World!"
        \item dați paste la mesaj pe \#cdl
        \end{itemize}
    \item Marcați-vă ca fiind away pe canalul \#cdl
    \item Reveniți la statutul "online"
    \end{itemize}
    \end{frame}

    \begin{frame}{IRC netiquette}
    \begin{itemize}
    \setlength{\itemsep}{0.5cm}
    \item Puneți întrebări clare
    \item Nu oferiti răspunsuri "pe pâine" - indicați documentația sau botul de documantare
    \item Folosiți un canal IRC specific problemei
        \begin{itemize}
        \item nu puneți întrebări legate de Linux sau de apeluri C non-standard pe \#\#c
        \end{itemize}
    \item Răspundeți (doar) acolo unde știți
    \end{itemize}
    \end{frame}

    \begin{frame}{Legături utile}
    \begin{itemize}
    \setlength{\itemsep}{0.5cm}
    \item \url{http://www.irchelp.org/}
    \item \url{http://en.wikipedia.org/wiki/Internet\_Relay\_Chat}
    \item \url{http://www.ircbeginner.com/ircinfo/ircc-commands.html}
    \item \url{http://www.livinginternet.com/r/r.htm}
    \item \url{http://www.faqs.org/rfcs/rfc1459.html}
    \end{itemize}
    \end{frame}

\end{document}

% vim: set tabstop=4 shiftwidth=4 expandtab:
