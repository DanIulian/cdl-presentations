% vim: set tw=78 aw:
\documentclass{beamer}

\usepackage[utf8x]{inputenc} % diacritice
\usepackage[romanian]{babel}
\usepackage{color}			 % highlight
\usepackage{alltt}			 % highlight
\usepackage{hyperref}        % folositi \url{http://...}
\mode<presentation>
\usetheme{CDL}

\title[id3lib]{GUI and smart tagger over id3lib}
\subtitle{proiect CDL}
\institute[ROSEdu]{ROSEdu}
\author[VV]{Vlad Voicu (vladv@rosedu.org)}

\begin{document}

\setbeamertemplate{frametitle continuation}[from second]
\setbeamertemplate{footline}[frame number]

\frame{\titlepage}

\section{}

\begin{frame}{Despre id3lib}
	\begin{itemize}
        \item id3lib este o bibliotecă. 
		\item se ocupă de tagurile fișierelor audio.
		\item are un API bogat și ușor de utilizat.
		\item este cross-platform.
	\end{itemize}
\end{frame}

\begin{frame}{Interesant de făcut}
  \begin{itemize}
    \item Un program care folosește util biblioteca.
  \end{itemize}
\end{frame}

\begin{frame}{Cerințe}
  \begin{itemize}
    \item Începător.
    \item POO.
	\item Team Work.
	\item Idei.
  \end{itemize}
\end{frame}
\end{document}
