% vim: set tw=78 aw:
\documentclass{beamer}

\usepackage[utf8x]{inputenc} % diacritice
\usepackage[romanian]{babel}
\usepackage{color}			 % highlight
\usepackage{alltt}			 % highlight
\usepackage{hyperref}        % folositi \url{http://...}
\mode<presentation>
\usetheme{CDL}

\title[ext3grep GUI]{ext3grep}
\subtitle{proiect CDL}
\institute[ROSEdu]{ROSEdu}
\author[Q]{Andrada Georgescu (andradaq@rosedu.org)}

\begin{document}

\setbeamertemplate{frametitle continuation}[from second]
\setbeamertemplate{footline}[frame number]

\frame{\titlepage}

\section{De ce?}
\begin{frame}{Ce face?}
	\begin{itemize}
        \item \textbf{rm -rf *}, \textbf{rm -rf .}, \textbf{rm -rf ~} and friends
        \item fișiere șterse accidental
	\end{itemize}
\end{frame}

\section{Despre ext3grep}
\begin{frame}{ext3grep}
  \begin{itemize}
    \item dezvoltat de Carlo Wood din necesitate
	\item scris în \textbf{C++}
	\item nu (prea) mai există programare similare
	\item interfața în linie comandă
	\item greoi de folosit (necesită cunoștințe despre file system)
  \end{itemize}
\end{frame}

\begin{frame}{Cum face?}
  \begin{itemize}
    \item datele șterse nu sunt suprascrise imediat
    \item scanare hard-disk în căutarea rămășițelor de date
	\item algoritmi euristici
  \end{itemize}
\end{frame}

\section{Despre ext3grepGUI}
\begin{frame}{ext3grep GUI}
 \begin{itemize}
 \item o interfață prietenoasă peste \textbf{ext3grep}
 \item bug-fixing
 \item popularizare
 \item posibilitate extindere și pentru \textbf{ext4}
 \end{itemize}
\end{frame}

\begin{frame}{Ce veți învăța?}
\begin{itemize}
 \item ce este un sistem de fișiere și cum este el organizat 
 \item dezvoltarea unei interfețe grafice
 \item \textbf{C++}, cel mai probabil
 \item ...
\end{itemize}
\end{frame}

\begin{frame}{Referințe}
\begin{itemize}
\item \url{http://www.xs4all.nl/~carlo17/howto/undelete\_ext3.html}
\item \url{http://code.google.com/p/ext3grep}
\end{itemize}
\end{frame}

\end{document}
