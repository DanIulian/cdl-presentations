% vim: set tw=78 aw:
\documentclass{beamer}

\usepackage[utf8x]{inputenc}		% diacritice
\usepackage[romanian]{babel}
\usepackage{color}			% highlight
\usepackage{alltt}			% highlight
\mode<presentation>
{ \usetheme{Rochester} }		% TODO: settle this

% Titlul nu foloseşte Unicode pentru că e o problemă căreia nu i-am dat de
% cap.
\title[Programe libere]{Programe libere. Istorie \c{s}i filosofie}
\subtitle{CDL - Cursul 1}
\institute{ROSEdu}
\author{Tibi Turbureanu \\ {\small tct@ceata.org}}

\begin{document}

% Slide-urile cu mai multe părţi sunt marcate cu textul (cont.)
\setbeamertemplate{frametitle continuation}[from second]
% Arătăm numărul frame-ului
\setbeamertemplate{footline}[frame number]

\frame{\titlepage}

\frame{\tableofcontents}

% NB: Secţiunile nu sunt marcate vizual, ci doar apar în cuprins.
\section{Istorie}

% Pentru reamintirea periodică a cuprinsului şi unde ne aflăm:
\frame{\tableofcontents[currentsection]}

% Titlul unui frame se specifică fie în acolade, imediat după \begin{frame},
% fie folosind \frametitle
\begin{frame}[allowframebreaks] % Spargem paginile mai mari automat
% Atenţie: allowframebreaks nu funcţionează cu overlay-uri
\frametitle{La început, programele erau libere}
\begin{itemize}
\item În evul întunecat al tehnologiei existau lămpile aranjate în porți logice
și tare anevoios mai era să implementezi un algoritm...
\item Între timp a apărut tranzistorul (1947)
\item Atunci omul a zis: „să fie programe!” și s-a aprins lumina pentru nopți
de lucru ;-)
\item La sfârșitul anilor '40 au început să fie scrise primele programe
pentru mașini 
\item Codul era format din instrucțiuni specifice mașinii
\item Neexistând un set standard de instrucțiuni pentru mașini, codul
nu era portabil
\item Scrierea programelor nu era deloc ușoară: trebuia ținut cont de
mașină (arhitectură, set de instrucțiuni), care de obicei era una singură
pentru un laborator de dezvoltare
\item Nu exista termenul de gestionare a activităților („job scheduling”),
doar cartele perforate, imprimante și... exercițiu fizic :-D
\item Apăreau des mașini noi necompatibile cu cele anterioare. Codul trebuia
rescris
\item În anii '50 apar primele limbaje de programare de nivel înalt: FORTRAN,
COBOL, ALGOL
\item Programele (cu codul sursă) sunt distribuite la pachet cu mașinile
\item Utilizatorii (clienții) le puteau modifica pentru a le adapta nevoilor
\item La început, programele erau libere :-)
\end{itemize}
\end{frame}

\begin{frame}[allowframebreaks] % Spargem paginile mai mari automat
\frametitle{Cultura hackerilor MIT}
\begin{itemize}

\item Hacker (prezent) = „cocalar” care fură exploatând găurile de securitate din
sistemele informatice; deseori găurile sunt de natură socială :-(
\item Hacker (sens MIT) = persoană care găsește atractivă explorarea codului
unui program și extinderea capacităților lui prin modificare (hack), spre
deosebire de restul utilizatorilor care preferă utilizarea lui minimală
\item În anii '60 ia naștere cultura hackerilor la MIT, pe un sol fertil:
programele sunt libere
\item În anii '70, Richard Matthew Stallman (rms) lucrează împreună cu alți
hackeri în cadrul Laboratorului MIT de Inteligență Artificială la proiecte ca
Emacs și sistemul de operare pentru mașina Lisp
\item Hack-uiește imprimanta ca să poată anunța utilizatorii întregului laborator
când este liberă pentru o altă activitate
\item Nu poate face același lucru cu noua imprimantă laser Xerox 9700, pentru
că driverul este proprietar
\item Această experiență îi schimbă viața. Realizează că libertatea
utilizatorului începe să fie pusă în pericol
\item La sfârșitul anilor '70 și începutul anilor '80 tot mai multe companii
decid să se protejeze contra concurenței prin impunerea unor licențe
restrictive pentru utilizatorii (clienții) lor
\item Dreptul de autor începe să se manifeste agresiv. Cultura hackerilor
intră în declin, mulți pleacă de la Laboratorul MIT și încep să scrie și să
distribuie la rândul lor, programe proprietare
\end{itemize}
\end{frame}


\section{Filosofie}
\frame{\tableofcontents[currentsection]}

\begin{frame}[allowframebreaks] % Spargem paginile mai mari automat
\frametitle{Proiectul GNU}
\begin{itemize}

\item RMS crede că este dreptul utilizatorilor de programe să aibă libertatea
să-și poată ajuta prietenii, să studieze și să modifice codul sursă 
după necesități
\item Programele propietare sunt astfel „anti-sociale” și „neetice”
\item În septembrie 1983, RMS anunță demararea proiectului GNU - pentru crearea
unui sistem de operare complet liber
\item În 1984 RMS renunță la slujba de la MIT și se dedică în totalitate
proiectului GNU
\item Arhitectura sistemului a fost aleasă să fie asemănătoare UNIX, acesta
fiind sistemul cel mai popular în mediile universitare
\item GNU = GNU's Not Unix pentru a face diferență dintre GNU și UNIX-ul
proprietar. GNU nu conține cod UNIX
\item În cadrul proiectului sunt dezvoltate componentele de bază ale
sistemului: gcc, binutils, bash, glibc, coreutils și sunt testate pe sisteme
UNIX
\end{itemize}
\end{frame}

\begin{frame}[allowframebreaks] % Spargem paginile mai mari automat
\frametitle{Drepturile utilizatorilor}
\begin{itemize}

\item Dreptul 0: Să \textbf{rulezi} oricum programul
\item Dreptul 1: Să \textbf{studiezi} și să \textbf{modifici} programul
\item Dreptul 2: Să \textbf{copiezi} programul ca să-ți ajuți prietenul
\item Dreptul 3: Să \textbf{îmbunătățești} programul și să-l
\textbf{distribui} îmbunătățit
\end{itemize}
\end{frame}

\begin{frame}[allowframebreaks] % Spargem paginile mai mari automat
\frametitle{Precizări}
\begin{itemize}

\item „Free software as in freedom, not as in free beer”
\item „Free software is a matter of freedom, not price”
\item Programele libere sunt mai ieftine sau gratuite pentru că piața mondială
încă nu este liberă; monopolul și restricțiile fac în continuare regula
\item Încălcările drepturilor utilizatorilor garantate prin licențe libere
sunt tratate de FSF (Free Software Foundation, RMS) și SFLC (Software Freedom Law
Center, Elben Moglen)
\item În cazul programelor libere, termenii clasici „dezvoltator”,
„utilizator” și „distribuitor” devin foarte apropiați
\item Contrar programelor proprietare, cele libere se pretează bine
necesităților utilizatorilor într-o eră a schimbului de informație (Internet)
\end{itemize}
\end{frame}

\begin{frame}[allowframebreaks] % Spargem paginile mai mari automat
\frametitle{Decizii}
\begin{itemize}

\item Pentru longevitatea proiectului, orice decizie (alegerea micronucleului,
includerea unui nou program, alegerea condițiilor unei noi licențe, lansarea ei, 
manifestările publice) trebuie să țină cont de filosofia proiectului
\item Datorită constrângerilor externe proiectului, unele decizii pot fi luate greșit
(exemplu: GNU Mach ca micronucleu pentru GNU HURD - 1987)
\item Nucleul GNU HURD nu este gata până la lansarea nucleului monolitic Linux
(1991, Linus Torvalds). Adoptarea licenței libere GPL face posibilă includerea Linux
în sistemul GNU
\end{itemize}
\end{frame}

\section{Prezent}
\frame{\tableofcontents[currentsection]}
\begin{frame}[allowframebreaks] % Spargem paginile mai mari automat
\frametitle{Contribuții}
\begin{itemize}
\item Afaceri. Companiile multinaționale devin și ele contribuitori la
programele libere (Sun, IBM, Intel, Novell, Red Hat, HP, AMD)
\item Apar distribuțiile GNU/Linux: Debian, Ubuntu, Gentoo, Red Hat, Fedora, 
OpenSuse, Slackware (nu în ordine)
\item Apar și sistemele libere FreeBSD, Debian GNU/FreeBSD,
OpenSolaris, MINIX 3 ;-)
\end{itemize}
\end{frame}

\begin{frame}[allowframebreaks] % Spargem paginile mai mari automat
\frametitle{FLOSS}
\begin{itemize}
\item Open Source Initiative și definiția Open Source (1998) - Bruce Perens,
Debian. Accentul cade pe disponibilitatea codului sursă, indiferent de condițiile 
suplimentar impuse
\item Open Source reprezintă perspectiva pragmatică asupra programelor libere
și este preferată de companii. Deschiderea surselor este considerată o
metodă eficientă de îmbunătățire a programelor, prin contribuții 
\item Open Source este un model de afaceri de succes larg adoptat
\item FLOSS = termen general și imparțial ce se referă atât la programele libere, 
cât și la cele open source
\end{itemize}
\end{frame}

\begin{frame}[allowframebreaks] % Spargem paginile mai mari automat
\frametitle{Vești bune și vești proaste}
\begin{itemize}
\item Patente = licențe pe idei de programe
\item Patentele pe programe din SUA amenință programele libere. Procese (2004
- prezent)
\item Dreptul de autor este folosit abuziv pentru a restricționa și mai mult
utilizatorii - Digital Rights Management (DRM)
\item Echipa Debian se alătură efortului internațional de a termina
revoluționarul, încă în lucru, GNU Hurd
\item Microsot, Adobe și Nokia încep să dezvolte programe Open Source, nu
toate libere
\end{itemize}
\end{frame}

\section{Bibliografie}
\frame{\tableofcontents[currentsection]}
\begin{frame}[allowframebreaks] % Spargem paginile mai mari automat
\frametitle{Referințe}
\begin{itemize}
\item \htmladdnormallink{History of software engineering}
{http://en.wikipedia.org/wiki/History\_of\_software\_engineering}
\item \htmladdnormallink{Logic gates}
{http://en.wikipedia.org/wiki/Logic\_gate}
\item \htmladdnormallink{MIT Artificial Intelligence Laboratory}
{http://en.wikipedia.org/wiki/MIT\_Artificial\_Intelligence\_Laboratory}
\item \htmladdnormallink{Haker culture}
{http://en.wikipedia.org/wiki/Hacker\_(programmer\_subculture)}
\item \htmladdnormallink{Richard Stallman} {http://en.wikipedia.org/wiki/Richard\_Stallman}
\item \htmladdnormallink{Linus Torvalds}
{http://en.wikipedia.org/wiki/Linus\_Torvalds}
\item \htmladdnormallink{Bruce Perens}
{http://en.wikipedia.org/wiki/Bruce\_Perens}
\item \htmladdnormallink{FSF}
{http://en.wikipedia.org/wiki/Free\_Software\_Foundation}
\item \htmladdnormallink{SFLC}
{http://en.wikipedia.org/wiki/Software\_Freedom\_Law\_Center}
\item \htmladdnormallink{OSI}
{http://en.wikipedia.org/wiki/Open\_Source\_Initiative}
\item \htmladdnormallink{GNU} {http://en.wikipedia.org/wiki/GNU}
\item \htmladdnormallink{GNU Binutils} {http://en.wikipedia.org/wiki/Binutils}
\item \htmladdnormallink{GNU Coreutils}
{http://en.wikipedia.org/wiki/Coreutils}
\item \htmladdnormallink{Linux} {http://en.wikipedia.org/wiki/Linux}
\item \htmladdnormallink{UNIX} {http://en.wikipedia.org/wiki/UNIX}
\item \htmladdnormallink{Freebsd} {http://en.wikipedia.org/wiki/Freebsd}
\end{itemize}
\end{frame}
\end{document}
