% vim: set tw=78 aw sw=2 sts=2 noet:
\documentclass{beamer}

\usepackage[utf8x]{inputenc} % diacritice
\usepackage[romanian]{babel}
\usepackage{color}			 % highlight
\usepackage{alltt}			 % highlight
\usepackage{code/highlight}	 % highlight
\usepackage{ulem} % not in the template; used for strikethrough (sout)
\usepackage{hyperref}        % folositi \url{http://...}
                             % sau \href{http://...}{Nume Link}
\mode<presentation>
{ \usetheme{Rochester} }		% TODO: settle this

% Titlul nu foloseşte Unicode pentru că e o problemă căreia nu i-am dat de
% cap.
\title{Editorul Vim}
\subtitle{CDL - Cursul 2}
\institute{ROSEdu}
\author{Vlad Dogaru \\ \texttt{ddvlad@rosedu.org}}

\begin{document}

% Slide-urile cu mai multe părţi sunt marcate cu textul (cont.)
\setbeamertemplate{frametitle continuation}[from second]
% Arătăm numărul frame-ului
\setbeamertemplate{footline}[frame number]

\frame{\titlepage}

\begin{frame}{Where do we stand?}
Până acum, la CDL, am învăţat:
\begin{itemize}
  \item<2-> ce este Free/Libre/Open Source Software
  \item<3-> cum să comunicăm în cadrul unui proiect
  \item<4-> cum să automatizăm procesul de compilare
  \item<5-> cum să ţinem sursele programului undeva sigur
\end{itemize}
\onslide<6>{Când învăţăm să \textit{scriem} cod?}
\end{frame}

\begin{frame}
  \frametitle{}
  \begin{center}
  {\Huge \bfseries Acum!}
  \vspace{2cm}

  \pause De fapt, azi, dar puţin mai târziu.

  \pause Acum învăţăm cum să \textit{edităm}, nu cum să scriem cod.
  \end{center}
\end{frame}

\begin{frame}{Warning!}
\begin{itemize}
  \item<2-> Morocănos
  \item<3-> Prezentare tehnică
  \item<4-> Exerciţii puţine
  \item<5-> Puneţi mâna singuri
    \begin{itemize}
    \item<6-> nu neapărat pe Vim
    \item<7-> e important să ştim bine \textit{un} editor, nu editorul
    \textit{meu}
    \end{itemize}
\end{itemize}
\end{frame}

\begin{frame}{Importanţa eficienţei}
De ce e important să scriem cod eficient:
\begin{itemize}
  \item<2-> când scriem repede, scriem mai mult
    \begin{itemize}
    \item<3-> mai mult nu înseamnă mai bine
    \item<4-> but we're sometimes ``in the zone''
    \end{itemize}
  \item<5-> dacă scriem mult şi ineficient, o să ajungem să urâm ce facem
    \begin{itemize}
    \item<6-> Da, o să ne doară mâinile. \onslide<7->{\textbf{Tare.}}
    \end{itemize}
  \item<8-> e un exerciţiu al minţii să edităm eficient
\end{itemize}
\end{frame}

\section{Introducere}
\frame{\tableofcontents[currentsection]}

\begin{frame}{De ce Vim?}
\begin{itemize}
  \item<1-> \sout{un editor text-mode este mult mai eficient decât un IDE
  grafic}
  \item<2-> \sout{vim face treaba mai bine decât emacs}
  \item<3-> contează cu ce ne obişnuim
    \begin{itemize}
    \item<4-> dar ar trebui să ne obişnuim cu ceva flexibil şi portabil
    \end{itemize}
\end{itemize}
\end{frame}

\begin{frame}{vi şi Vim}
\onslide<1-> \texttt{vi}
\begin{itemize}
  \item<1-> Bill Joy, 1976
  \item<1-> parte a standardului POSIX
\end{itemize}

\onslide<2-> Vim
\begin{itemize}
  \item<2-> clonă de \texttt{vi} (Vi iMproved)
  \item<2-> Bram Moolenaar, 1991
  \item<2-> free and open source
  \item<2-> portabil
  \item<2-> interfaţă grafică
  \item<2-> multe alte facilităţi
    %\begin{itemize}
    %\item personalizare folosind un limbaj intern
    %\item auto-complete
\end{itemize}

Cei doi termeni (\texttt{vi} şi Vim) vor fi folosiţi ca sinonime în
prezentare, deşi nu sunt.
\end{frame}

\begin{frame}{Editor modal}
Câteva moduri importante:
\begin{itemize}
  \item<1-> Normal -- implicit, aşa porneşte \texttt{vi}
    \begin{itemize}
    \item<1-> deplasare rapidă
    \end{itemize}
  \item<2-> Insert -- introducere de text
    \begin{itemize}
    \item<2-> din Normal, apăsăm \texttt{i}
    \item<2-> ne întoarcem în Normal cu \texttt{Esc}
    \item<2-> nu rămânem în Insert mai mult decât e necesar
    \end{itemize}
  \item<3-> Command -- introducere de comenzi ``avansate''
    \begin{itemize}
    \item<3-> apăsăm \texttt{:} din Normal şi ne fuge cursorul pe ultima linie
    \item<3-> ne întoarcem lansând comanda (\texttt{Enter}) sau cu
    \texttt{Esc}
    \end{itemize}
  \item<4-> Visual -- ``selectăm'' bucăţi de text
    \begin{itemize}
    \item<4-> din Normal, apăsând \texttt{v}
    \item<4-> ne întoarcem cu \texttt{Esc} sau lansând o comandă de editare
    asupra textului
    \end{itemize}
\end{itemize}
\end{frame}

\section{Editare}
\frame{\tableofcontents[currentsection]}

\begin{frame}{Cum edităm un fişier}
\begin{itemize}
  \item<1-> Lansăm Vim în execuţie cu \texttt{vim filename}
  \item<2-> Pentru a salva schimbările, \texttt{:w}
    \begin{itemize}
    \item<3-> ``save as'' -- \texttt{:w new-filename}
    \end{itemize}
  \item<4-> Ieşim din Vim cu \texttt{:q}
    \begin{itemize}
    \item<5-> ne ``răstim'' pentru a nu salva modificările: \texttt{:q!}
    \item<6-> salvăm şi închidem dintr-o singură comandă: \texttt{:wq}
    \end{itemize}
\end{itemize}

\onslide<7->\textbf{Exerciţiu:} creaţi un fişier cu numele \texttt{test-01} şi
scrieţi numele cursului la care participaţi în el. Exersaţi salvând în alt
fişier şi renunţând la unele modificări mai puţin utile la ieşirea din Vim.
\end{frame}

\begin{frame}{Basic movement and first aid}
\begin{itemize}
  \item<1-> Ne deplasăm cu \texttt{hjkl} (\textit{în modul Normal})
    \begin{itemize}
    \item<2-> \texttt{h} şi \texttt{l} -- stânga şi dreapta
    \item<3-> \texttt{j} şi \texttt{k} -- sus şi jos
    \item<4-> mai rapid -- suntem pe ``home row''
    \item<5-> \texttt{j} are un semn pe unele tastaturi
    \end{itemize}
  \item<6-> \texttt{:help} -- de departe cea mai utilă comandă Vim
    \begin{itemize}
    \item<7-> folosiţi \texttt{:q} pentru a închide fereastra pe care o
    deschide \texttt{:h}
    \end{itemize}
\end{itemize}
\end{frame}

\begin{frame}{More advanced movement}
Cum suntem eficienţi în modul Normal?
\begin{itemize}
  \item<2-> \texttt{\^} şi \texttt{\$} -- home şi end
  \item<3-> \texttt{w} şi \texttt{b} -- sărim peste un cuvânt la dreapta / la
  stânga (\textit{word / back})
  \item<4-> \texttt{gg} şi \texttt{G} -- prima / ultima linie din fişier
    \begin{itemize}
    \item<5-> \texttt{:42} sau \texttt{42G} ne duce la linia 42
    \end{itemize}
  \item<6-> \texttt{\{} şi \texttt{\}} -- paragraful precedent / paragraful
  următor
  \item<7-> \texttt{fx} şi \texttt{Fx} -- ne poziţionează pe prima apariţie a
  caracterului x de pe linia curentă, căutând spre dreapta / stânga
    \begin{itemize}
    \item<8-> \texttt{\underline{i}nt main(char argc, char **argv)} 
    \item<9-> \texttt{fc}, apoi \texttt{cw} pentru a schimba \texttt{char} în
    \texttt{int}
    \end{itemize}
  \item<10-> \texttt{\%} mă duce la perechea parantezei pe care mă aflu
\end{itemize}
\end{frame}

\begin{frame}{More advanced editing (1)}
Ştergere:
\begin{itemize}
  \item<1-> \texttt{x} şterge caracterul de sub cursor
  \item<2-> \texttt{d} şterge caracterele date de mişcarea care îl urmează
    \begin{itemize}
    \item<3-> \texttt{dl} este echivalent cu \texttt{x}
    \item<4-> \texttt{dj} şterge linia curentă şi pe cea de sub ea
    \item<5-> \texttt{d\%} şterge o pereche de paranteze
    \item<6-> \texttt{dd} şterge o linie (shortcut)
    \end{itemize}
\end{itemize}

\onslide<7->Copy and Paste:
\begin{itemize}
  \item<8-> \texttt{y} (yank) copiază caracterele date de mişcarea care îl urmează
  \item<9-> \texttt{p} (put) este echivalentul lui \textit{paste}
    \begin{itemize}
    \item<10-> \texttt{P} puts \textit{before} the character under cursor
    \end{itemize}
  \item<11-> \textbf{Atenţie:} \texttt{d} pune ce şterge în ``clipboard''
    \begin{itemize}
    \item<12-> dacă ştergem o linie goală înainte de a face \texttt{p}, tocmai
    am pierdut conţinutul ``clipboard-ului''
    \item<13-> \textbf{Soluţia:} mai multe registre: \texttt{"ry},
    \texttt{"rp}
    \end{itemize}
\end{itemize}
\end{frame}

\begin{frame}{More advanced editing (2)}
\begin{itemize}
  \item<1-> \texttt{rx} -- înlocuieşte caracterul de sub cursor cu \texttt{x}
  \item<2-> \texttt{c} (change) -- şterge caracterele de sub mişcarea
  următoare şi intră în modul Insert pentru a introduce text
    \begin{itemize}
    \item<3-> \texttt{C} -- echivalent cu \texttt{c\$}
    \item<4-> textul şters se duce în registrul implicit
    \end{itemize}
\end{itemize}
\end{frame}

\begin{frame}{Visual}
Modul Visual:
\begin{itemize}
  \item<1-> putem selecta o regiune complexă pentru a opera asupra ei
  \item<2-> \texttt{v} pentru a intra în modul Visual
  \item<3-> ne mişcăm pentru a selecta text
  \item<4-> \texttt{d}, \texttt{y}, etc. pentru a opera asupra lui şi a ieşi
  din modul Visual
\end{itemize}

\onslide<5-> Variaţiuni pe aceeaşi temă:
\begin{itemize}
  \item<6-> \texttt{V} -- Visual Line mode; selectăm numai linii întregi
  \item<7-> \texttt{Ctrl-V} -- Visual Block mode; selectăm ``un dreptunghi''
  de text
\end{itemize}
\end{frame}

\begin{frame}{Loose ends}
\begin{itemize}
  \item<1-> \texttt{u} şi \texttt{Ctrl-R} -- undo şi redo
  \item<2-> \texttt{/text} caută \texttt{text}
    \begin{itemize}
    \item<2-> \texttt{n} şi \texttt{N} -- next şi previous
    \end{itemize}
  \item<3-> \texttt{:s/old\_pattern/new\_pattern/[options]} -- substituie
  \texttt{old\_pattern} cu \texttt{new\_pattern}
    \begin{itemize}
    \item<4-> operează implicit asupra liniei curente
    \item<5-> folosim Visual pentru a selecta uşor mai multe linii
    \item<6-> sau folosim \texttt{\%} între \texttt{:} şi \texttt{s} pentru a
    selecta întregul fişier
    \item<7-> opţiunea \texttt{g} înlocuieşte toate apariţiile de pe o linie
    \item<8-> \texttt{:\%s/old/new/g} ar înlocui toate instanţele lui
    \texttt{old} cu \texttt{new}
    \end{itemize}
\end{itemize}
\end{frame}

\begin{frame}{Exerciţiu}
\begin{itemize}
  \item descărcaţi şi dezarhivaţi arhiva
  \url{http://rosedu2.cs.pub.ro/~ddvlad/vim\_example.tar.bz2}
  \item trebuie să faceţi fişierul \texttt{myfile} să fie identic cu
  \texttt{myfile.original}, folosind unele dintre comenzile de editare
  prezentate
  \item \textbf{nu rămâneţi în Insert mai mult decât este necesar}
  \item pentru a vedea cele două fişiere unul lângă altul, deschideţi
  \texttt{myfile.original}, apoi scrieţi \texttt{:vsp myfile}
  \item vă deplasaţi între cele două jumătăţi cu \texttt{Ctrl-W Ctrl-W} (da,
  de două ori)
  \item \textbf{maximizaţi terminalul înainte de a începe}
\end{itemize}
\end{frame}

\section{Opţiuni Vim}
\frame{\tableofcontents[currentsection]}

\begin{frame}{Ce sunt opţiunile}
\begin{itemize}
  \item<1-> comportamentul Vim poate fi modificat de opţiuni
  \item<2-> folosim \texttt{:set} pentru a modifica sau interoga opţiunile
  \item<3-> modificăm folosind \texttt{:set optiune=valoare}
    \begin{itemize}
    \item<4-> pentru help despre o opţiune, \texttt{:h 'opţiune'} (observaţi
    apostroafele)
    \end{itemize}
  \item<5-> aflăm valoarea unei opţiuni cu \texttt{:set optiune?}
  \item<6-> unele opţiuni sunt de tip boolean; convenţia este \texttt{:set opt}
  şi \texttt{:set noopt}
  \item<7-> foarte multe opţiuni au şi forme prescurtate
\end{itemize}
\end{frame}

\begin{frame}{Exemple de opţiuni}
\begin{itemize}
  \item<1-> \texttt{textwidth (tw)} -- la dimensiunea liniei curente de mai
  mult de \texttt{tw} caractere, linia este spartă; \texttt{tw=0} înseamnă că
  nu există limită;
  \item<2-> \texttt{filetype (ft)} -- tipul fişierului curent, aşa cum îl
  ``vede'' Vim; poate fi sursă (C, bash, etc.), Makefile, fişier de
  configurare, sau chiar mesaj e-mail;
  \item<3-> \texttt{softtabstop (sts)} şi \texttt{shiftwidth (sw)} --
  controlează cu câte spaţii se indentează codul când apăsăm pe \texttt{Tab};
  \item<4-> \texttt{autoindent (ai)} şi \texttt{cindent (cin)} -- opţiuni care
  controlează modul în care textul este (sau nu) indentat automat; încercaţi
  \texttt{cin} pentru cod C;
  \item<5-> \texttt{wildmenu} -- face tab-completion în Command să funcţioneze
  asemănător cu bash.
\end{itemize}
\end{frame}

\begin{frame}{Vimrc}
Putem automatiza şi personaliza Vim folosind fişierul vimrc.
\begin{itemize}
  \item<2-> vimrc local este în \texttt{\$HOME/.vimrc}
  \item<3-> conţine comenzi care sunt executate de Vim la începutul sesiunii
    \begin{itemize}
    \item<3-> fără prefixul \texttt{:}
    \end{itemize}
  \item<4-> putem pune comenzi \texttt{set} în vimrc
\end{itemize}
\end{frame}

\section{Editarea mai multor fişiere}
\frame{\tableofcontents[currentsection]}

\begin{frame}{Mai multe buffere}
\begin{itemize}
  \item<1-> Vim poate edita mai multe fişiere odată
  \item<2-> vedem uşor: \texttt{vim file1 file2 ... filen}
  \item<3-> \texttt{:next} şi \texttt{:Next} -- următorul fişier / fişierul
  precedent
\end{itemize}
\end{frame}

\begin{frame}{Mai multe tab-uri}
\begin{itemize}
  \item<1-> Hello, Firefox!
  \item<2-> tab-uri -- indiciu vizual permanent cu privire la fişierele deschise
  \item<3-> deschidem un tab nou cu \texttt{:tabnew filename}
  \item<4-> \texttt{gt} şi \texttt{gT} ciclează prin taburi
  \item<5-> \texttt{ngt} ne duce la al n-lea tab
  \item<6-> închidem un tab cu \texttt{:q}
  \item<7-> \texttt{:qa}, \texttt{:qa!}, \texttt{:wqa} -- operează asupra
  tutror tab-urilor
\end{itemize}
\end{frame}

\section{...and beyond}
\frame{\tableofcontents[currentsection]}

\begin{frame}{Make şi ctags}
Make:
\begin{itemize}
  \item<1-> putem rula \texttt{make} direct din Vim -- \texttt{:make}
  \item<2-> în cazul eventualelor erori, folosim \texttt{:cnext} pentru a
  merge la linia care conţine eroarea următoare
\end{itemize}

\onslide<3-> ctags:
\begin{itemize}
  \item<4-> fişier care conţine informaţii despre definirea funcţiilor C
  \item<5-> \texttt{ctags -aR *} pentru a genera fişierul
  \item<6-> opţiunea \texttt{tags} din Vim ţine numele fişierului în care
  căutăm definiţiile
  \item<7-> \texttt{:tag tagname}, \texttt{Ctrl-]}, \texttt{Ctrl-T}
\end{itemize}
\end{frame}

\begin{frame}{Modeline}
Ce problemă rezolvă un modeline?
\begin{itemize}
  \item<1-> vreau să îţi dau un fişier
    \begin{itemize}
    \item<2-> dar vreau să îl vezi aşa cum vreau eu
    \item<3-> şi ştiu că foloseşti vim :-)
    \end{itemize}
  \item<4-> un modeline conţine comenzi pe care vim trebuie să le execute când
  citeşte fişierul
  \item<5-> \texttt{/* vim: set tw=72 sts=4 sw=4 cindent: */}
  \item<6-> utilizatorul trebuie să aibă \texttt{modeline} setat
    \begin{itemize}
    \item<7-> \texttt{set modeline} e un candidat bun pentru orice vimrc
    \end{itemize}
\end{itemize}
\end{frame}

\begin{frame}{More awesome goodness}
\begin{itemize}
  \item rulăm comenzi de shell cu \texttt{:!comanda}
  \pause
  \item autocompletion
  \pause
  \item registrul \texttt{+} este clipboard-ul sistemului: \texttt{"+y} şi
  \texttt{"+p}
\end{itemize}
\end{frame}

\begin{frame}{Random Goodness for your vimrc}
\begin{itemize}
  \item<1-> \texttt{imap jj <Esc>}
  \item<2-> \texttt{autocmd FileType c 	set tw=72 cindent}
\end{itemize}
\end{frame}

\begin{frame}
  \frametitle{}
  \begin{center}
    {\Huge \bfseries Întrebări}
  \end{center}
\end{frame}

\end{document}
