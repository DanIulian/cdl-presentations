\documentclass{beamer}

\usepackage[utf8x]{inputenc} % diacritice
\usepackage[romanian]{babel}
\usepackage{color}			 % highlight
\usepackage{alltt}			 % highlight
\usepackage{highlight/highlight}	 % highlight
\usepackage{hyperref}        % folositi \url{http://...}
                             % sau \href{http://...}{Nume Link}
\mode<presentation>
{\usetheme{Rochester}}

\title[Coding Style]{Coding Style}
\subtitle{CDL - Cursul 2}
\institute{ROSEdu}
\author{Andrei Soare \\ {\footnotesize andrei.soare@rosedu.org}}

\begin{document}
\setbeamertemplate{frametitle continuation}[from second]
\setbeamertemplate{footline}[frame number]

\frame{\titlepage}

\frame{\tableofcontents}


\section{Introducere}

  \frame{\tableofcontents[currentsection]}

  \begin{frame}{De ce avem nevoie de Coding Style ?}
  Pentru ca:\\
  \vspace{0.8cm}
  \begin{itemize}
  \setlength{\itemsep}{0.3cm}
  \pause \item Vrem sa intelegem si peste o luna codul propriu
  \pause \item Vrem sa poata contribui si altcineva la proiectul nostru
  \pause \item Vrem sa putem modifica rapid codul:\begin{itemize} \item debug \item adaugare functionalitati \end{itemize}
  \pause \item Vrem ca proiectul nostru sa aiba viitor :)
  \end{itemize}
  \end{frame}

  \begin{frame}{Ce presupune Coding Style ?}
  \begin{itemize}
  \setlength{\itemsep}{0.3cm}
  \item Indentarea codului \pause
  \item Folosirea de spatii \pause
  \item Nume de variabile/functii bine alese \pause
  \item Comentarii sugestive ({\ttfamily `Ce ?'}, nu {\ttfamily `Cum ?'}) \pause
  \item Functii de marime rezonabila \pause
  \item Organizarea codului (headere, module, functii, structuri)
  \end{itemize}
  \end{frame}

  \begin{frame}{Anti-pattern}
  Design pattern aparent benefic, dar care in final este foarte ineficient
  \pause \\\vspace{0.3cm}Exemple:\vspace{0.3cm}
  \begin{itemize}
  \item Organizationale \begin{itemize} \item Design by committee \item Escalation of commitment \item Mushroom Management \end{itemize}
  \item Programare \begin{itemize} \item Accidental Complexity \item Action at a distance \item Hard Code \end{itemize}
  \item ...
  \end{itemize}
  \end{frame} 


\section{Exemple de stiluri}

  \frame{\tableofcontents[currentsection]}

  \begin{frame}[allowframebreaks]
  \frametitle{K\&R Coding Style}
  Nume de functii/variabile:\vspace{0.5cm}
  \begin{itemize}
  \setlength{\itemsep}{0.3cm}
  \item Lungimea numelui direct proportionala cu scope-ul si cu importanta
  \item locale: scurte si la obiect\\tmp, nu TemporaryVariable
  \item globale: explicative\\count\_active\_users(), nu cntusr()
  \item macrouri: LITERE\_MARI
  \end{itemize}
  \framebreak
  Indentare si spatiere:\vspace{0.5cm}
  \begin{itemize}
  \setlength{\itemsep}{0.3cm}
  \item TAB, nu spatii
  \item TAB de 8 spatii
  \item Latimea unei surse: maxim 80 de caractere
  \end{itemize}
  \framebreak
  Acolade:\\ \vspace{0.5cm} \input{highlight/kr1} \\ \framebreak \input{highlight/kr3} \\
  \framebreak
  Comentarii:\\\vspace{0.5cm}
  \begin{itemize}
  \setlength{\itemsep}{0.3cm}
  \item {\ttfamily /*  ...  */}, nu {\ttfamily //  ...}
  \item CE?, nu CUM?
  \item Nu se pun in interiorul functiilor, ci la inceputul lor\\\vspace{0.2cm}
  \end{itemize}
  \input{highlight/kr2}
  \end{frame}

  \begin{frame}[allowframebreaks]
  \frametitle{GNU Coding Style}
  \input{highlight/gnu} \\ {\footnotesize  (sursa:wikipedia)}
  \end{frame}

  \begin{frame}{Java Coding Style}
  \input{highlight/j1}
  \end{frame}


\section{Organizarea Proiectelor in C}
  
  \frame{\tableofcontents[currentsection]}

  \begin{frame}[allowframebreaks]  
  \frametitle{Organizarea Proiectelor}
  \begin{itemize}
  \setlength{\itemsep}{0.5cm}
  \item Codul prea mare $\Rightarrow$ nu incape intr-o sursa
  \item Impartirea codului in module (fisiere .c) cu functionalitate bine definita
  \item Latimea unei surse: 80
  \item Lungimea unei functii: invers proportionala cu complexitatea\\De obicei, maxim un ecran
  \framebreak
  \item Fisierele header (.h): declaratii publice de variabile, functii si macro-uri
  \item Fisierele sursa (.c): codul executabil (implementarea functiilor din header)
  \vspace{1cm}
  \item {\ttfamily static}
  \item {\ttfamily extern}
  \framebreak
  \item Fiecare header trebuie sa aiba o protectie:\\\input{highlight/header}
  \framebreak
  \item Atentie! Un proiect contine o singura functie {\ttfamily main()}\begin{itemize} \item Compilatorul produce fisiere obiect din fiecare sursa, care sunt unite de catre "linker" intr-un singur fisier output (executabil / biblioteca)\end{itemize}
  \end{itemize}
  \end{frame}

\section{Documentatie. Doxygen}

  \frame{\tableofcontents[currentsection]}

  \begin{frame}[allowframebreaks]
  \frametitle{Documentatie}
  \begin{itemize}
  \item Documentatia si codul unui proiect sunt la fel de importante
  \item De multe ori, este neglijata :( \\ $\rightarrow$ motiv comun pentru esecul proiectelor Open Source \\ \vspace{1.5cm}
  \item Vrem sa atragem contribuitori
  \item Vrem sa le fie usor sa se adapteze la codul proiectului
  \end{itemize}
  \framebreak
  Ce trebuie sa cuprinda documentatia ?\\
  \begin{itemize}
  \item Ce face ?
  \item Cum functioneaza?
  \item De ce a fost scris asa ?
  \item Ce face fiecare parte ?
  \item Unde gasesc fiecare parte ?
  \item Cine l-a scris ?
  \item Cand a fost scris ?
  \end{itemize}
  \vspace{0.5cm}
  Tool-uri care genereaza pagini de documentatie din comentariile surselor:\\
  \begin{itemize}
  \item Doxygen
  \item Javadoc
  \end{itemize}
  \end{frame}

  \begin{frame}[allowframebreaks]
  \frametitle{Doxygen}
  \begin{itemize}
  \item Extrage comentariile functiilor, claselor, structurilor de date din codul sursa si in formateaza in:\\ \begin{itemize} \item HTML \item CHM \item RTF \item PDF \item LaTeX \item manpages \end{itemize}
  \item Este necesar un anumit format al comentariilor in surse
  \end{itemize}
  \framebreak
  Exemplu comentarii:\\ \input{highlight/dox}
  \\ Cum folosim doxygen ? \\
  \begin{itemize}
  \item {\ttfamily doxygen -g <config\_file>}
  \item edit config\_file
  \item {\ttfamily doxygen <config\_file>} 
  \end{itemize}
  \framebreak
  Cum documentam sursele in format doxygen ? \\ 
  \vspace{0.5cm} 
  \input{highlight/dox2} \\
  \framebreak
  \input{highlight/dox3}
  \end{frame}

  \begin{frame}{Practice}
  \begin{itemize}
  \setlength{\itemsep}{0.5cm}
  \item Descarcati proiectul polish de pe git.rosedu.org (cdl/curs2/andrei.soare)
  \item Rearanjati-l conform unui coding style bine definit (preferabil K\&R)
  \item Faceti-i un Makefile
  \item Comentati-l in format specific Doxygen si creati \_doar\_ documentatie in format html. [hint: cautati prin fisierul de configurare]
  \item Use Vim !
  \end{itemize}
  \end{frame}

\section{Concluzie}
  
  \frame{\tableofcontents[currentsection]}
  
  \begin{frame}{Concluzie:}
  \begin{itemize}
  \item ``e mai bine prost dar consecvent prost decat bine dar in 20 de moduri''\\ \begin{flushright}-- Razvan Deaconescu \end{flushright}
  \end{itemize}
  \end{frame}

\section{Link-uri Utile}

  \frame{\tableofcontents[currentsection]}

  \begin{frame}{Link-uri utile}
  \begin{itemize}
  \setlength{\itemsep}{0.5cm}
  \item \href{http://www.cs.su.oz.au/\~scilect/tpop/handouts/Style.htm}{K\&R Coding Style}
  \item \href{http://lxr.linux.no/linux/Documentation/CodingStyle}{Linux Coding Style}
  \item \href{http://www.gnu.org/prep/standards/standards.html}{GNU Coding Style}
  \item \href{http://java.sun.com/docs/codeconv/}{Java Coding Style}
  \item \href{http://www.cs.cf.ac.uk/Dave/C/node35.html}{Project Management in C}
  \item \href{http://www.stack.nl/\~dimitri/doxygen/}{Doxygen}
  \item \href{http://en.wikipedia.org/wiki/Anti-pattern}{Anti-pattern}
  \end{itemize}
  \end{frame}

\end{document}
