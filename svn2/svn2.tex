% vim: set tw=78 aw:
\documentclass{beamer}

\usepackage[utf8x]{inputenc} % diacritice
\usepackage[romanian]{babel}
\usepackage{color}			 % highlight
\usepackage{alltt}			 % highlight
\usepackage{code/highlight}	 % highlight
\usepackage{hyperref}        % folositi \url{http://...}
                             % sau \href{http://...}{Nume Link}
\mode<presentation>
{ \usetheme{Rochester} }		% TODO: settle this

% Titlul nu foloseşte Unicode pentru că e o problemă căreia nu i-am dat de
% cap.
\title[SVN 2]{SVN - notiuni avansate}
\subtitle{CDL - Cursul 2}
\institute{ROSEdu}
\author{Lucian Adrian Grijincu \\ \texttt{lucian.grijincu@rosedu.org}}

\begin{document}

% Slide-urile cu mai multe părţi sunt marcate cu textul (cont.)
\setbeamertemplate{frametitle continuation}[from second]
% Arătăm numărul frame-ului
\setbeamertemplate{footline}[frame number]

\frame{\titlepage}

\frame{\tableofcontents}

% NB: Secţiunile nu sunt marcate vizual, ci doar apar în cuprins.
\section{Organizarea unui proiect}

% Pentru reamintirea periodică a cuprinsului şi unde ne aflăm:
\frame{\tableofcontents[currentsection]}

\begin{frame}{Noțiuni organizare proiect}
  \begin{itemize}[<+->]
  \item \textbf{trunk/} - ramura principală de dezvoltare
  \item \textbf{branches/} - ramuri adiționale de dezvoltare
  \item \textbf{tags/} - versiuni statice cu semnificație specială
    \begin{itemize}
      \item \textbf{1.0.2} - sursele în momentul în care s-a lansat versiunea 1.0.2
      \item \textbf{2.2.0-rc1} - ``release-candidate'' pentru versiunea 2.2.0 
      \item tag-urile sunt read-only - odată creat un tag nu mai poate fi modificat
    \end{itemize}
\end{itemize}
\end{frame}


\begin{frame}{Care e echivalentul în SVN?}
  \begin{itemize}[<+->]
  \item în SVN \textbf{trunk/}, \textbf{branches/} și \textbf{tags/} sunt doar niște directoare
  \item nu au o semnificație specială
  \item nu e obligatorie structura asta. Următoarea e validă: \\
    \texttt{master/      \\
      branches-linux/    \\
      branches-win32/    \\
      branches-bsd/      \\
      releases/          \\
      release-candidates/\\
      other-tags/}
  \end{itemize}
\end{frame}


\begin{frame}{Creare branch/tag în SVN}
  \begin{itemize}[<+->]
  \item crearea unui branch pornind din trunk:\\
    \texttt{svn \textbf{copy} trunk/ branches/experimental-feature}

  \item crearea unui tag pornind din trunk:\\
    \texttt{svn \textbf{copy} trunk/ tags/v1.0.0}
   
  \end{itemize}
\end{frame}

\begin{frame}{svn branch - exercițiu}
  \begin{itemize}[<+->]
  \item luați o copie a repo-ului de cdl: \texttt{svn://svn.rosedu.org/cdl\_test/}
  \item creați un branch din \textbf{trunk/} în calea \textbf{branches/\$NUME}
  \item modificați mesajul din funcția cu numele vostru și commit-eți modificările
  \end{itemize}
\end{frame}





\section{svn merge}
\frame{\tableofcontents[currentsection]}



\begin{frame}{svn merge – combinarea unor branch-uri}
  \begin{itemize}[<+->]
  \item \texttt{svn merge SRC1 SRC2 [WorkingCopyDir]}
  \item aplică diferențele între două versiuni asupra unui WorkingCopyDir (implicit ``.'')
  \item \textbf{exercițiu:} combinați branch-ul propriu cu \textbf{trunk/}
  \end{itemize}
\end{frame}


\begin{frame}{svn merge – combinarea unui branch înapoi în trunk}
  \begin{itemize}[<+->]
    \item determinăm revizia la care s-a creat branch-ul:\\
      \small{\texttt{svn log --stop-on-copy svn://server/branch}}
    \item din \textbf{trunk/} executăm:\\
      \small{\texttt{svn merge -rXXX:HEAD svn://server/branch}}
    \item rezolvare conflicte merge
    \item un ultim commit de rezolvare conflicte
  \end{itemize}
\end{frame}


\section{svn resolve}
\frame{\tableofcontents[currentsection]}

\begin{frame}{svn rezolve – conflicte}
  \begin{itemize}[<+->]
    \item conflicte apar la \textbf{svn update} și la \textbf{svn merge}
    \item svn ``decorază'' sursele cu \texttt{====}, \texttt{<<<}, \texttt{>>>>}
    \item rezolvarea se efectuează cu \textbf{svn resolve}
  \end{itemize}
\end{frame}


\begin{frame}{svn resolve – utilizare}
  \begin{itemize}[<+->]
    \item \texttt{svn resolve --accept}
      \begin{itemize}[<+->]
      \item \textbf{base} - păstrează fișierul din ultima revizie
      \item \textbf{working} - dacă s-a rezolvat manual conflictul
      \item \textbf{mine-full} - rezolvă păstrând fișierele locale așa cum erau înainte de \textbf{svn update}
      \item \textbf{theirs-full} - rezolvă folosind fișierele așa cum sunt pe server
      \end{itemize}
  \end{itemize}
\end{frame}

\end{document}
