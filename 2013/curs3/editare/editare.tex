% vim: set tw=78 aw sw=2 sts=2 noet:
\documentclass{beamer}

\usepackage[utf8x]{inputenc} % diacritice
\usepackage[romanian]{babel}
\usepackage{hyperref}        % folositi \url{http://...}
% sau \href{http://...}{Nume Link}
\mode<presentation>
\usetheme{CDL}

% Titlul nu foloseşte Unicode pentru că e o problemă căreia nu i-am dat de
% cap.
\title[]{Editarea codului surs\u{a}. Vim}
\subtitle{CDL - Cursul 3}
\institute[CDL 2013]{ROSEdu}
\author[]{Vlad Dogaru (The Master) \\ \texttt{ddvlad@rosedu.org} \\ Alex Juncu (The Apprentice)\\
\texttt{alexj@rosedu.org}}

\begin{document}

\maketitle

\tableofcontents

\begin{frame}{FFS, Why Vim?}
  \begin{itemize}
    \item pentru că asta știu eu :-)
    \pause
    \item try Emacs, seriously
    \pause
    \item nu sunt chiar atât de diferite
    \pause
    \item vreau să sar la începutul cuvântului
    \begin{itemize}
      \item vi: \texttt{b} (eventual \texttt{Esc} înainte)
      \item emacs: \texttt{Alt-b}
    \end{itemize}
    \item vreau să execut o comandă de 10 ori
    \begin{itemize}
      \item vi: \texttt{10<comandă>}
      \item emacs: \texttt{C-u10<comandă>}
    \end{itemize}
    \pause
  \end{itemize}
\end{frame}

\section{More than Notepad}
\begin{frame}{More than Notepad}
  \begin{itemize}
    \item hjkl \pause \hspace{1cm} \textbf{Use them}
    \pause
    \item modul Insert \pause \hspace{1cm} \textbf{It's evil}
    \pause
    \item advanced movement (\^{}, \$, f, F, G, gg, 10G, :10, \% ctrl-o)
    \pause
    \item comenzi compuse
    \pause
    \begin{itemize}
      \item \texttt{d} -- șterge
      \begin{itemize}
        \item ce?
        \pause
        \item contează ce urmează: \texttt{d\$}, \texttt{df(}, \texttt{di(},
          \texttt{dG}, \texttt{d\`{}a}
      \end{itemize}
    \end{itemize}
    \pause
    \item copy-pasta -- it's bad \pause \hspace{1cm} \textbf{But we have it}
    \pause \textbf{on steroids}
    \begin{itemize}
      \pause
      \item \texttt{"zyi[}, \texttt{"zp}
      \item \texttt{:help registers}
    \end{itemize}
  \end{itemize}
\end{frame}

\section{Configuration}
\begin{frame}{Configuration}
  \begin{itemize}
    \item defaults just won't cut it
    \item e dificil să scriem propriul config
    \item copiem de la alții
    \pause
    \begin{itemize}
      \item but give credit where it is due
    \end{itemize}
  \end{itemize}
  \pause
  $$ \frac{picture}{words} = 1000 $$
  \pause
But be careful!
\end{frame}

\section{Multiple files}

\begin{frame}{Buffers}
  \begin{itemize}
    \item buffer == un fișier deschis de vim
    \item nu trebuie obligatoriu să fie afișat
    \item \texttt{vim a.txt b.txt c.txt} -- se deschide \texttt{a.txt}
    \item use \texttt{:next}, \texttt{:prev} to change buffers
    \item or \texttt{:buffer N} to skip to buffer \texttt{N}
    \item \texttt{:e filename} creates a new buffer
    \item \texttt{:ls} to list all buffers
    \item \texttt{:help hidden}
  \end{itemize}
\end{frame}

\begin{frame}{Windows}
  \begin{itemize}
    \item creare: \texttt{:split} (\texttt{Ctrl-W s}), \texttt{:vsplit}
    (\texttt{Ctrl-W v})
    \item deplasare: \texttt{Ctrl-W [hjkl]}
    \item redimensionare: \texttt{Ctrl-W [+-<>]}
    \item ``maximize'': \texttt{Ctrl-W \_}
    \pause
  \end{itemize}
\end{frame}

\begin{frame}{Tabs}
  \begin{itemize}
    \item un tab == una sau mai multe ferestre
    \pause
    \begin{itemize}
      \item recomandabil mai multe
      \item usually wrong: one file, one tab
    \end{itemize}
    \pause
    \item new tab: \texttt{:tabnew}
    \item next/previous: \texttt{gt}/\texttt{gT}
    \item go to tab N: \texttt{Ngt}
  \end{itemize}
\end{frame}

\begin{frame}{Sessions}
  \begin{itemize}
    \item so I got this setup: 6 tabs, dozens of open files... the works
    \item \texttt{:mksession} $\Rightarrow$ \texttt{Session.vim}
    \item later: \texttt{vim -S}
  \end{itemize}
\end{frame}

\section{Bigger projects}
\begin{frame}{ctags/cscope}
  \begin{itemize}
    \item utilitar pentru analiza surselor C
    \item integrabil cu Vim
    \item needs a little config
    \pause
    \item \textbf{exemplu:} surse Pidgin, \texttt{pidgin/gtkblist.c}
  \end{itemize}
\end{frame}

\begin{frame}{Autocomplete}
  \begin{itemize}
    \item \texttt{Ctrl-N} și \texttt{Ctrl-P} în mod Insert
    \item nu numai pentru cod
    \item știe să se uite în headere incluse (C, C++)
    \pause
      \begin{itemize}
        \item dar cam atât
      \end{itemize}
    \pause
    \item mai multe: OmniComplete
  \end{itemize}
\end{frame}

\section{Where to next?}
\begin{frame}{Where to next?}
  Some teasers:
  \begin{itemize}
    \item editing and moving: you never know enough
    \item editing inside archives
    \item modeline
    \item search and replace -- \texttt{sed} in a can
    \item code folding
    \item macros
    \item vimscript
    \item endless plugins, e.g. the NERD Tree
    \item \url{http://vim.wikia.com/}
  \end{itemize}
\end{frame}

\end{document}
